	\begin{UseCase}{CUC1}{Gestión de Proyecto}{Gestionar los Proyectos asignados al usuario coordinador.}
		\UCitem{Versión}{1.0}
		\UCitem{Actor(es)}{Coordinador.}
		\UCitem{Propósito}{Consultar y gestionar los proyectos coordinados por el usuario.}
		\UCitem{Resumen}{Agrupa las funcionalidades de gestión de información relacionada con los proyectos.}
		\UCitem{Entradas}{Ninguna.}
		\UCitem{Salidas}{Ninguna.}
		\UCitem{Precondiciones}{Ninguna.}
		\UCitem{Postcondiciones}{Ninguna.}
		%\UCitem{Autor}{caca}
		\UCitem{Referencias}{SIDAM-BESP-P2}
		\UCitem{Tipo}{Primario.}
		\UCitem{Módulo}{Coordinación}
	\end{UseCase}
	
	% 1.- escriba solo una trayectoria principal
	% 2.- El actor es quien siempre inicia un CU
	% 3.- evite usar ``si ... entonces ...'' o ``mientras ...'' o ``para cada ...''
	% 4.- No olvide mencionar todas las verificaciones y cálculos realizados por el sistema
	% 5.- Evite mencionar palabras como: Tabla, BD, conexión, etc.
	
	\begin{UCtrayectoria}{Principal}
		\UCpaso[\UCactor] Selecciona la opción \IUbutton{Gestionar Proyectos} de la pantalla \IUref{IUMenuCoordinador}{Menú del Coordinador}.
		\UCpaso Busca los Proyectos del Usuario.\Trayref{A}
		\UCpaso Muestra la pantalla \IUref{IUGestProyectos}{Gestión de Proyectos} donde se despliegan los datos de los Proyectos y las opciones de Pre-registrar Proyecto y Registrar Proyecto.\UCExtensionPoint{CUC1.1}{Pre-registrar Proyecto}\UCExtensionPoint{CUC1.2}{Registrar Proyecto}\label{paso:CUC1MuestraDatosProyecto}
	\end{UCtrayectoria}

	\begin{UCtrayectoriaA}{A}{Resultado de busqueda vacía}{No se encontrarón registros.}
		\UCpaso Muestra mensaje (MSG-5) indicando que no se encontraron Proyectos.\ref{MSG5}
	\end{UCtrayectoriaA}

%--------- Agregar Proyecto
	\begin{UseCase}{CUC1.1}{Pre-registrar Proyecto}{El usuario Pre-registra un nuevo Proyecto.}
			\UCitem{Versión}{1.1}
			\UCitem{Actor(es)}{Coordinador y Empleado SMA.}
			\UCitem{Propósito}{Brindar un mecanismo para Preregistrar un proyecto.}
			\UCitem{Resumen}{Permite proporcionar los datos básicos para el primer paso en el proceso de registro de proyecto.}
			\UCitem{Entradas}{Datos de {\em pre-registro del Proyecto} \ref{dd:ProyectoPreregistrado}.}
			\UCitem{Salidas}{Mensaje de confirmación del proyecto preregistrado.}
			\UCitem{Precondiciones}{Ninguna.}
			\UCitem{Postcondiciones}{El Proyecto pre-registrado adquiere el estado de {\em Pendiente} hasta su validación y registro.}
			%\UCitem{Autor}{Pérez Domínguez Alberto Michel}
			\UCitem{Referencias}{SIDAM-BESP-P1}
			\UCitem{Tipo}{Secundario/Primario. Puede venir del \UCref{CUC1}}
			\UCitem{Módulo}{Coordinación}
	\end{UseCase}

	\begin{UCtrayectoria}{Principal}
			\UCpaso[\UCactor] Oprime el botón \IUbutton{Preregistrar Proyecto} de la pantalla \IUref{IUGestProyectos}{Gestión de Proyectos} o de la pantalla de consulta ciudadana.
			\UCpaso Muestra la pantalla \IUref{IUPreregistrarProyecto}{Preregistrar Proyecto}.
			\UCpaso Obtiene los datos del Usuario que inicio sesión.\Trayref{A}
			\UCpaso Coloca los datos del responsable en el formulario\ref{datosResponsable}.
			\UCpaso Deshabilita los campos de los datos del responsable del formulario.
			\UCpaso [\UCactor] Ingresa los datos del Proyecto .\label{paso:CUC1.1IngresaDatosProyecto} \Trayref{B}
			\UCpaso [\UCactor] Oprime el botón \IUbutton{Aceptar}.
			\UCpaso Verifica que se cumpla la regla de negocio \BRref{RN18}.\Trayref{C}
			\UCpaso Revisa que los datos correspondan con la Definición de Datos \Trayref{D}
			\UCpaso Verifica que se cumpla con la la regla de negocio \BRref{RN32}.\Trayref{F}\Trayref{G}\Trayref{H}
			\UCpaso Preregistra el proyecto.\label{paso:CUC1.1PreRegistraProyecto}
			\UCpaso Muestra el mensaje (MSG-4) de operación exitosa.\ref{MSG4}

	\end{UCtrayectoria}

	\begin{UCtrayectoriaA}{A}{Usuario aún no registrado}{El usuario no ha iniciado sesión en el sistema.}
			\UCpaso [\UCactor] Ingresa los datos del Responsable del proyecto\ref{datosResponsable}.
			\UCpaso [\UCactor] Ingresa los datos del Proyecto.\Trayref{B}
			\UCpaso [\UCactor] Oprime el botón \IUbutton{Aceptar}.
			\UCpaso Verifica que se cumpla la regla de negocio \BRref{RN18}.\Trayref{C}
			\UCpaso Revisa que los datos correspondan con el Diccionario de Datos \Trayref{D}
			\UCpaso Verifica que el Periodo corresponda a la regla de negocio \BRref{RN32}.\Trayref{E}\Trayref{F}\Trayref{G}
			\UCpaso Solicita al Usuario llenar un formulario ``Captcha''.
			\UCpaso [\UCactor] Ingresa el ``Captcha''.
			\UCpaso Preregistra el proyecto.\label{paso:CUC1.1PreRegistraProyecto}
			\UCpaso Muestra el mensaje (MSG-4) de operación exitosa.\ref{MSG4}
	\end{UCtrayectoriaA}

	\begin{UCtrayectoriaA}{B}{Cancelar operación}{El usuario abandona el Caso de Uso.}
			\UCpaso[\UCactor] Decide ya no agregar un nuevo Proyecto.
			\UCpaso[\UCactor] Oprime el botón \IUbutton{Cancelar}.
			\UCpaso Continúa en el paso \ref{paso:CUC1IdentificaUsuario}.
	\end{UCtrayectoriaA}

	\begin{UCtrayectoriaA}{C}{Datos del proyecto incompletos}{Los datos proporcionados para el preregistro del proyecto están incompletos.}
			\UCpaso Muestra el mensaje (MSG-1) indicando los campos faltantes.\ref{MSG1}.
			\UCpaso Continúa en el paso \ref{paso:CUC1.1IngresaDatosProyecto}.
	\end{UCtrayectoriaA}

	\begin{UCtrayectoriaA}{D}{Datos del Proyecto Incorrectos}{Los datos del Proyecto no corresponden con lo especificado en el diccionario de datos.}
			\UCpaso Muestra el mensaje (MSG-2) indicando que los datos ingresados son incorrectos.\ref{MSG2}
			\UCpaso Continúa en el paso \ref{paso:CUC1.1IngresaDatosProyecto}.
	\end{UCtrayectoriaA}


	\begin{UCtrayectoriaA}{E}{Periodo de tipo ``Determinado``}{El Usuario eligió establecer el Periodo ``Por fechas'' especificando las fechas de inicio y término.}
			\UCpaso Calcula el número de días desde la fechas de inicio hasta la fecha de fin y asigna como ''Duración'' el valor obtenido.
			\UCpaso Continúa en el paso \ref{paso:CUC1.1PreRegistraProyecto}.
	\end{UCtrayectoriaA}

	\begin{UCtrayectoriaA}{F}{Periodo de tipo ``Relativo``}{El Usuario eligió establecer el Periodo ``Por duración`` especificando la Duración del proyecto.}
			\UCpaso Asigna como fechas de inicio y fin un valor nulo.
			\UCpaso Continúa en el paso \ref{paso:CUC1.1PreRegistraProyecto}.
	\end{UCtrayectoriaA}

	\begin{UCtrayectoriaA}{G}{Periodo de tipo ``Indeterminado``}{El usuario no indicó la duración del proyecto ni sus fechas de inicio y término.}
			\UCpaso Asigna como ''Duración'' y fechas de inicio-fin un valor de nulo.
			\UCpaso Continúa en el paso \ref{paso:CUC1.1PreRegistraProyecto}.
	\end{UCtrayectoriaA}

%-------------------------Registrar Proyecto

	\begin{UseCase}{CUC1.2}{Registrar Proyecto}{Registrar los Proyectos aprobados por el Administrador.}
		\UCitem{Versión}{1.0}
		\UCitem{Actor(es)}{Coordinador}
		\UCitem{Propósito}{Proporcionar los datos necesarios para el registro del Proyecto.}
		\UCitem{Resumen}{Se muestran los datos del Proyecto pre-registrado con la posibilidad de editar los datos del Proyecto.}
		\UCitem{Entradas}{Falta.}
		\UCitem{Salidas}{Mensaje de confirmación del registro de proyecto.}
		\UCitem{Precondiciones}{Se presentan las siguientes precondiciones
		  \begin{itemize}
		    \item Que existan Programas registrados.
		    \item Que existan Estructuras definidas para los Programas.
		    \item Que existan Ejes Temáticos registrados.
		    \item Que existan Temas Transversales registrados.
		    \item Que exista un Proyecto Pre-registrado.    
		  \end{itemize}}
		\UCitem{Postcondiciones}{El proyecto queda listo para definir metas, indicadores, acciones y participantes.}
		%\UCitem{Autor}{Omar Juárez Gambino}
		\UCitem{Referencias}{SIDAM-BESP-P1}
		\UCitem{Tipo}{Secundario. Viene del CU CUC1}
		\UCitem{Módulo}{Coordinación}
	\end{UseCase}
	
	\begin{UCtrayectoria}{Principal}
			\UCpaso[\UCactor] Oprime el botón \IUbutton{Registrar Proyecto} de la pantalla \IUref{IUGestProyectos}{Gestión de Proyectos}.
			\UCpaso Muestra la pantalla \IUref{IURegistrarProyecto}{Registrar Proyecto}.
			\UCpaso Obtiene los datos del Proyecto Pre-registrado.
			\UCpaso Muestra los datos del Proyecto Pre-registrado.
			\UCpaso Deshabilita los campos Siglas, Nombre y Datos del Responsable.
			\UCpaso Obtiene los Ejes Temáticos registrados.
			\UCpaso Muestra los Ejes Temáticos.
			\UCpaso Obtiene los Temas Transversales registrados.
			\UCpaso Muestra los Temas Transversales.
			\UCpaso Obtiene las estructuras definidas del Programa Sectorial y de los demás Programas.
			\UCpaso Muestra las estructuras obtenidas.
			\UCpaso [\UCactor] Ingresa los datos del Proyecto.\label{paso:CUC1.2IngresaDatosProyecto} \Trayref{A}
			\UCpaso [\UCactor] Activa la casilla de verificación para alinear el Proyecto con un elemento EP1N del Programa Sectorial. \Trayref{B}
			\UCpaso [\UCactor] Selecciona el elemento EP1N del Programa Sectorial.
			\UCpaso [\UCactor] Activa la casilla de verificación para alinear el Proyecto con un elemento EP1N de otro Programa. \Trayref{C} \label{paso:CUC1.2SeleccionaAlineacionPrograma}
			\UCpaso [\UCactor] Selecciona el elemento EP1N de otro Programa.
			\UCpaso [\UCactor] Oprime el botón \IUbutton{Aceptar}.\label{paso:CUC1.2OprimeAceptar}
			\UCpaso Revisa que los datos correspondan con la Definición de Datos. \Trayref{D}
			\UCpaso Verifica que se cumpla la regla de negocio \BRref{RN44}.\Trayref{E}
			\UCpaso Verifica que se cumpla la regla de negocio \BRref{RN43}.\Trayref{F}
			\UCpaso Verifica que se cumpla la regla de negocio \BRref{RN32}.\Trayref{G}
			\UCpaso Verifica que se cumpla la regla de negocio \BRref{RN33}. \Trayref{H}
			\UCpaso Verifica que se cumpla la regla de negocio \BRref{RN40}. \Trayref{I}
			\UCpaso Registra el Proyecto.
			\UCpaso Muestra el mensaje (MSG-4) de operación exitosa.\ref{MSG4}
	\end{UCtrayectoria}

	\begin{UCtrayectoriaA}{A}{Cancelar operación}{El usuario abandona el Caso de Uso.}
			\UCpaso[\UCactor] Decide ya no registrar un Proyecto.
			\UCpaso[\UCactor] Oprime el botón \IUbutton{Cancelar}.
			\UCpaso Continúa en el paso \ref{paso:CUC1MuestraDatosProyecto}.
	\end{UCtrayectoriaA}

	\begin{UCtrayectoriaA}{B}{Sin alineación a Programa Sectorial}{El usuario no alinea el Proyecto con el Programa Sectorial.}
			\UCpaso[\UCactor] No activa la casilla de verificación de alineación con el Programa Sectorial.
			\UCpaso El proyecto no se alinea con ningún elemento EP1N del Programa Sectorial.
			\UCpaso Continúa en el paso \ref{paso:CUC1.2SeleccionaAlineacionPrograma}.
	\end{UCtrayectoriaA}

	\begin{UCtrayectoriaA}{C}{Sin alineación a otros Programas}{El usuario no alinea el Proyecto con los Programas no sectorial.}
			\UCpaso[\UCactor] No activa la casilla de verificación de alineación con otros Programas.
			\UCpaso El proyecto no se alinea con ningún elemento EP1N del otros Programas.
			\UCpaso Continúa en el paso \ref{paso:CUC1.2OprimeAceptar}.
	\end{UCtrayectoriaA}

	\begin{UCtrayectoriaA}{D}{Datos del Proyecto Incorrectos}{Los datos del Proyecto no corresponden con lo especificado en el diccionario de datos.}
			\UCpaso Muestra el mensaje (MSG-1) indicando que los datos ingresados son incorrectos.\ref{MSG1}
			\UCpaso Continúa en el paso \ref{paso:CUC1.2IngresaDatosProyecto}.
	\end{UCtrayectoriaA}

	\begin{UCtrayectoriaA}{E}{Datos del proyecto incompletos}{Los datos proporcionados para el Registro del proyecto están incompletos.}
			\UCpaso Muestra el mensaje (MSG-2).\ref{MSG2}
			\UCpaso Continúa en el paso \ref{paso:CUC1.2IngresaDatosProyecto}.
	\end{UCtrayectoriaA}

	\begin{UCtrayectoriaA}{F}{Alineación de Proyecto}{Proyecto que se desea registrar no esta bien alineado.}
			\UCpaso Muestra el mensaje (MSG-RN-43).\ref{MSG_RN43}
			\UCpaso Continúa en el paso \ref{paso:CUC1.2IngresaDatosProyecto}.
	\end{UCtrayectoriaA}

	\begin{UCtrayectoriaA}{G}{Periodo mal definido}{El periodo del Proyecto que se desea registrar no está bien definido.}
			\UCpaso Muestra el mensaje (MSG-RN-32).\ref{MSG_RN32}
			\UCpaso Continúa en el paso \ref{paso:CUC1.1IngresaDatosProyecto}.
	\end{UCtrayectoriaA}

	\begin{UCtrayectoriaA}{H}{El nombre del Proyecto repetido}{El Nombre del Proyecto ya se encuentra registrado.}
			\UCpaso Muestra el mensaje (MSG-3) \ref{MSG3}
			\UCpaso Continúa en el paso \ref{paso:CUC1.2IngresaDatosProyecto}.
	\end{UCtrayectoriaA}

	\begin{UCtrayectoriaA}{I}{Las siglas del Proyecto repetidas}{Las siglas del Proyecto ya se encuentra registradas.}
			\UCpaso Muestra el mensaje (MSG-3) \ref{MSG3}
			\UCpaso Continúa en el paso \ref{paso:CUC1.2IngresaDatosProyecto}.
	\end{UCtrayectoriaA}
%-------------------------------------- TERMINA descripción del caso de uso.
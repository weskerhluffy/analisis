\subsection{Pantalla: Consultar bitacora gerente}

\subsubsection{Objetivo}
  Mostrar los mensajes acerca de restricciones del proyecto de o hacia el gerente y permitir atender restricciones o turnar atencion a restricciones.

\subsubsection{Diseño}
\IUfig[0.8]{CUG1/consultaBitacoraGerente.png}{IUConsultaBitacoraGerente}{Consulta bitacora, vista del gerente.}
\IUfig[0.8]{CUG1/consultaBitacoraGerente_atenderRestriccion.png}{IUConsultaBitacoraGerenteAtenderRestriccion}{Atiende restricción en la bitacora, vista del gerente.}
\IUfig[0.8]{CUG1/consultaBitacoraGerente_turnarRestriccion.png}{IUConsultaBitacoraGerenteTurnarRestriccion}{Turnar restricción en la bitacora, vista del gerente.}

\subsubsection{Salidas}
  En esta pantalla se muestran los mensajes entre el gerente con los demas usuarios agrupados por asunto y estan ordenados de forma descendente con respecto a la fecha de los asuntos.
  Para la vista se usara un acordeon para mostrar/ocultar los mensajes de cada asunto.

  Tipos de mensajes
  
  Describir orden: Primero las gestiones y luego los avisos. Las gestiones se ordenan por estatus, primero las turnadas, luego las que estan pendientes y por ultimas las que estan atendidas. Al interior cada una se ordenara por fecha de ultima atencion descendente.

  Los avisos se ordenan por fecha en orden descendente.

\subsubsection{Controles}
\begin{itemize}
 \item Expandir y contraer mensajes: Esta opción permite al coordinador ver a detalle los datos de la restricción u ocultarlos para poder ver otras restricciones.
\end{itemize}

\subsubsection{Comandos}
\begin{itemize}
 \item \includegraphics[width=.1\textwidth]{images/CUG1/Atender.png}: El sistema muestra actualizara la pantalla \IUref{IUConsultaBitacoraGerente}{Consulta bitacora, vista del gerente.} para poder agregar las instrucciones para la solución de la restricción.
 \item \includegraphics[width=.1\textwidth]{images/CUG1/Turnar.png}: El sistema muestra actualizara la pantalla \IUref{IUConsultaBitacoraGerente}{Consulta bitacora, vista del gerente.} para poder agregar la descripción de la restricción y poder turnarla.
 \item \includegraphics[width=.16\textwidth]{images/CUG1/AtenderRestriccion.png}: El sistema muestra actualizara la pantalla \IUref{IUConsultaBitacoraGerenteAtenderRestriccion}{Atiende restricción en la bitacora, vista del gerente.} para registrar las instrucciones.
 \item \includegraphics[width=.16\textwidth]{images/CUG1/HabilitarEdicion.png}: El sistema actualizara la pantalla \IUref{IUConsultaBitacoraGerenteAtenderRestriccion}{Atiende restricción en la bitacora, vista del gerente.} para poder habilitar la edición del proyecto.
 \item \includegraphics[width=.16\textwidth]{images/CUG1/TurnarRestriccion.png}: El sistema muestra actualizara la pantalla \IUref{IUConsultaBitacoraGerenteTurnarRestriccion}{Turnar restricción en la bitacora, vista del gerente.} para registrar la descripción de la restriccion y poner pendiente la restricción.
 \item \includegraphics[width=.1\textwidth]{images/CUG1/Cancelar.png}: El sistema muestra actualizara las pantallas \IUref{IUConsultaBitacoraGerenteAtenderRestriccion}{Atiende restricción en la bitacora, vista del gerente.} / \IUref{IUConsultaBitacoraGerenteTurnarRestriccion}{Turnar restricción en la bitacora, vista del gerente.} para eliminar los campos de atender/turnar y no registrar cualquiera de las dos acciones.
\end{itemize}

\subsubsection{Mensajes}
\begin{itemize}
  \item MSG1: Debe ingresar todos los datos.
  \item MSG4: La operación se relizo exitosamente..
\end{itemize}
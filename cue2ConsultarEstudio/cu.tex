% Descripción: Describe la funcionalidad ofrecida por el CU
% Propósito: Describe el objetivo o razón de ser del CU
% Resumen: Describe brevemente lo que hace el CU


\begin{UseCase}{CUE2}{Consultar Estudios}{Ofrece un mecanismo que muestra los diferentes estudios registrados y ofrece la funcionalidad de consulta de éstos, ya sea por tema transversal o por eje temático.}
		\UCitem{Versión}{2.0}
		\UCitem{Actor(es)}{Ciudadano}
		\UCitem{Propósito}{Contar con un mecanismo que permita consultar los estudios.}
		\UCitem{Resumen}{Mecanismo para consultar los diferentes {\bf Estudios} registrados para consultarlos por tematransversal o por eje temático.}
		\UCitem{Entradas}{Ninguna.}
		\UCitem{Salidas}{Lista con los estudios registrados.}
		\UCitem{Precondiciones}{Ninguna.}
		\UCitem{Postcondiciones}{Ninguna.}
		%\UCitem{Autor}{Vázquez Flores Jorge Aarón}
		\UCitem{Referencias}{Ninguna}
		\UCitem{Tipo}{Primario.}
		\UCitem{Módulo}{Consulta Ciudadana}
	\end{UseCase}
	
	% 1.- escriba solo una trayectoria principal
	% 2.- El actor es quien siempre inicia un CU
	% 3.- evite usar ``si ... entonces ...'' o ``mientras ...'' o ``para cada ...''
	% 4.- No olvide mencionar todas las verificaciones y cálculos realizados por el sistema
	% 5.- Evite mencionar palabras como: Tabla, BD, conexión, etc.
	
	
	
	
	\begin{UCtrayectoria}{Principal}
		\UCpaso[\UCactor] Selecciona la opción \IUbutton{Consulta Estudio} en el menú \IUref{IUMenuCiudadano}{Menú del Ciudadano}.
		\UCpaso Busca los Estudios registrados.\Trayref{A}\label{Paso:CUE2BuscaEstudios}
		\UCpaso Muestra en la pantalla \IUref{IUConsultarEstudios}{Consultar Estudios.} los estudios registrados con las opciones para Consultar Estudios por Tema Transversal o por Eje Temático\UCExtensionPoint{CUE2.1}{Consultar Estudio por Tema Transversal} \UCExtensionPoint{CUE2.2}{Consultar Estudio por Eje Temático}
	\end{UCtrayectoria}

	\begin{UCtrayectoriaA}{A}{No hay estudios registrados.}{No se encuentran estudios registrados en la base de datos.}
			\UCpaso Se muestra el mensaje (MSG5) notificando al usuario que no se encontraron registros.\ref{MSG5}
	\end{UCtrayectoriaA}


%---------------------------frece un mecanismo para consultar los estudios por tema transversal o por eje temático.------------------------------ Consultar Estudio por Tema Transversal
	
	\begin{UseCase}{CUE2.1}{Consultar Estudio por Tema Transversal}{El ciudadano selecciona un estudio por Tema Transversal, y se muestra la información registrada acerca de dicho estudio para su consulta.}
			\UCitem{Versión}{2.0}
			\UCitem{Actor(es)}{Ciudadano}
			\UCitem{Propósito}{Consultar el estudio seleccionado por Tema Transversal.}
			\UCitem{Resumen}{Se muestra la informacion del estudio seleccionado por Tema Transversal.}		\UCitem{Entradas}{Ninguna.}	
			\UCitem{Salidas}{La información del estudio seleccionado.}
			\UCitem{Precondiciones}{Debe existir almenos un estudio registrado.}
			\UCitem{Postcondiciones}{Ninguna.}
			%\UCitem{Autor}{Vázquez Flores Jorge Aarón}
			\UCitem{Referencias}{Ninguna}
			\UCifrece un mecanismo para consultar los estudios por tema transversal o por eje temático.tem{Tipo}{Secundario. Viene del \UCref{CUE2}}
			\UCitem{Módulo}{Consulta Ciudadana}
	\end{UseCase}

	
	\begin{UCtrayectoria}{Principal}
			\UCpaso[\UCactor] Oprime el botón \IUbutton{Consultar por Tema Transversal} de la pantalla \IUref{IUConsultarEstudios}{Consultar estudios.}
			\UCpaso Muestra la pantalla \IUref{IUConsultarPorTema}{Consulta por Tema Transversal}.
			\UCpaso [\UCactor] Oprime el botón \IUbutton{Aceptar}.			
			\UCpaso Continua en el paso 2 de \UCref{CUE2}.
	\end{UCtrayectoria}
	
%----------------------------------------------------- Consultar Estudio por Eje Tematico

	\begin{UseCase}{CUE2.2}{Consultar Estudio por Eje Temático}{El ciudadano selecciona un estudio por Eje Temático, y se muestra la información registrada acerca de dicho estudio para su consulta.}
			\UCitem{Versión}{2.0}
			\UCitem{Actor(es)}{Ciudadano}
			\UCitem{Propósito}{Consultar el estudio seleccionado por Eje Temático.}
			\UCitem{Resumen}{Se muestra la información del estudio seleccionado por Tema Transversal.}		\UCitem{Entradas}{Ninguna.}	
			\UCitem{Salidas}{La información del estudio seleccionado.}
			\UCitem{Precondiciones}{Debe existir almenos un estudio registrado.}
			\UCitem{Postcondiciones}{Ninguna.}
			%\UCitem{Autor}{Vázquez Flores Jorge Aarón}
			\UCitem{Referencias}{Ninguna}
			\UCitem{Tipo}{Secundario. Viene del \UCref{CUE2}}
			\UCitem{Módulo}{Consulta Ciudadana}\UCitem{Entradas}{Ninguna.}
	\end{UseCase}

	\begin{UCtrayectoria}{Principal}
			\UCpaso[\UCactor] Oprime el botón \IUbutton{Consultar por Eje Temático} de la pantalla \IUref{IUConsultarEstudios}{Consultar estudios.}
			\UCpaso Muestra la pantalla \IUref{IUConsultarPorEje}{Consulta por Eje Temático}.
			\UCpaso [\UCactor] Oprime el botón \IUbutton{Aceptar}.			
			\UCpaso Continua en el paso 2 de \UCref{CUE2}.
	\end{UCtrayectoria}

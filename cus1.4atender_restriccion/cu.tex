% Descripción: Describe la funcionalidad ofrecida por el CU
% Propósito: Describe el objetivo o razón de ser del CU
% Resumen: Describe brevemente lo que hace el CU
% 
%========Aqui se describen los casos de uso que se derivan de el caso de uso CU2.1.3 Revisar seguimiento de proyecto para el modulo de secretaría

%--------- Atender restricción en el proyecto
	\begin{UseCase}{CUS1.4}{Atender restricción.}{Permite enviar instrucciones para la solución de una restricción del proyecto al gerente, además, permite al secretario conocer las restricciones del proyecto para evaluarlas.}
		\UCitem{Versión}{2.0}
		\UCitem{Actor(es)}{Secretario.}
		\UCitem{Propósito}{Permitir al secretario conocer las restricciones del proyecto, y enviarle instrucciones al gerente.}
		\UCitem{Resumen}{El secretario evaluara y dará las restricciones del proyecto.}
		\UCitem{Entradas}{Instrucciones para la solución de la restricción del proyecto.\ref{dd:Atencion}}
		\UCitem{Salidas}{Mensaje de confirmación.\ref{MSG5}}
		\UCitem{Precondiciones}{El proyecto debe estar en ejecución.}
		\UCitem{Postcondiciones}{Se registran las instrucciones para la solución a la restricción y el estado de edición del proyecto.}
		%\UCitem{Autor}{Torres Govea Miguel Angel}
		\UCitem{Referencias}{SIDAM-BESP-P1-Especificación de Catálogos}
		\UCitem{Tipo}{Secundario. Viene del \UCref{CU2.1.3}.}
		\UCitem{Módulo}{Secretaría.}
	\end{UseCase}
		
	\begin{UCtrayectoria}{Principal}
		\UCpaso[\UCactor] Da clic en el botón \IUbutton{Atender.} en la restricción a atender \IUref{IUConsultaBitacoraSecretario}{Bitacora del seguimiento del proyecto, vista
del secretario.}.
		\UCpaso Actualiza la pantalla para la atención de la restricción.\IUref{IUConsultaBitacoraSecretarioAtenderRestriccion}{Bitacora del seguimiento del proyecto, vista
del secretario, atender restricción.}
		\UCpaso [\UCactor] Agrega las instrucciones para la solución de la restricción.\label{paso:CUS1.4ingresaDatosAtencion}
		\UCpaso [\UCactor] Da clic en el botón \IUbutton{Atender Restricción}. \Trayref{A}
		\UCpaso Verifica que se cumpla la regla de negocio \BRref{RN2}. \Trayref{B}
		\UCpaso Verifica que el registro coincida con la definición en el diccionario de datos \ref{dd:Atencion}. \Trayref{C}
		\UCpaso Registra la atención.
		\UCpaso Muestra el mensaje (MSG-4) de operación exitosa.\ref{MSG4}
	\end{UCtrayectoria}
		
	\begin{UCtrayectoriaA}{A}{Cancelar operación}{El usuario abandona el Caso de Uso.}
			\UCpaso[\UCactor] Decide ya no registrar la atención.
			\UCpaso[\UCactor] Oprime el botón \IUbutton{Cancelar}.
			\UCpaso Actualiza la pantalla removiendo el campo de atención y no la registra. \IUref{IUConsultaBitacoraSecretario}{Bitacora del seguimiento del proyecto, vista
del secretario.}
	\end{UCtrayectoriaA}

	\begin{UCtrayectoriaA}{B}{Datos de la Atención Incompletos.}{Algunos campos del formulario no han sido capturados.}
			\UCpaso Muestra el mensaje (MSG-1) indicando al usuario verifique que todos los campos hayan sido capturados.\ref{MSG1}
			\UCpaso Continúa en el paso \ref{paso:CUS1.4ingresaDatosAtencion} del \UCref{CUS1.4}.
	\end{UCtrayectoriaA}

	\begin{UCtrayectoriaA}{C}{Datos de la Atención Inconsistentes.}{Algunos datos no son validos.}
			\UCpaso Muestra el mensaje (MSG-2) indicando al usuario verifique que todos los campos son validos.\ref{MSG2}
			\UCpaso Continúa en el paso \ref{paso:CUS1.4ingresaDatosAtencion} del \UCref{CUS1.4}.
	\end{UCtrayectoriaA}
\subsection{Pantalla: Consultar bitacora secretario}

\subsubsection{Objetivo}
  Mostrar los mensajes acerca de restricciones del proyecto de o hacia el secretario, permitir atender restricciones y/o habilitar la edición del proyecto.

\subsubsection{Diseño}
\IUfig[0.8]{CUS1/consultaBitacoraSecretario.png}{IUConsultaBitacoraSecretario}{Consulta bitacora, vista del secretario.}
\IUfig[0.8]{CUS1/consultaBitacoraSecretario_atenderRestriccion.png}{IUConsultaBitacoraSecretarioAtenderRestriccion}{Atender restricción en bitacora, vista del secretario.}	

\subsubsection{Salidas}
  En esta pantalla se muestran los mensajes entre el secretario con los demas usuarios agrupados por asunto y estan ordenados de forma descendente con respecto a la fecha de los asuntos.
  Para la vista se usara un acordeon para mostrar/ocultar los mensajes de cada asunto.

  Tipos de mensajes
  
  Describir orden: Primero las gestiones y luego los avisos. Las gestiones se ordenan por estatus, primero las turnadas, luego las que estan pendientes y por ultimas las que estan atendidas. Al interior cada una se ordenara por fecha de ultima atencion descendente.

  Los avisos se ordenan por fecha en orden descendente.

\subsubsection{Controles}
\begin{itemize}
 \item Expandir y contraer mensajes: Esta opción permite al coordinador ver a detalle los datos de la restricción u ocultarlos para poder ver otras restricciones.
\end{itemize}

\subsubsection{Comandos}
\begin{itemize}
 \item \IUbutton{Atender}: El sistema actualizara la pantalla \IUref{IUConsultaBitacoraSecretario}{Consulta bitacora, vista del secretario.} para poder atender la restricción.
 \item \IUbutton{Atender Restricción}: El sistema actualizara la pantalla \IUref{IUConsultaBitacoraSecretarioAtenderRestriccion}{Atender restriccion en bitacora, vista del secretario.} para poder registrar la atención siempre que los campos esten completos y mostrará el MSG4. En otro caso el sistema mostrará el MSG2 (por la regla \BRref{RN1})). 
 \item \IUbutton{Cancelar}: El sistema actualizara la pantalla \IUref{IUConsultaBitacoraSecretarioAtenderRestriccion}{Atender restriccion en bitacora, vista del secretario.} para eliminar el campos de atender y no registrar la atención.
\end{itemize}

\subsubsection{Mensajes}
\begin{itemize}
  \item MSG2: Debe ingresar todos los datos.
  \item MSG4: La operación se ha realizado exitosamente.
\end{itemize}
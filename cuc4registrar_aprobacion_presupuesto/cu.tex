	\begin{UseCase}{CUC4}{Registrar aprobación de Presupuesto}{El usuario determina el monto que se aprueba de un presupuesto registrado.}
			\UCitem{Versión}{3.5}
			\UCitem{Estado}{Finalizado}
			\UCitem{Actor(es)}{Coordinador.}
			\UCitem{Propósito}{Saber el monto aprobado para el proyecto.}
			\UCitem{Resúmen}{Se ingresa el monto y la fecha de aprobación del presupuesto para un proyecto.}
			\UCitem{Entradas}{Datos de la aprobación presupuesto \ref{dd:DEPresupuestos}.}
			\UCitem{Salidas}{Se registra el monto aprobado.}
			\UCitem{Precondiciones}{Que exista al menos un presupuesto  en estado de registrado para el proyecto seleccionado.}
			\UCitem{Postcondiciones}{Presupuesto aprobado listo para ser ejercido.}
			%\UCitem{Autor}{Hermosillo García Karen Adriana }
			\UCitem{Referencias}{CU-P2}
			\UCitem{Tipo}{Secundario. Viene del \UCref{CU2.1.2}.}
			\UCitem{Módulo}{Coordinación.}
	\end{UseCase}

	\begin{UCtrayectoria}{Principal}
			\UCpaso[\UCactor] Oprime el botón \IUbutton{Dictaminar presupuesto}  de la pantalla \IUref{IURevisarProyectos}{Revisar Proyecto}.
			\UCpaso Muestra la pantalla \IUref{IUAprobarPresupuesto}{Aprobar Presupuesto} 
			\UCpaso[\UCactor] Selecciona el presupuesto.\label{paso:CU2Apresupuesto}
      \UCpaso[\UCactor] Oprime el botón \IUbutton{Aprobar} del presupuesto en cuestión.
			\UCpaso[\UCactor] Ingresa los datos de monto aprobado y fecha de aprobación. \Trayref{A} \label{paso:CUC4ingresarDatos}.

			\UCpaso [\UCactor] Oprime el botón  \IUbutton{Aceptar}.
			\UCpaso Revisa que los datos cumplan la regla de negocios \BRref{RN2}. \Trayref{B}
			\UCpaso Revisa los datos de acuerdo al diccionario \ref{dd:DEPresupuestos} \Trayref{C}
			\UCpaso Verifica que se cumpla la regla de negocios \BRref{RN57}. \Trayref{D} 
      \UCpaso Verifica que se cumpla la regla de negocios \BRref{RN72}. \Trayref{D}    
			\UCpaso El sistema registra la aprobación del presupuesto.
			\UCpaso Muestra el mensaje \ref{MSG4}.
			\UCpaso Regresa a la pantalla anterior.
	\end{UCtrayectoria}
	\newpage
	\begin{UCtrayectoriaA}{A}{Cancelar operación}{El usuario abandona el Caso de Uso.}
			\UCpaso[\UCactor] Decide no aprobar el presupuesto.
			\UCpaso[\UCactor] Oprime el botón \IUbutton{Cancelar}.
			\UCpaso Continúa en el paso \ref{paso:CU2Apresupuesto} del \UCref{CU2.1.2}.
	\end{UCtrayectoriaA}
		
	\begin{UCtrayectoriaA}{B}{Datos nulos}{Los datos ingresados por el usuario  no cumplen con la regla de negocios \BRref{RN2}.}
			\UCpaso Muestra el mensaje de datos nulos \ref{MSG1}.
			\UCpaso Continúa en el paso \ref{paso:CUC4ingresarDatos} del \UCref{CUC4}.
	\end{UCtrayectoriaA}
	\begin{UCtrayectoriaA}{C}{Datos incorrectos}{Los datos ingresados por el usuario  no cumplen con los datos del presupuesto \ref{dd:PresupuestoEjercido}.}
			\UCpaso Muestra el mensaje de datos incorrectos. \ref{MSG2}
			\UCpaso Continúa en el paso \ref{paso:CUC4ingresarDatos} del \UCref{CUC4}.
	\end{UCtrayectoriaA}

	\begin{UCtrayectoriaA}{D}{Monto menor a 0}{El monto ingresado no cumple con la regla de negocios \BRref{RN57}.}
		\UCpaso Muestra el mensaje. \ref{MSG_RN57}
		\UCpaso Continúa en el paso \ref{paso:CUC4ingresarDatos} del \UCref{CUC4}.
	\end{UCtrayectoriaA}


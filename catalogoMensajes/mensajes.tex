%---------------Validación de RN2
\section{Mensaje de validación para la RN2: Datos nulos} \label{MSG1}

\subsubsection{Objetivo}
Indicar al usuario que uno de los campos que ha llenado en el formulario esta vacío, provocando un conflicto con la \BRref{RN2}.

\subsubsection{Descripción del mensaje}
\begin{itemize}
\item Nombre: MSG-1
\item Cuerpo: Favor de introducir el/la CAMPO.

\end{itemize}
Donde el CAMPO es el elemento del formulario que se encuntra vacío.

%---------------Validación Diccionario de Datos

\section{Mensaje de validación de Diccionario de Datos: Datos Incorrectos}\label{MSG2}

\subsubsection{Objetivo}
Indicar al usuario que uno de los campos que ha llenado en el formulario no cumple con lo especificado en el Diccionario de Datos.

\subsubsection{Descripción del mensaje}
\begin{itemize}
\item Nombre: MSG-2
\item Cuerpo: El  valor del CAMPO es incorrecto, favor de introducir un dato válido, solo se admiten VALORES y el tamaño debe ser menor a TAMAÑO.

\end{itemize}
Donde el CAMPO es el elemento del formulario que contiene el error, VALORES es el valor asignado en el diccionario de datos y el TAMAÑO es el la longitud especificada en el diccionario de datos

%--------------------------Mensaje para inicio de sesión incorrecto
\section{Mensaje de validación de Usuario: Datos Incorrectos}\label{MSG2-2}

\subsubsection{Objetivo}
Indicar al usuario que uno de los campos que ha llenado para iniciar sesión es incorrecto.

\subsubsection{Descripción del mensaje}
\begin{itemize}
\item Nombre: MSG2-2
\item Cuerpo: Login y/o contraseña incorrectos.

\end{itemize}

\section{Mensaje de validación Nombre de Usuario Incorrecto}\label{MSG2-3}

\subsubsection{Objetivo}
Indicar al Usuario que el Nombre de Usuario que ingresó es incorrecto.

\subsubsection{Descripción del mensaje}
\begin{itemize}
\item Nombre: MSG2-3
\item Cuerpo: Nombre de Usuario incorrecto.

\end{itemize}
%---------------Validación de RN1

\section{Mensaje de validación para la RN1, RN33, RN40: Nombre/Siglas con un valor repetido}\label{MSG3}

\subsubsection{Objetivo}
Indicar al usuario que el campo llamado Nombre/Siglas en el formulario ya ha sido registrado, provocando un conflicto con la \BRref{RN1}, la \BRref{RN33} o la \BRref{RN40}.

\subsubsection{Descripción del mensaje}
\begin{itemize}
\item Nombre: MSG-3
\item Cuerpo: Un/a ENTIDAD con el mismo nombre/siglas ya ha sido registrado, favor de elegir otro/otras nombre/siglas o verificar que no se este intentando registrar el mismo.

\end{itemize}
Donde ENTIDAD es el elemento que se esta intentando registrar. Se debe mencionar que este mensaje de validación se debe usar la palabra \textit{sigla} para la validación de la regla de negocio \BRref{RN40} 

%---------------Operación exitosa

\section{Mensaje de confirmación para una operación exitosa.}\label{MSG4}

\subsubsection{Objetivo}
Indicar al usuario que la operación que acaba de realizar se ha llevado a cabo con exito.

\subsubsection{Descripción del mensaje}
\begin{itemize}
\item Nombre: MSG-4
\item Cuerpo: El/La NOMBRE DE ENTIDAD se OPERACIÓN exitosamente. 
\end{itemize}
Donde NOMBRE DE ENTIDAD es el nombre de la entidad sobre la cual se realizó la operación y la OPERACIÓN es: Agregó, Modificó o Eliminó.

%---------------No existen registros

\section{Mensaje de notificación para el resultado de una busqueda vacía.}\label{MSG5}

\subsubsection{Objetivo}
Indicar al usuario de los casos de uso `Gestión de catalogos` que la busqueda de registros no encontró resultados.

\subsubsection{Descripción del mensaje}
\begin{itemize}
\item Nombre: MSG-5
\item Cuerpo: No se encontrarón registros.

\end{itemize}
Donde los resultados son los registros del catalogo a buscar.

%---------------Seleccionar elemento de una lista

\section{Mensaje de notificación para seleccionar un elemento de una lista.}\label{MSG6}

\subsubsection{Objetivo}
Indicar al usuario que debe seleccionar un elemento de la lista.

\subsubsection{Descripción del mensaje}
\begin{itemize}
\item Nombre: MSG-6
\item Cuerpo: Debe seleccionar un/a CAMPO.

\end{itemize}
Donde CAMPO es el elemento del formulario de tipo lista.


%===============================================================
\section{Mensajes de eliminación por cada caso de uso.}

%------------------------  Eliminar usuario
\subsubsection{Mensaje para eliminar usuario con proyectos asociados.}\label{MSG-6A1}

\subsubsection{Objetivo}
Indicar al usuario del caso de uso `Gestión de Usuarios` que no puede eliminar el usuario deseado ya que este tiene datos asociados.

\subsubsection{Descripción del mensaje}
\begin{itemize}
\item Nombre: MSG-6A1
\item Cuerpo: No se puede eliminar el usuario ya que tiene proyectos asociados.
\end{itemize}

%------------------------  Eliminar área

\subsubsection{Mensaje de validación para la eliminación del Área RN50: Datos asociados al área} \label{MSG-6A7}
 
 \subsubsection{Objetivo}
 Indicar al usuario que el área que desea eliminar se encuentra asociada a un Usuario y entra en conflicto con la regla de negocio \BRref{RN50}.

 \subsubsection{Descripción del mensaje}
 \begin{itemize}
 \item Nombre: MSG-6A7
 \item Cuerpo: El área que desea eliminar esta asociada con uno o varios usuarios.
 \end{itemize}

%------------------------  Eliminar programa 1N

\subsubsection{Mensaje de validación para la eliminación de un Programa 1N: Proyectos en curso} \label{MSG-6G1}
 
 \subsubsection{Objetivo}
 Indicar al usuario que el Programa1N que esta intentando eliminar tiene asignados proyectos que estan en curso.

 \subsubsection{Descripción del mensaje}
 \begin{itemize}
 \item Nombre: MSG-6G1
 \item Cuerpo: El Programa que desea eliminar esta asociado a Proyectos que aún estan en curso.
 \end{itemize}

\subsubsection{Mensaje de validación para la eliminación de un Programa 1N: Proyectos en curso} \label{MSG-6G1.1}
 
 \subsubsection{Objetivo}
 Indicar al usuario que el Programa1N que esta intentando eliminar es un Programa sectorial.

 \subsubsection{Descripción del mensaje}
 \begin{itemize}
 \item Nombre: MSG-6G1.1
 \item Cuerpo: No se puede eliminar el Programa ya que es un Programa Sectorial.
 \end{itemize}

%------------------------  Eliminar Tema Transversal 1N

\subsubsection{Mensaje de validación para la eliminación del Tema Transversal: Proyectos asociados} \label{MSG-6G2}
 
 \subsubsection{Objetivo}
 Indicar al usuario que el Tema Transversal que esta intentando eliminar tiene asignados proyectos.

 \subsubsection{Descripción del mensaje}
 \begin{itemize}
 \item Nombre: MSG-6G2
 \item Cuerpo: El Tema Transversal que desea eliminar esta asociado con Proyectos.
 \end{itemize}

%------------------------  Eliminar Eje Temático

\subsubsection{Mensaje de validación para la eliminación del Eje temático: Proyectos asociados} \label{MSG-6G3}
 
 \subsubsection{Objetivo}
 Indicar al usuario que el Eje temático que esta intentando eliminar tiene asignados proyectos.

 \subsubsection{Descripción del mensaje}
 \begin{itemize}
 \item Nombre: MSG-6G3
 \item Cuerpo: El Eje temático que desea eliminar esta asociado con Proyectos.
 \end{itemize}

%---------------------eliminación de un nivel intermedio
\subsubsection{Mensaje de validación para la eliminación de un nivel intermedio.}\label{MSG-6D1}

\subsubsection{Objetivo}
Indicar al usuario que esta intentando eliminar un nivel intermedio perteneciente a un Programa 1N.

\subsubsection{Descripción del mensaje}
\begin{itemize}
\item Nombre: MSG-6D1
\item Cuerpo: El nivel NOMBRE que desea eliminar, no se puede eliminar ya que es un nivel intermedio. Antes debe eliminar los niveles con posiciones superiores.

\end{itemize}
Donde NOMBRE es el nombre del nivel que se desea eliminar.

\subsubsection{Mensaje de validación para la eliminación de un nivel con datos asociados.}\label{MSG-6D2}

\subsubsection{Objetivo}
Indicar al usuario que esta intentando eliminar un nivel que esta asociado con una Estructura de Programa de 1er Nivel.

\subsubsection{Descripción del mensaje}
\begin{itemize}
\item Nombre: MSG-6D2
\item Cuerpo: El nivel NOMBRE que desea eliminar, no se puede eliminar ya que esta asociado a una Estructura de Programa de 1er Nivel.
\end{itemize}
Donde NOMBRE es el nombre del nivel que se desea eliminar.

%-------------------eliminación de una estructura
\section{Mensaje de validación para la RN36} \label{MSG_RN36}

\subsubsection{Objetivo}
Indica al usuario que no es posible eliminar la estructura debido a que ya tiene proyectos alineados a ella, lo que contradice la \BRref{RN36}.

\subsubsection{Descripción del mensaje}
\begin{itemize}
\item Nombre: MSG-RN-36
\item Cuerpo: No se puede eliminar este ELEMENTO debido a que ya tiene proyectos asociados.
\end{itemize}
Donde ELEMENTO es el nombre del nivel de la estructura según el programa al que pertenece.

%------------------------Mensaje de contacto
\section{Mensaje de notificación ingresar contacto principal de un Tipo de contacto con un contacto principal ya existente.}\label{MSG6A}

\subsubsection{Objetivo}
Indicar al usuario que el contacto que quiere ingresar se marca como principal pero no puede hacerlo porque ya tiene un contacto principal .

\subsubsection{Descripción del mensaje}
\begin{itemize}
\item Nombre: MSG-6A
\item Cuerpo: no puedes Agregar este contacto como principal porque ya tienes un Contacto principal de este tipo de Contacto.

\end{itemize}

\section{Mensaje de notificación cambio del contacto principal .}\label{MSG7A}

\subsubsection{Objetivo}
Indicar al usuario que el contacto que quiere modificar se marcará como principal y que el princpila existente se marcara como normal..

\subsubsection{Descripción del mensaje}
\begin{itemize}
\item Nombre: MSG-7A
\item Cuerpo: cuidado: si marcas este contacto como principal el principla existente de este tipo de contacto pasara a ser normal.

\end{itemize}

\section{Mensaje de notificación: Datos Obligatorios.}\label{MSG8A}

\subsubsection{Objetivo}
Indicar al usuario que el contacto que quiere eliminar es un dato obligatorio y por tanto no puede ser eliminado.

\subsubsection{Descripción del mensaje}
\begin{itemize}
\item Nombre: MSG-8A
\item Cuerpo: cuidado: El contacto que desea eliminar s obligatorio por tanto no puede ser elimimnado.

\end{itemize}
%------------------------Mensaje tipo de contacto
\section{Mensaje de notificación para eliminar datos restringidos.}\label{MSG6B}

\subsubsection{Objetivo}
Indicar al usuario que esos datos no pueden ser eliminados.

\subsubsection{El dato que desea eliminar no puede ser eliminado por reglas d negocio 29}
\begin{itemize}
\item Nombre: MSG-6B
\item Cuerpo: El Tipo de contacto no se puede eliminar porque el usuario debe tener como minimo un correo y un telefono principal .
\end{itemize}

%===========================Mensajes especiales para Estructura====================================

\section{Mensaje de validación para la RN32} \label{MSG_RN32}

\subsubsection{Objetivo}
Indica al usuario que el periodo que se pretende dar a la Estructura/Proyecto no esta bien definido, lo que entra en conflicto con la \BRref{RN32}.

\subsubsection{Descripción del mensaje}
\begin{itemize}
\item Nombre: MSG-RN-32
\item Cuerpo: El periodo no esta bien definido. Lea el manual para saber como se define un periodo.
\end{itemize}

\section{Mensaje de validación para la RN7} \label{MSG_RN7}

\subsubsection{Objetivo}
Indica al usuario que el periodo que se pretende dar a la estructura tiene conflicto con el periodo de al menos uno de los ancestros o hijos, lo que contradice la \BRref{RN7}.

\subsubsection{Descripción del mensaje}
\begin{itemize}
\item Nombre: MSG-RN-7
\item Cuerpo: El periodo no es valido, por favor revise los ancestros e hijos de esta ELEMENTO.
\end{itemize}
Donde ELEMENTO es un elemento según se define en la \BRref{RN37}

\section{Mensaje de validación para la RN48} \label{MSG_RN48}

\subsubsection{Objetivo}
Indica al usuario que no es posible agregar la Estructura ya que supera el maximo nivel definido para el programa según la regla \BRref{RN48}.

\subsubsection{Descripción del mensaje}
\begin{itemize}
\item Nombre: MSG-RN-48
\item Cuerpo: No se puede agregar la estructura, el nivel maximo del programa ha sido superado.
\end{itemize}

\section{Mensaje de validación para la RN43} \label{MSG_RN43}

\subsubsection{Objetivo}
Indica al usuario que es necesario alinear un proyecto con algun programa 1n según la \BRref{RN43}.

\subsubsection{Descripción del mensaje}
\begin{itemize}
\item Nombre: MSG-RN-43
\item Cuerpo: Es necesario alinear el proyecto con algún elemento del programa sectorial o de otros programas.
\end{itemize}

\section{Mensaje de validación para la RN57} \label{MSG_RN57}
=======
%===========================Mensajes de confirmación de envío de proyecto====================================
\subsubsection{Objetivo}
Indica al usuario que es necesario que el monto ingresado sea mayor a 0 según la \BRref{RN57}.

\subsubsection{Descripción del mensaje}
\begin{itemize}
\item Nombre: MSG-RN-57
\item Cuerpo: El monto debe tener un valor mayor a 0.
\end{itemize}

\section{Mensaje de validación para la RN58} \label{MSG_RN58}

\subsubsection{Objetivo}
Indica al usuario que no se puede eliminar el presupuesto porque ya ha sido aprobado,  según la \BRref{RN58}.

\subsubsection{Descripción del mensaje}
\begin{itemize}
\item Nombre: MSG-RN-58
\item Cuerpo: El presupuesto no se puede eliminar porque ya ha sido aprobado
\end{itemize}

\section{Mensaje de confirmación para el envío del proyecto} \label{MSGE1}

=======
\subsubsection{Objetivo}
Indica al usuario las restricciones que tendrá su proyecto una vez que sea enviado para revisión.

\subsubsection{Descripción del mensaje}
\begin{itemize}
\item Nombre: MSG-E1
\item Cuerpo: ¿Esta seguro de enviar su proyecto para revisión? Si envía su proyecto ya no podrá: editar los datos del proyecto, gestionar las acciones, gestionar los indicadores físicos y gestionar el presupuesto solicitado. 
\end{itemize}


%===========================Mensajes de confirmación de envío de proyecto====================================
\section{Mensaje de validación para la RN59} \label{MSG_RN59}

\subsubsection{Objetivo}
Indica al usuario que el proyecto no cumple con las restricciones establecidas por las regla de negocio 59.

\subsubsection{Descripción del mensaje}
\begin{itemize}
\item Nombre: MSG-RN-59
\item Cuerpo: Para el envió del proyecto se deben cumplir las siguientes restricciones: alinear el proyecto con al menos un programa, definir al menos una acción, definir al menos un indicador por cada acción, la suma de los pesos de los indicadores de cada acción debe ser igual a 100. 
\end{itemize}

\section{Mensaje de validación para la RN61} \label{MSG_RN59}

\subsubsection{Objetivo}
Indica al usuario que el peso debe tener un valor superior a 0

\subsubsection{Descripción del mensaje}
\begin{itemize}
\item Nombre: MSG-RN-61
\item Cuerpo: El valor del peso debe ser mayor a 0
\end{itemize}

\section{Mensaje de validación para la RN71} \label{MSG_RN71}

\subsubsection{Objetivo}
Indica al usuario que no ha registrado avances.

\subsubsection{Descripción del mensaje}
\begin{itemize}
\item Nombre: MSG-RN-71
\item Cuerpo: Debe registrar avance en al menos un indicador.
\end{itemize}


%===========================Mensajes de revision de proyectos====================================
\section{Mensaje de revision de proyectos} \label{MSGg11}

\subsubsection{Objetivo}
Indica al usuario que el proyecto ha cambiado al estado ``Ejecucion''.

\subsubsection{Descripción del mensaje}
\begin{itemize}
\item Nombre: MSG-G1-1
\item Cuerpo: El proyecto ha cambiado de estado de ``Revision'' a ``Ejecución''.
\end{itemize}

\section{Mensaje de revision de proyectos} \label{MSGg12}

\subsubsection{Objetivo}
Indica al usuario que el proyecto ha cambiado al estado ``Edición''.

\subsubsection{Descripción del mensaje}
\begin{itemize}
\item Nombre: MSG-G1-2
\item Cuerpo: El proyecto ha cambiado de estado de ``Revisión'' a ``Edición''.
\end{itemize}





\section{Mensaje de reporte de avance de proyecto exitoso} \label{MSG_CUC6}

\subsubsection{Objetivo}
Indica al usuario que el avance de uno o mas indicadores ha sido registrado exitosamente.

\subsubsection{Descripción del mensaje}
\begin{itemize}
\item Nombre: MSG-CUC6
\item Cuerpo: El avance del proyecto PROYECTO fue modificado. 
\end{itemize}

Donde PROYECTO es el nombre del proyecto.

\section{Mensaje de confirmación de evidencia} \label{MSGc7}
\subsubsection{Objetivo}
Informar que la evidencia se agregó exitosamente.
\subsubsection{Descripción del mensaje}
\begin{itemize}
\item Nombre: MSG-C7
\item Cuerpo: La evidencia se agregó exitosamente.
\end{itemize}


\section{Mensaje de confirmación de datos de proyecto editados} \label{MSGCUC1}
\subsubsection{Objetivo}
Informar que los datos del proyecto y/o de su alineacion se realizaron exitosamente.
\subsubsection{Descripción del mensaje}
\begin{itemize}
\item Nombre: MSGCUC1
\item Cuerpo: La definicion del proyecto PROYECTO se ha modificado exitosamente.
\end{itemize}

Donde PROYECTO es el nombre del proyecto.

	\begin{UseCase}{CUA4}{Gestionar Unidades}{Administra un catálogo en el que se puede realizar Altas, Bajas y Cambios de las Unidades.}
		\UCitem{Versión}{2.0}
		\UCitem{Actor(es)}{Administrador.}
		\UCitem{Propósito}{Mantener organizada la información con respecto a las Unidades así como visualizar las Unidades registradas.}
		\UCitem{Resumen}{Se muestran las Unidades registradas con la posibilidad de agregar, modificar y eliminar una Unidad.}
		\UCitem{Entradas}{Ninguna.}
		\UCitem{Salidas}{Despliegue de las Unidades registradas.}
		\UCitem{Precondiciones}{Ninguna.}
		\UCitem{Postcondiciones}{Ninguna.}
		\UCitem{Autor}{Aquino Carrillo Héctor Hyacinth.}
		\UCitem{Referencias}{SIDAM-BESP-P2-Especificación de Catálogos}
		\UCitem{Tipo}{Primario.}
		\UCitem{Módulo}{Administración}
	\end{UseCase}
	
	% 1.- escriba solo una trayectoria principal
	% 2.- El actor es quien siempre inicia un CU
	% 3.- evite usar ``si ... entonces ...'' o ``mientras ...'' o ``para cada ...''
	% 4.- No olvide mencionar todas las verificaciones y cálculos realizados por el sistema
	% 5.- Evite mencionar palabras como: Tabla, BD, conexión, etc.
	
	
	
	
	\begin{UCtrayectoria}{Principal}
		\UCpaso[\UCactor] Oprime el botón \IUbutton{Gestionar Unidades} en el menú \IUref{IUMenuAdministrador}{Menú de Administrador}.
		\UCpaso Busca las Unidades registradas.\Trayref{A} \label{paso:CUA4buscarUnidades}
		\UCpaso Muestra la pantalla \IUref{IUGestUnidades}{Gestionar Unidades} donde se muestran los datos de las Unidades registradas y las opciones para Agregar, Modificar y Eliminar Unidades.\UCExtensionPoint{CUA4.1}{Agregar unidad} \UCExtensionPoint{CUA4.2}{Modificar unidad} \UCExtensionPoint{CUA4.3}{Eliminar unidad}
	\end{UCtrayectoria}
	  
	\begin{UCtrayectoriaA}{A}{No existen datos de la unidad}{No se encontraron unidades registradas}
			\UCpaso[\UCactor] Muestra el mensaje (MSG-5) indicando que no se encontrarón unidades registradas.\ref{MSG5}
			\UCpaso Continúa en el paso \ref{paso:CUA4buscarUnidades} del \UCref{CUA4}.
	\end{UCtrayectoriaA}
%--------- Agregar Unidad
	\begin{UseCase}{CUA4.1}{Agregar Unidad}{El usuario registra una nueva Unidad con sus respectivos datos.}
		\UCitem{Versión}{2.0}
		\UCitem{Actor(es)}{Administrador.}
		\UCitem{Propósito}{Agregar una nueva Unidad.}
		\UCitem{Resumen}{Se agrega una Unidad registrando sus datos correspondientes.}
		\UCitem{Entradas}{Datos de la Unidad \ref{dd:Unidad}.}
		\UCitem{Salidas}{Mensaje de operacion exitosa\ref{MSG4}.}
		\UCitem{Precondiciones}{Que exista al menos un tipo de unidad registrado.}
		\UCitem{Postcondiciones}{Nueva Unidad registrada. La Unidad está disponible para ser utilizada.}
		\UCitem{Autor}{Aquino Carrillo Héctor Hyacinth.}
		\UCitem{Referencias}{SIDAM-BESP-P2-Especificación de Catálogos}
		\UCitem{Tipo}{Secundario. Viene del \UCref{CUA4}}
		\UCitem{Módulo}{Administración}
	\end{UseCase}

	\begin{UCtrayectoria}{Principal}
		\UCpaso[\UCactor] Oprime el botón \IUbutton{\includegraphics[scale=0.1]{images/icons/agregar.png}} en la página \IUref{IUGestUnidades}{Gestionar Unidades}.
		\UCpaso Busca los Tipos de unidad registrados \label{paso:CUA4.1buscaTiposUnidad}
		\UCpaso Muestra la pantalla \IUref{IUAgregarUnidad}{Agregar Unidad}.
		\UCpaso Ingresa Datos de la Unidad. \Trayref{A}\label{paso:CUA4.1ingresaDatosUnidad}
		\UCpaso [\UCactor] Oprime el botón \IUbutton{Aceptar}.
		\UCpaso Verifica que los datos correspondan con la definición en el diccionario de datos \ref{dd:Unidad}. \Trayref{B}
		\UCpaso Verifica que se cumpla la regla de negocio \BRref{RN2}\Trayref{C}
		\UCpaso Verifica que se cumpla la regla de negocio \BRref{RN1}\Trayref{D}
 		\UCpaso Registra la Unidad.
		\UCpaso Muestra el mensaje (MSG-4) indicando que se ha agregado correctamente la unidad.\ref{MSG4}
		\UCpaso Continúa en el paso \ref{paso:CUA4buscarUnidades} del \UCref{CUA4}.
	\end{UCtrayectoria}
		
	\begin{UCtrayectoriaA}{A}{Cancelar operación}{El usuario abandona el Caso de Uso.}
		\UCpaso[\UCactor] Decide ya no agregar una nueva Unidad.
		\UCpaso[\UCactor] Oprime el botón \IUbutton{Cancelar}.
			\UCpaso Continúa en el paso \ref{paso:CUA4buscarUnidades} del \UCref{CUA4}.
	\end{UCtrayectoriaA}
	\begin{UCtrayectoriaA}{B}{Datos Incorrectos de la Unidad }{Los datos de la Unidad no cumplen con lo especificado en el diccionario de datos.}
			\UCpaso Muestra el mensaje (MSG-2).\ref{MSG2} 
			\UCpaso Continúa en el paso \ref{paso:CUA4.1ingresaDatosUnidad} del \UCref{CUA4.1}.
	\end{UCtrayectoriaA}

	\begin{UCtrayectoriaA}{C}{Datos de unidad incompletos}{Algunos datos de la unidad no han sido capturados}
		\UCpaso Muestra el mensage (MSG-1).\ref{MSG1}
		\UCpaso Continúa en el paso \ref{paso:CUA4.1ingresaDatosUnidad} del \UCref{CUA4.1}	

	\end{UCtrayectoriaA}

	\begin{UCtrayectoriaA}{D}{Datos de la Unidad ya existen}{No cumple con la regla de negocio\BRref{RN1}}
			\UCpaso Muestra el mensaje (MSG-3) \ref{MSG3}	
			\UCpaso Continúa en el paso \ref{paso:CUA4.1ingresaDatosUnidad} del \UCref{CUA4.1}
	\end{UCtrayectoriaA}	






%--------- Modificar Unidad
	\begin{UseCase}{CUA4.2}{Modificar Unidad}{El usuario selecciona una Unidad registrada para modificar sus datos.}
			\UCitem{Versión}{2.0}
			\UCitem{Actor(es)}{Administrador.}
			\UCitem{Propósito}{Modificar los datos de una Unidad y actualizar su registro en el sistema.}
			\UCitem{Resumen}{El sistema muestra los datos de una Unidad seleccionada para la modificación de sus datos.}
			\UCitem{Entradas}{Datos modificados de la Unidad seleccionada \ref{dd:Unidad}.}
			\UCitem{Salidas}{Datos actuales de la Unidad. Mensaje de operación exitosa \ref{MSG4}.}
			\UCitem{Precondiciones}{Que exista al menos una unidad registrada.}
			\UCitem{Postcondiciones}{Se actualizan los datos de la Unidad seleccionada.}
			\UCitem{Autor}{Aquino Carrillo Héctor Hyacinth.}
			\UCitem{Referencias}{SIDAM-BESP-P2-Especificación de Catálogos}
			\UCitem{Tipo}{Secundario. Viene del \UCref{CUA4}}
			\UCitem{Módulo}{Administración}
	\end{UseCase}

	\begin{UCtrayectoria}{Principal}
			\UCpaso[\UCactor] Oprime el botón \IUbutton{\includegraphics[scale=0.1]{images/icons/editar.png}} de la pantalla \IUref{IUGestUnidades}{Gestionar Unidades}.
			\UCpaso Muestra la pantalla \IUref{IUModificarUnidad}{Modificar Unidad} con los datos de la Unidad seleccionada.
			\UCpaso Muestra el campo Tipo para solo lectura.
			\UCpaso [\UCactor] Ingresa los datos de la Unidad.\label{paso:CUA4.2ingresaDatosUnidad}\Trayref{A}
			\UCpaso [\UCactor] Oprime el botón \IUbutton{Aceptar}.
			\UCpaso Verifica que los datos esté de acuerdo al diccionario de datos \ref{dd:Unidad}. \Trayref{B} 
			\UCpaso Verifica la regla de negocio \BRref{RN2}\Trayref{C}
			\UCpaso Actualiza el registro de la Unidad.
			\UCpaso Muestra el mensaje (MSG-4) indicando que se ha modificado correctamente la unidad.\ref{MSG4}
			\UCpaso Continúa en el paso \ref{paso:CUA4buscarUnidades} del \UCref{CUA4}.
	\end{UCtrayectoria}

		\begin{UCtrayectoriaA}{A}{Cancelar operación}{El usuario abandona el Caso de Uso.}
			\UCpaso[\UCactor] Decide ya no modificar los datos de la Unidad.
			\UCpaso[\UCactor] Oprime el botón \IUbutton{Cancelar}.
			\UCpaso Continúa en el paso \ref{paso:CUA4buscarUnidades} del \UCref{CUA4}.
		\end{UCtrayectoriaA}
		\begin{UCtrayectoriaA}{B}{Los datos ingresados de la Unidad son incorrectos}{Los datos de la Unidad no cumplen con lo especificado en el diccionario de datos.}
			\UCpaso Muestra el mensaje (MSG-2).\ref{MSG2}
			\UCpaso Continúa en el paso \ref{paso:CUA4.2ingresaDatosUnidad} del \UCref{CUA4.2}.
		\end{UCtrayectoriaA}
		
		\begin{UCtrayectoriaA}{C}{Datos de la Unidad incompletos}{Algunos de los datos de la unidad hacen falta}
		 	\UCpaso Muestra el mensaje (MSG-1)\ref{MSG1}
			\UCpaso Continúa en el paso\ref{paso:CUA4.2ingresaDatosUnidad} del \UCref{CUA4.2}
		\end{UCtrayectoriaA}






%--------- Eliminar Unidad
	\begin{UseCase}{CUA4.3}{Eliminar Unidad}{El usuario elimina una Unidad registrada.}
			\UCitem{Versión}{2.0}
			\UCitem{Actor(es)}{Administrador.}
			\UCitem{Propósito}{Eliminar una Unidad.}
			\UCitem{Resumen}{El sistema muestra las Unidades registradas y permite seleccionar una Unidad para eliminarla.}
			\UCitem{Entradas}{Ninguna}
			\UCitem{Salidas}{Datos de la unidad bloqueados \ref{dd:Unidad}. Mensaje de operación exitosa \ref{MSG4}.}
			\UCitem{Precondiciones}{Que exista al menos una unidad registrada.}
			\UCitem{Postcondiciones}{La Unidad se elimina de los registros.}
			\UCitem{Autor}{Aquino Carrillo Héctor Hyacinth.}
			\UCitem{Referencias}{SIDAM-BESP-P2-Especificación de Catálogos}
			\UCitem{Tipo}{Secundario. Viene del \UCref{CUA4}}
			\UCitem{Módulo}{Administración}
	\end{UseCase}

	\begin{UCtrayectoria}{Principal}
 			
			\UCpaso[\UCactor] Oprime el botón \IUbutton{\includegraphics[scale=0.1]{images/icons/eliminar.png}} de la pantalla \IUref{IUGestUnidades}{Gestionar Unidades}.
			\UCpaso Muestra la pantalla \IUref{IUEliminarUnidad}{Eliminar Unidad} con los datos de la Unidad seleccionada.\label{paso:CUA4.3muestraUnidad}
			\UCpaso [\UCactor] Oprime el botón \IUbutton{Aceptar}. \Trayref{A}
			\UCpaso Verifica la regla de negocio \BRref{RN29}\Trayref{B}
			\UCpaso La Unidad seleccionado se elimina.
			\UCpaso Muestra el mensaje (MSG-4) indicando que se ha eliminado correctamente la unidad.\ref{MSG4}
			\UCpaso Continúa en el paso \ref{paso:CUA4buscarUnidades} del \UCref{CUA4}.
	\end{UCtrayectoria}

		
		\begin{UCtrayectoriaA}{A}{Cancelar operación}{El usuario abandona el Caso de Uso.}
			\UCpaso[\UCactor] Decide ya no eliminar la Unidad. \label{Datos_Asoc_Equipo}
			\UCpaso[\UCactor] Oprime el botón \IUbutton{Cancelar}.
			\UCpaso Continúa en el paso \ref{paso:CUA4buscarUnidades} del \UCref{CUA4}.
		\end{UCtrayectoriaA}
	
	
		\begin{UCtrayectoriaA}{B}{Datos Asociados}{La Unidad tiene Datos Asociados}
		 	\UCpaso El sistema lista las descripciones de los indicadores asociados
			\UCpaso El sistema muestra el mensaje (MSG-6C)\ref{MSG6C}
			\UCpaso [\UCactor] selecciona la unidad equivalente 
			\UCpaso Continúa en el paso \ref{paso:CUA4.3muestraUnidad} del \UCref{CUA4.3}			
		\end{UCtrayectoriaA}
		

		%\begin{UCtrayectoriaA}{B}{Datos Asociados}{El Equipo que se desea eliminar tiene datos que están asociados.}
		%	\UCpaso Muestra un mensaje indicando que no se puede eliminar porque tiene datos asociados.
		%	\UCpaso [\UCactor] Oprime el botón \IUbutton{Cancelar}.
		%	\UCpaso Continúa en el paso \ref{PECU14} del \UCref{CU14}.
		%\end{UCtrayectoriaA}
%-------------------------------------- TERMINA descripción del caso de uso.
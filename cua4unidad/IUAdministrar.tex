\subsection{Pantalla: Gestionar Unidades}

\subsubsection{Objetivo}
	Mostrar la información correspondiente a las Unidades a fin de mantener actualizada la información, ver Figura~\ref{IUGestUnidades}. Viene de Menú Gestión de Catálogos.

\IUfig[0.8]{cua4/AdministrarUnidad.png}{IUGestUnidades}{Gestionar Unidades.}

\subsubsection{Salidas}

	En esta pantalla se muestran los datos de las Unidades registradas en una tabla, ordenados por Nombre de forma ascendente.

\subsubsection{Controles}

	A la izquierda de cada Unidad aparece un {\em radio button} para seleccionar la Unidad con la que se desea trabajar.
	

\subsubsection{Comandos}
\begin{itemize}
 \item \IUbutton{\includegraphics[scale=0.1]{images/icons/agregar.png}} Esta opción permite registrar una nueva Unidad, al oprimirlo se mostrará la pantalla \IUref{IUAgregarUnidad}{Agregar Unidad}. Si la unidad se registra correctamente esta aparecerá en la tabla.
 
 \item \IUbutton{\includegraphics[scale=0.1]{images/icons/editar.png}}: Esta opción permite actualizar los datos de una Unidad, tras seleccionar una unidad esta opción lo llevará a la pantalla \IUref{IUModificarUnidad}{Modificar Unidad}. Si los datos son actualizados correctamente, los cambios se verán en la tabla.

 \item \IUbutton{\includegraphics[scale=0.1]{images/icons/eliminar.png}}: Esta opción permite eliminar la Unidad seleccionada, tras seleccionar una unidad esta opción lo llevará a la pantalla \IUref{IUEliminarUnidad}{Eliminar Unidad}. Si la unidad se elimina correctamente, esta deberá desaparecer de la tabla.
\end{itemize}


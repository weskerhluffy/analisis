% Descripción: Describe la funcionalidad ofrecida por el CU
% Propósito: Describe el objetivo o razón de ser del CU
% Resumen: Describe brevemente lo que hace el CU
%  
	\begin{UseCase}{CUC2.2.1}{Gestionar Acciones}{Gestiona un catálogo en el que se pueden realizar Altas, Bajas y Cambios de las Acciones.}
		\UCitem{Versión}{2}
		\UCitem{Actor(es)}{Coordinador.}
		\UCitem{Propósito}{Mantener organizada la información con respecto a las Acciones, así como visualizar las Acciones registradas.}
		\UCitem{Resumen}{Se muestran los Acciones registradas con la posibilidad de agregar, modificar y eliminar.}
		\UCitem{Entradas}{Ninguna.}
		\UCitem{Salidas}{Lista de las Acciones registradas.}
		\UCitem{Precondiciones}{Debe existir al menos un proyecto registrado en estado de edición.}
		\UCitem{Postcondiciones}{Ninguna.}
		\UCitem{Autor}{Hermosillo García Karen Adriana}
		\UCitem{Referencias}{SIDAM-BESP-P1-Especificación de Catálogos.¿Qué es un catalogo?}
		\UCitem{Tipo}{Secundario}
		\UCitem{Módulo}{Coordinación.}
	\end{UseCase}
	
	% 1.- escriba solo una trayectoria principal
	% 2.- El actor es quien siempre inicia un CU
	% 3.- evite usar ``si ... entonces ...'' o ``mientras ...'' o ``para cada ...''
	% 4.- No olvide mencionar todas las verificaciones y cálculos realizados por el sistema
	% 5.- Evite mencionar palabras como: Tabla, BD, conexión, etc.
		
	\begin{UCtrayectoria}{Principal}
		%\UCpaso[\UCactor] Selecciona la opción \IUbutton{Acción} del registro deseado en el menú \IUref{IUacciones}{Menú de gestión de proyectos} 
		\UCpaso Busca las Acciones registradas en el Proyecto seleccionado.\Trayref{A} \label{paso:CUC13buscarAcciones}
		\UCpaso Muestra en la pantalla \IUref{IUGestAcciones}{Gestionar Acciones} los datos de las acciones registradas en el proyecto y el diagrama de gantt.
		\UCpaso el sistema muestra las opciones de agregar, eliminar y modificar acciones, asi como reportar avances y definición de indicadores.\UCExtensionPoint{CUC2.2.1.1}{Agregar Acción}, \UCExtensionPoint{CUC2.2.1.2}{Modificar Acción}, \UCExtensionPoint{CUC2.2.1.3}{Eliminar Acción}, \UCExtensionPoint{CUC2.2.1.4}{Gestión de indicadores Fisicos}.
	\end{UCtrayectoria}
	
	\begin{UCtrayectoriaA}{A}{No existen datos de Acciones}{No se encontraron Acciones registradas}
			\UCpaso[\UCactor] Muestra el mensaje (MSG-5) indicando que no se encontrarón acciones registrados.\ref{MSG5}
			\UCpaso Continúa en el paso \ref{paso:CUC13buscarAcciones} del \UCref{CUG13}.
	\end{UCtrayectoriaA}

%--------- Agregar Acción
	\begin{UseCase}{CUC2.2.1.1}{Agregar Acción}{El Coordinador registra una nueva acción a un Proyecto.}
			\UCitem{Versión}{2}
			\UCitem{Actor(es)}{Coordinador.}
			\UCitem{Propósito}{Agregar una nueva Accion al Proyecto seleccionado por el Coordinador.}
			\UCitem{Resumen}{Se agrega una Accion registrando los datos correspondientes.}
			\UCitem{Entradas}{Datos de la acción \ref{dd:Accion}.}
			\UCitem{Salidas}{Mensaje de operación exitosa.}
			\UCitem{Precondiciones}{Debe existir al menos un proyecto en estado de edición.}
			\UCitem{Postcondiciones}{Nueva acción agregada al proyecto.}
			\UCitem{Autor}{Hermosillo García Karen Adriana.}
			\UCitem{Referencias}{SIDAM-BESP-P1-Especificación de Catálogos.}
			\UCitem{Tipo}{Secundario. Viene del \UCref{CUC2.2.1}.}
			\UCitem{Módulo}{Coordinación}
	\end{UseCase}

	\begin{UCtrayectoria}{Principal}
			\UCpaso[\UCactor] Selecciona la opción \IUbutton{\includegraphics[scale=0.15]{images/icons/agregar.png}} de la pantalla \IUref{IUGestAcciones}{Gestionar Acciones}.
			\UCpaso Muestra la pantalla \IUref{IUAgregarAccion}{Agregar Acción}.
			\UCpaso [\UCactor] Ingresa los datos de la Acción. \Trayref{A}\label{paso:CUC13.1ingresaDatosAccion}
			\UCpaso [\UCactor] Oprime el botón \IUbutton{Aceptar}.
			\UCpaso Verifica que se cumpla la regla de negocio \BRref{RN2}. \Trayref{B} 
			\UCpaso Revisa que los datos correspondan con el Diccionario de Datos. \ref{dd:Accion}. \Trayref{C}
			\UCpaso Verifica que se cumpla la regla de negocio \BRref{RN33}. \Trayref{D} 
			\UCpaso Verifica que se cumpla la regla de negocio \BRref{RN7}. \Trayref{E} 
			\UCpaso Registra la nueva Accion.
			\UCpaso Muestra el mensaje (MSG-4) indicando que se ha agregado correctamente el registro.\ref{MSG4}
			\UCpaso Continúa en el paso \ref{paso:CUC13buscarAcciones} del \UCref{CUC2.2.1.1}.
	\end{UCtrayectoria}

	\begin{UCtrayectoriaA}{A}{Cancelar operación}{El usuario abandona el Caso de Uso.}
			\UCpaso[\UCactor] Decide ya no agregar la acción.
			\UCpaso[\UCactor] Oprime el botón \IUbutton{Cancelar}.
			\UCpaso Continúa en el paso \ref{paso:CUC13buscarAcciones} del \UCref{CUC2.2.1}.
	\end{UCtrayectoriaA}
		
	\begin{UCtrayectoriaA}{B}{Datos de la Acción Incompletos}{Algunos campos de la acción no han sido capturados.}
			\UCpaso Muestra el mensaje (MSG-1).\ref{MSG1}
			\UCpaso Continúa en el paso \ref{paso:CUC13.1ingresaDatosAccion} del \UCref{CUC2.2.1.1}.
	\end{UCtrayectoriaA}

	\begin{UCtrayectoriaA}{C}{Datos de la acción Incorrectos}{Los datos de la acción no cumplen con lo especificado en el diccionario de datos.}
			\UCpaso Muestra el mensaje (MSG-2).\ref{MSG2}
			\UCpaso Continúa en el paso \ref{paso:CUC13.1ingresaDatosAccion} del \UCref{CUC2.2.1.1}.
	\end{UCtrayectoriaA}

		\begin{UCtrayectoriaA}{D}{La acción ya se encuentra registrada}{No cumple con la regla de negocios \BRref{RN33}}
			\UCpaso Muestra el mensaje (MSG-3) que indica que el nombre está repetido.\ref{MSG3}
			\UCpaso Continúa en el paso \ref{paso:CUC13.1ingresaDatosAccion} del \UCref{CUC2.2.1.1}.
		\end{UCtrayectoriaA}
		\begin{UCtrayectoriaA}{E}{Periodo inválido}{No cumple con la regla de negocios \BRref{RN7}}
			\UCpaso Muestra el mensaje  que indica que el perido es inválido
			\UCpaso Continúa en el paso \ref{paso:CUC13.1ingresaDatosAccion} del \UCref{CUC2.2.1.1}.
		\end{UCtrayectoriaA}

%--------- Modificar Acción
	\begin{UseCase}{CUC2.2.1.2}{Modificar acción}{El Coordinador selecciona una acción registrada para modificar sus datos.}
			\UCitem{Versión}{2}
			\UCitem{Actor(es)}{Coordinador.}
			\UCitem{Propósito}{Modificar los datos de una acción y actualizar el registro en el sistema.}
			\UCitem{Resumen}{El sistema despliega los Datos de las acciones registradas y permite seleccionar una acción para modificar sus datos, guardando los cambios.}
			\UCitem{Entradas}{Datos actuales de la acción seleccionada \ref{dd:Accion}.}
			\UCitem{Salidas}{Datos modificados de la acción seleccionada \ref{dd:Accion}.}
			\UCitem{Precondiciones}{Que exista al menos una acción registrada para un proyecto en estado de edición.}
			\UCitem{Postcondiciones}{La acción seleccionada se actualiza.}
			\UCitem{Autor}{Hermosillo García Karen Adriana}
			\UCitem{Referencias}{SIDAM-BESP-P1-Especificación de Catálogos.}
			\UCitem{Tipo}{Secundario. Viene del \UCref{CUC2.2.1}.}
			\UCitem{Módulo}{Coordinación.}
	\end{UseCase}

	\begin{UCtrayectoria}{Principal}
			\UCpaso[\UCactor] Selecciona la fila del registro que desea modificar en la pantalla \IUref{IUGestAcciones}{Gestionar Acciones}.	
			\UCpaso[\UCactor] Presiona el botón \IUbutton{Modificar}. 
			\UCpaso Muestra la pantalla \IUref{IUModificarAccion}{Modificar Acción} con los datos de la Acción seleccionada
                        \UCpaso [\UCactor] Modifica los datos de la acción.\label{paso:CUC13.2modificarDatosAccion}\Trayref{A}
			\UCpaso [\UCactor] Oprime el botón \IUbutton{Aceptar}.
			\UCpaso Verifica que se cumpla la regla de negocio \BRref{RN2}. \Trayref{B} 
			\UCpaso Revisa los datos de acuerdo al diccionario de datos \ref{dd:Accion}. \Trayref{C}
			\UCpaso Verifica que se cumpla la regla de negocio \BRref{RN33}. \Trayref{D} 
			\UCpaso Verifica que se cumpla la regla de negocio \BRref{RN7}. \Trayref{E} 
			\UCpaso Actualiza los datos de la Acción.
			\UCpaso Muestra el mensaje (MSG-4) indicando que se ha modificado correctamente el registro.\ref{MSG4}
			\UCpaso Continúa en el paso \ref{paso:CUC13buscarAcciones} del \UCref{CUC2.2.1}.
	\end{UCtrayectoria}

		\begin{UCtrayectoriaA}{A}{Cancelar operación}{El usuario abandona el Caso de Uso.}
			\UCpaso[\UCactor] Decide ya no modificar los datos de la Acción.
			\UCpaso[\UCactor] Oprime el botón \IUbutton{Cancelar}.
			\UCpaso Continúa en el paso \ref{paso:CUC13buscarAcciones} del \UCref{CUC2.2.1}.
		\end{UCtrayectoriaA}

	\begin{UCtrayectoriaA}{B}{Datos de la Acción Incompletos}{Algunos campos de la acción no han sido capturados.}
			\UCpaso Muestra el mensaje (MSG-1).\ref{MSG1}
			\UCpaso Continúa en el paso \ref{paso:CUC13.2modificarDatosAccion} del \UCref{CUC2.2.1.2}.
	\end{UCtrayectoriaA}

	\begin{UCtrayectoriaA}{C}{Datos de la Acción Incorrectos}{Los datos de la acción no cumplen con lo especificado en el diccionario de datos.}
			\UCpaso Muestra el mensaje (MSG-2).\ref{MSG2}
			\UCpaso Continúa en el paso \ref{paso:CUC13.2modificarDatosAccion} del \UCref{CUC2.2.1.2}.
	\end{UCtrayectoriaA}
		\begin{UCtrayectoriaA}{D}{La acción ya se encuentra registrada}{No cumple con la regla de negocios \BRref{RN33}}
			\UCpaso Muestra el mensaje (MSG-3) que indica que el nombre está repetido.\ref{MSG3}
			\UCpaso Continúa en el paso \ref{paso:CUC13.1ingresaDatosAccion} del \UCref{CUC2.2.1.1}.
		\end{UCtrayectoriaA}

		\begin{UCtrayectoriaA}{E}{Periodo inválido}{No cumple con la regla de negocios \BRref{RN7}}
			\UCpaso Muestra el mensaje  que indica que el perido es inválido
			\UCpaso Continúa en el paso \ref{paso:CUC13.2modificarDatosAccion} del \UCref{CUC2.2.1.2}.
		\end{UCtrayectoriaA}

%--------- Eliminar Acción
	\begin{UseCase}{CUC2.2.1.3}{Eliminar Acción}{El usuario elimina la acción perteneciente a un Proyecto.}
			\UCitem{Versión}{2}
			\UCitem{Actor(es)}{Coordinador}
			\UCitem{Propósito}{Eliminar una Acción.}
			\UCitem{Resumen}{El sistema despliega los Datos de las acciones registradas y permite seleccionar una  acción para eliminarlo.}
			\UCitem{Entradas}{Ninguna}
			\UCitem{Salidas}{Datos de la Acción seleccionada \ref{dd:Accion}. Mensaje de operación exitosa}
			\UCitem{Precondiciones}{Que exista al menos una acción registrada para un proyecto en estado de edición}
			\UCitem{Postcondiciones}{La acción se elimina permanente}
			\UCitem{Autor}{Hermosillo García Karen Adriana}
			\UCitem{Referencias}{SIDAM-BESP-P1-Especificación de Catálogos.}
			\UCitem{Tipo}{Secundario. Viene del \UCref{CUC2.2.1}.}
			\UCitem{Módulo}{Coordinación.}
	\end{UseCase}

	\begin{UCtrayectoria}{Principal}
			\UCpaso[\UCactor] Selecciona la fila del registro que desea eliminar en la pantalla \IUref{IUGestAcciones}{Gestionar Acciones}.
			\UCpaso[\UCactor] Presiona el boton \IUbutton{Eliminar}.
			\UCpaso Verifica que se cumpla la regla de negocio \BRref{RN56}. \Trayref{A}
			\UCpaso Muestra la pantalla \IUref{IUEliminarAccion}{Eliminar Acción} con los datos del Nivel seleccionado. \label{paso:CUG13.3eliminarDatosAccion}
			\UCpaso [\UCactor] Oprime el botón \IUbutton{Aceptar}. \Trayref{B}
			\UCpaso La acción seleccionada se elimina.
			\UCpaso Muestra el mensaje (MSG-4) indicando que se ha eliminado correctamente el registro.\ref{MSG4}
			\UCpaso Continúa en el paso \ref{paso:CUC13buscarAcciones} del \UCref{CUC2.2.1}.
	\end{UCtrayectoria}


		\begin{UCtrayectoriaA}{A}{Acción con avances}{La acción que se desea eliminar ya cuenta con avances reportados.}
			\UCpaso Muestra el mensaje  al Coordinador indicando que no se puede eliminar la acción ya que se han reportado avances.
			\UCpaso Continúa en el paso \ref{paso:CUG13.3eliminarDatosAccion} del \UCref{CUC2.2.1.3}.
		\end{UCtrayectoriaA}
		\begin{UCtrayectoriaA}{B}{Cancelar operación}{El coordinador abandona el Caso de Uso.}
			\UCpaso[\UCactor] Decide ya no eliminar la acción.
			\UCpaso[\UCactor] Oprime el botón \IUbutton{Cancelar}.
			\UCpaso Continúa en el paso \ref{paso:CUC13buscarAcciones} del \UCref{CUC2.2.1}.
		\end{UCtrayectoriaA}
		
%-------------------------------------- TERMINA descripción del caso de uso.
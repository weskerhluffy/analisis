% Descripción: Describe la funcionalidad ofrecida por el CU
% Propósito: Describe el objetivo o razón de ser del CU
% Resumen: Describe brevemente lo que hace el CU

	\begin{UseCase}{CU2.1.2.1}{Revisar Reportes de Avance de una Acción}{Permite ver los reportes de avance registrados de la acción.}
		\UCitem{Versión}{3.0}
		%\UCitem{Estado}{Finalizado}
		\UCitem{Actor(es)}{Director General, Gerente, Coordinador, Secretario}
		\UCitem{Propósito}{Contar con un mecanismo que ayude a revisar los reportes de avance de una acción.}
		\UCitem{Resumen}{Se muestran los datos de los indicadores y sus respectivas evidencias.}
		\UCitem{Entradas}{Ninguna.}
		\UCitem{Salidas}{Datos de los indicadores y las evidencias registradas.}
		\UCitem{Precondiciones}{Debe haber una acción seleccionada}
		\UCitem{Postcondiciones}{Ninguna.}
		\UCitem{Autor}{Santiago Alvarez Carlos Rogelio}
		\UCitem{Referencias}{Ninguna}
		\UCitem{Tipo}{Secundario. Viene de \UCref{CU2.1.2}}
		\UCitem{Módulo}{Coordinación, Gerencia.}
	\end{UseCase}
	
	% 1.- escriba solo una trayectoria principal
	% 2.- El actor es quien siempre inicia un CU
	% 3.- evite usar ``si ... entonces ...'' o ``mientras ...'' o ``para cada ...''
	% 4.- No olvide mencionar todas las verificaciones y cálculos realizados por el sistema
	% 5.- Evite mencionar palabras como: Tabla, BD, conexión, etc.
	
	\begin{UCtrayectoria}{Principal}
		\UCpaso[\UCactor] Selecciona la opción \IUbutton{Revisar Reportes de Avance} de una acción de la pantalla \IUref{IURevisarAvancesProyecto}{Revisar Avances de Proyecto}.
		\UCpaso Busca los indicadores en la Acción seleccionada y las evidencias registradas en cada indicador.\Trayref{A}\Trayref{B}
		\UCpaso Muestra en la pantalla \IUref{IURevisarReportesAvance}{Revisar Reportes de Avance}. el nombre de los indicadores registrados en la acción, el nombre de sus evidencias y la fecha en que se dieron de alta.\ref{dd:Indicador}
		\UCpaso Muestra el botón [Bajar Evidencia]. \UCExtensionPoint{CUC2.1.2.1.1}{Subir Evidencia} 
		
	\end{UCtrayectoria}
	
		\begin{UCtrayectoriaA}{A}{No hay indicadores registrados.}{No se encuentran indicadores registrados.}
			\UCpaso Se muestra el mensaje (MSG5) notificando al usuario que no se encontraron registros.\ref{MSG5}
			\UCpaso[\UCactor] Oprime el botón [REGRESAR].
			\UCpaso Regresa a la pantalla \IUref{IURevisarAvancesProyecto}{Revisar Avances de Proyecto}
		\end{UCtrayectoriaA}
    
    \begin{UCtrayectoriaA}{B}{El indicador no tiene evidencias registradas.}{No se encuentran evidencias registradas en el indicador.}
      \UCpaso No se muestra el registro del indicador sin evidencias.
    \end{UCtrayectoriaA}
%-------------------------------------- TERMINA descripción del caso de uso.
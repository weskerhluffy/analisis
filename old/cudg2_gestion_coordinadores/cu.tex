
%  
	\begin{UseCase}{CUG2}{Gestión de Coordinadores}{Gestiona un catálogo en el que se pueden realizar Altas, Bajas y Cambios de Coordinadores de direcciones.}
		\UCitem{Versión}{1.0}
		\UCitem{Actor(es)}{Director general}
		\UCitem{Propósito}{Permitir al Director General poder gestionar altas, bajas y cambios de coordinadores generales.}
		\UCitem{Resumen}{Se muestran los Coordinadores de dirección con la posibilidad de agregar, modificar y eliminar.}
		\UCitem{Entradas}{Ninguna.}
		\UCitem{Salidas}{Lista de Coordinadores de dirección.}
		\UCitem{Precondiciones}{
			\begin{itemize}
				\item Deben existir almacenados en el sistema Coordinadores de dirección.
			\end{itemize}
		}
		\UCitem{Postcondiciones}{Ninguna.}
		\UCitem{Autor}{Felipe.}
		\UCitem{Referencias}{}
		\UCitem{Tipo}{Primario.}
		\UCitem{Módulo}{Dirección General}
	\end{UseCase}
	
		
	\begin{UCtrayectoria}{Principal}
		\UCpaso[\UCactor] Selecciona la opción Gestión de Coordinadores del menu [Menu del director general]
		\UCpaso Busca todos los Coordinadores de dirección registrados. [Trayectoria A]
		\UCpaso Muestra la pantalla [Pantalla] con los datos de los Coordinadores de dirección encontrados.
	\end{UCtrayectoria}
	
	\begin{UCtrayectoriaA}{A}{No existen datos de Coordinadores}{No se encontraron Coordinadores registrados}
			\UCpaso Muestra el mensaje (MSG-X) indicando que no se encontraron Coordinadores registrados.\ref{X}
	\end{UCtrayectoriaA}

%--------- Agregar Nivel
	\begin{UseCase}{CUG2.1}{Agregar Coordinador}{El Director general registra un nuevo Coordinador.}
			\UCitem{Versión}{1}
			\UCitem{Actor(es)}{Director General.}
			\UCitem{Propósito}{Agregar un nuevo Coordinador.}
			\UCitem{Resumen}{Se agrega un nuevo Coordinador con sus datos correspondientes.}
			\UCitem{Entradas}{Datos del Coordinador, Proyecto }%\ref{dd:DatosEditablesNivel}.}
			\UCitem{Salidas}{Mensaje de operación exitosa (MSG-X)\ref{X}.}
			\UCitem{Precondiciones}{}
			\UCitem{Postcondiciones}{El nuevo Coordinador debe ser almacenado en los registros del Sistema y tener un proyecto asociado.}
			\UCitem{Autor}{Felipe.}
			\UCitem{Referencias}{}
			\UCitem{Tipo}{Secundario. Viene del \UCref{CUG2}.}
			\UCitem{Módulo}{Dirección General.}
	\end{UseCase}

	\begin{UCtrayectoria}{Principal}
			\UCpaso[\UCactor] Selecciona la opción \IUbutton{\includegraphics[scale=0.15]{images/icons/agregar.png}} de la pantalla \IUref{IUAdministrar}{Menú del Director General}.
			\UCpaso Busca las Direcciones registradas.
			\UCpaso Muestra la pantalla \IUref{IUAgregar}{Agregar Coordinador} que contiene una lista de Direcciones.
			\UCpaso [\UCactor] Ingresa los datos del Coordinador. \Trayref{A}\label{paso:CUG2.1.1IngresaDatosNivel}
			\UCpaso [\UCactor] Selecciona una Dirección.
			\UCpaso [\UCactor] Oprime el botón \IUbutton{Aceptar}.
			\UCpaso Verifica que se cumpla la regla de negocio \BRref{RN2}. \Trayref{B} 
			\UCpaso Revisa que los datos correspondan con el Diccionario de Datos. \ref{dd:DatosEditablesNivel}. \Trayref{C}
			\UCpaso Verif.ica que se cumpla la regla de negocio \BRref{RN37}. \Trayref{D} 
			\UCpaso Registra el nuevo Coordinador.
			\UCpaso Asocia la Dirección con el Coordinador registrado.
			\UCpaso Muestra el mensaje (MSG-4) indicando que se ha agregado correctamente el registro.\ref{MSG4}
			\UCpaso Continúa en el paso \ref{paso:CUG2.1buscarNiveles} del \UCref{CUG2}.
	\end{UCtrayectoria}

	\begin{UCtrayectoriaA}{A}{Cancelar operación}{El usuario abandona el Caso de Uso.}
			\UCpaso[\UCactor] Decide ya no agregar un nuevo Nivel.
			\UCpaso[\UCactor] Oprime el botón \IUbutton{Cancelar}.
			\UCpaso Continúa en el paso \ref{paso:CUG2.1buscarNiveles} del \UCref{CUG2.1}.
	\end{UCtrayectoriaA}
		.
	\begin{UCtrayectoriaA}{B}{Datos del Nivel Incompletos}{Algunos campos del Nivel no han sido capturados.}
			\UCpaso Muestra el mensaje (MSG-1).\ref{MSG1}
			\UCpaso Continúa en el paso \ref{paso:CUG2.1.1IngresaDatosNivel} del \UCref{CUG2.1.1}.
	\end{UCtrayectoriaA}

	\begin{UCtrayectoriaA}{C}{Datos del Nivel Incorrectos}{Los datos del Nivel no cumplen con lo especificado en el diccionario de datos.}
			\UCpaso Muestra el mensaje (MSG-2).\ref{MSG2}
			\UCpaso Continúa en el paso \ref{paso:CUG2.1.1IngresaDatosNivel} del \UCref{CUG2.1.1}.
	\end{UCtrayectoriaA}

		\begin{UCtrayectoriaA}{D}{El Nivel ya se encuentra registrado}{El Nivel que se desea agregar ya se encuentra registrado.}
			\UCpaso Muestra el mensaje (MSG-3) que indica que el nombre está repetido.\ref{MSG3}
			\UCpaso Continúa en el paso \ref{paso:CUG2.1.1IngresaDatosNivel} del \UCref{CUG2.1.1}.
		\end{UCtrayectoriaA}

%--------- Modificar Nivel
	\begin{UseCase}{CUG2.1}{Modificar Coordinador}{El Director General selecciona un Coordinador registrado para modificar sus datos.}
			\UCitem{Versión}{1.0}
			\UCitem{Actor(es)}{Director general}
			\UCitem{Propósito}{Modificar los datos de un Coordinador y actualizar el registro en el sistema.}
			\UCitem{Resumen}{El sistema despliega los Datos de los Coordinadores registrados y permite seleccionar un Coordinador para modificar sus datos, guardando los cambios.}
			\UCitem{Entradas}{Selección del Coordinador. Datos a actualizar del Coordinador seleccionado \ref{dd:Nivel}.}
			\UCitem{Salidas}{Datos actualizados del Coordinador \ref{dd:DatosEditablesNivel}.}
			\UCitem{Precondiciones}{}
			\UCitem{Postcondiciones}{Los datos del Coordinador seleccionado se actualiza.}
			\UCitem{Autor}{Ernesto Alvarado.}
			\UCitem{Referencias}{}
			\UCitem{Tipo}{Secundario. Viene del \UCref{CUG2}.}
			\UCitem{Módulo}{Dirección General.}
	\end{UseCase}

	\begin{UCtrayectoria}{Principal}			
			\UCpaso[\UCactor] Oprime el botón \IUbutton{\includegraphics[scale=0.1]{images/icons/editar.png}} del nivel que desea modificar en la pantalla \IUref{IURevisarPrograma}{Gestión del Programa}.	
			\UCpaso Muestra la pantalla \IUref{IUModificarNivel}{Modificar Nivel} con los datos del Nivel seleccionado y sujeto a la regla de negocio \BRref{RN38}.
                        \UCpaso [\UCactor] Ingresa los datos del Nivel.\label{paso:CUG2.1.2ingresaDatosNivel}\Trayref{A}
			\UCpaso [\UCactor] Oprime el botón \IUbutton{Aceptar}.
			\UCpaso Verifica que se cumpla la regla de negocio \BRref{RN2}. \Trayref{B} 
			\UCpaso Revisa los datos de acuerdo al diccionario de datos \ref{dd:Nivel}. \Trayref{C}
			\UCpaso Verifica que se cumpla las regla de negocio \BRref{RN37} . \Trayref{D} 
			\UCpaso Actualiza los datos del Nivel.
			\UCpaso Muestra el mensaje (MSG-4) indicando que se ha modificado correctamente el registro.\ref{MSG4}
			\UCpaso Continúa en el paso \ref{paso:CUG2.1buscarNiveles} del \UCref{CUG2.1}.
	\end{UCtrayectoria}

		\begin{UCtrayectoriaA}{A}{Cancelar operación}{El usuario abandona el Caso de Uso.}
			\UCpaso[\UCactor] Decide ya no modificar los datos del Nivel.
			\UCpaso[\UCactor] Oprime el botón \IUbutton{Cancelar}.
			\UCpaso Continúa en el paso \ref{paso:CUG2.1buscarNiveles} del \UCref{CUG2.1}.
		\end{UCtrayectoriaA}

	\begin{UCtrayectoriaA}{B}{Datos del Nivel Incompletos}{Algunos campos del NivelGerencia no han sido capturados.}
			\UCpaso Muestra el mensaje (MSG-1).\ref{MSG1}
			\UCpaso Continúa en el paso \ref{paso:CUG2.1.1IngresaDatosNivel} del \UCref{CUG2.1.1}.
	\end{UCtrayectoriaA}

	\begin{UCtrayectoriaA}{C}{Datos del Nivel Incorrectos}{Los datos del Nivel no cumplen con lo especificado en el diccionario de datos.}
			\UCpaso Muestra el mensaje (MSG-2).\ref{MSG2}
			\UCpaso Continúa en el paso \ref{paso:CUG2.1.1IngresaDatosNivel} del \UCref{CUG2.1.1}.
	\end{UCtrayectoriaA}

		\begin{UCtrayectoriaA}{D}{El Nivel ya se encuentra registrado}{El Nivel que se desea agregar ya se encuentra registrado.}
			\UCpaso Muestra el mensaje (MSG-3) que indica que el nombre está repetido.\ref{MSG3}
			\UCpaso Continúa en el paso \ref{paso:CUG2.1.1IngresaDatosNivel} del \UCref{CUG2.1.1}.
		\end{UCtrayectoriaA}

%--------- Eliminar Tipos Aviso
	\begin{UseCase}{CUG2.1.3}{Eliminar Nivel}{El usuario elimina un Nivel perteneciente al programa.}
			\UCitem{Versión}{1.0}
			\UCitem{Actor(es)}{Gerente.}
			\UCitem{Propósito}{Eliminar un Nivel.}
			\UCitem{Resumen}{El sistema despliega los Datos de los Niveles registrados y permite seleccionar un Nivel para eliminarlo.}
			\UCitem{Entradas}{Selección del nivel a borrar.}
			\UCitem{Salidas}{Datos del Nivel seleccionado \ref{dd:DatosEditablesNivel} y posición.}
			\UCitem{Precondiciones}{}
			\UCitem{Postcondiciones}{El Nivel se elimina de los registros.}
			\UCitem{Autor}{Ernesto Alvarado Gaspar.}
			\UCitem{Referencias}{}
			\UCitem{Tipo}{Secundario. Viene del \UCref{CUG2}.}
			\UCitem{Módulo}{Gerencia.}
	\end{UseCase}

	\begin{UCtrayectoria}{Principal}
			\UCpaso[\UCactor] Oprime el botón \IUbutton{\includegraphics[scale=0.1]{images/icons/eliminar.png}} del nivel que desea eliminar en la pantalla \IUref{IURevisarPrograma}{Gestión del Programa}.
			\UCpaso Verifica que se cumpla el segundo aspecto de la regla de negocio \BRref{RN39}.\Trayref{A}
			\UCpaso Verifica que se cumpla el primer aspecto de la regla de negocio \BRref{RN39}. \Trayref{C}
			\UCpaso Muestra la pantalla \IUref{IUEliminarNivel}{Eliminar Nivel} con los datos del Nivel seleccionado. 
			\UCpaso [\UCactor] Oprime el botón \IUbutton{Aceptar}. \Trayref{B}
			\UCpaso El Nivel seleccionado se elimina.
			\UCpaso Muestra el mensaje (MSG-4) indicando que se ha eliminado correctamente el registro.\ref{MSG4}
			\UCpaso Continúa en el paso \ref{paso:CUG2.1buscarNiveles} del \UCref{CUG2.1}.
	\end{UCtrayectoria}

		\begin{UCtrayectoriaA}{A}{No se puede eliminar el nivel seleccionado}{El nivel que ha sido seleccionado es un nivel intermedio del Programa.}
			\UCpaso Muestra al Director el mensaje (MSG-6D1) indicando que no se puede eliminar el nivel seleccionado.\ref{MSG-6D1}
			\UCpaso Continúa en el paso \ref{paso:CUG2.1buscarNiveles} del \UCref{CUG2.1}.
		\end{UCtrayectoriaA}

		\begin{UCtrayectoriaA}{B}{Cancelar operación}{El Gerente abandona el Caso de Uso.}
			\UCpaso[\UCactor] Decide ya no eliminar el Nivel.
			\UCpaso[\UCactor] Oprime el botón \IUbutton{Cancelar}.
			\UCpaso Continúa en el paso \ref{paso:CUG2.1buscarNiveles} del \UCref{CUG2.1}.
		\end{UCtrayectoriaA}
		
		\begin{UCtrayectoriaA}{C}{Datos Asociados}{El Nivel que se desea eliminar tiene datos que están asociados.}
			\UCpaso Muestra el mensaje (MSG-6D2) al Director indicando que no se puede eliminar el nivel ya que tiene datos asociados.\ref{MSG-6D2}
			\UCpaso Continúa en el paso \ref{paso:CUG2.1buscarNiveles} del \UCref{CUG2.1}.
		\end{UCtrayectoriaA}
%-------------------------------------- TERMINA descripción del caso de uso.
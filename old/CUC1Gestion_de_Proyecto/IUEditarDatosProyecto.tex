\subsection{Pantalla: Editar Datos de Proyecto}

\subsubsection{Objetivo}
El objetivo de esta pantalla es editar los datos de un Proyecto a cargo de un Coordinador y permitir actuazlizaciones a la información almacenada, ademas de asociar Ejes temáticos y Temas transversales así como Alinear el Proyecto con un elemento EP1N.

\IUfig[0.6]{CUC1/regProy.png}{IUEditarDatosProyecto}{Pantalla Editar Datos de Proyecto.}

\subsubsection{Salidas}
\begin{itemize}
 \item Datos del Proyecto.
 \item Lista de Ejes Temáticos.
 \item Lista de Temas Transversales.
 \item Lista de Programas.
 \item Lista de Estructuras por Programa.
 \item El mensaje de operación exitosa una si el proceso termino correctamente.
\end{itemize}

\subsubsection{Entradas}
\begin{itemize}
 \item Datos a modificar del Proyecto \ref{dd:DatosEditablesProyecto}.
 \item Identificador de los Ejes Temáticos seleccionados.
 \item Identificador de los Tema Transversales seleccionados.
 \item Indentificador de los Programas seleccionados.
 \item Indentificador de la Estructura del Programas.
\end{itemize}

\subsubsection{Comandos}
\begin{itemize}
 \item \IUbutton{Aceptar}: Registra los datos de un nuevo Proyecto y lleva a la pantalla \textit{PANTALLA DE DONDE SALE ESTA COSA!!!}.  El sistema mostrará los mensajes MSG-1  \ref{MSG1} en caso de datos incompletos (por la regla \BRref{RN2}), si no son correctos mostrará el mensaje MSG-2 \ref{MSG2} (por validación de Diccionario de Datos\ref{dd:DatosEditablesProyecto}), si la alineacion del proyecto es incorrecta mostrara el mensaje MSG-RN-43 por validacion de la \BRref{RN43}, si el periodo no esta bien definido mostrara el mensaje MSG-RN-32 por validacion de la \BRref{RN32}, si el periodo no es valido mostrara el mensaje MSG-RN-7 por validacion de la \BRref{RN7}.
 \item \IUbutton{Cancelar}: Regresa a la pantalla \textit{PANTALLA DE DONDE SALE ESTA COSA!!!}.
\end{itemize}
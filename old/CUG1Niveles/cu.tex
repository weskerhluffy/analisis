% Descripción: Describe la funcionalidad ofrecida por el CU
% Propósito: Describe el objetivo o razón de ser del CU
% Resumen: Describe brevemente lo que hace el CU
%  
	\begin{UseCase}{CUG1.1}{Gestionar Niveles}{Gestiona un catálogo en el que se pueden realizar Altas, Bajas y Cambios de los Niveles.}
		\UCitem{Versión}{1.1}
		\UCitem{Actor(es)}{Gerente.}
		\UCitem{Propósito}{Mantener organizada la información con respecto a los Niveles, así como visualizar los Niveles registrados.}
		\UCitem{Resumen}{Se muestran los Niveles registrados con la posibilidad de agregar, modificar y eliminar.}
		\UCitem{Entradas}{Ninguna.}
		\UCitem{Salidas}{Lista de los Niveles registrados.}
		\UCitem{Precondiciones}{Debe existir al menos un Programa registrado por el Gerente que inicio sesión.}
		\UCitem{Postcondiciones}{Ninguna.}
		\UCitem{Autor}{Omar Juárez.}
		\UCitem{Referencias}{SIDAM-BESP-P1-Especificación de Catálogos.}
		\UCitem{Tipo}{Secundario viene de CUD1 Gestión de Programas\UCref{CUG1}.}
		\UCitem{Módulo}{Gerencia.}
	\end{UseCase}
	
	% 1.- escriba solo una trayectoria principal
	% 2.- El actor es quien siempre inicia un CU
	% 3.- evite usar ``si ... entonces ...'' o ``mientras ...'' o ``para cada ...''
	% 4.- No olvide mencionar todas las verificaciones y cálculos realizados por el sistema
	% 5.- Evite mencionar palabras como: Tabla, BD, conexión, etc.
		
	\begin{UCtrayectoria}{Principal}
		\UCpaso[\UCactor] Selecciona un Programa del menú \IUref{IUGestProgramas1N}{Menú de gestión de programas} oprimiendo el botón \IUbutton{Nivel} del registro deseado.
		\UCpaso Busca los Niveles registrados en el Programa seleccionado.\Trayref{A} \label{paso:CUG1.1buscarNiveles}
		\UCpaso Muestra en la pantalla \IUref{IUGestNiveles}{Gestionar Niveles} los datos de los niveles registrados para el Programa de Primer Nivel seleccionado ordenados por posición y las opciones para Agregar, Modificar o Eliminar un Nivel.\UCExtensionPoint{CUG1.1.1}{Agregar Nivel} \UCExtensionPoint{CUG1.1.2}{Modificar Nivel} \UCExtensionPoint{CUG1.1.3}{Eliminar Nivel}.
	\end{UCtrayectoria}
	
	\begin{UCtrayectoriaA}{A}{No existen datos de Niveles}{No se encontraron Niveles registrados}
			\UCpaso[\UCactor] Muestra el mensaje (MSG-5) indicando que no se encontraron niveles registrados.\ref{MSG5}
	\end{UCtrayectoriaA}

%--------- Agregar Nivel
	\begin{UseCase}{CUG1.1.1}{Agregar Nivel}{El Gerente registra un nuevo Nivel a un Programa.}
			\UCitem{Versión}{1.1}
			\UCitem{Actor(es)}{Gerente.}
			\UCitem{Propósito}{Agregar un nuevo Nivel al Programa seleccionado por el Gerente.}
			\UCitem{Resumen}{Se agrega un Nivel registrando los datos correspondientes.}
			\UCitem{Entradas}{Datos del Nivel \ref{dd:Nivel}.}
			\UCitem{Salidas}{Programa seleccionado, Posición como valor autogenerado y el mensaje de operación exitosa (MSG-4)\ref{MSG4}.}
			\UCitem{Precondiciones}{Debe existir al menos un Programa registrado por el gerente que inicio sesión.}
			\UCitem{Postcondiciones}{Nuevo Nivel agregado al Programa.}
			\UCitem{Autor}{Pérez Domínguez Alberto Michel.}
			\UCitem{Referencias}{SIDAM-BESP-P1-Especificación de Catálogos.}
			\UCitem{Tipo}{Secundario. Viene del \UCref{CUG1.1}.}
			\UCitem{Módulo}{Gerencia.}
	\end{UseCase}

	\begin{UCtrayectoria}{Principal}
			\UCpaso[\UCactor] Selecciona la opción \IUbutton{\includegraphics[scale=0.15]{images/icons/agregar.png}} de la pantlalla \IUref{IUGestNiveles}{Gestionar Niveles}.
			\UCpaso Muestra la pantalla \IUref{IUAgregarNivel}{Agregar Nivel} con el nombre del Programa autocomplementado en el campo ``Programa'', `Posición` autogenerado y ambos campos bloqueados.
			\UCpaso [\UCactor] Ingresa los datos del Nivel. \Trayref{A}\label{paso:CUG1.1.1IngresaDatosNivel}
			\UCpaso [\UCactor] Oprime el botón \IUbutton{Aceptar}.
			\UCpaso Verifica que se cumpla la regla de negocio \BRref{RN2}. \Trayref{B} 
			\UCpaso Revisa que los datos correspondan con el Diccionario de Datos. \ref{dd:Nivel}. \Trayref{C}
			\UCpaso Verifica que se cumpla la regla de negocio \BRref{RN37}. \Trayref{D} 
			\UCpaso Registra el nuevo Nivel.
			\UCpaso Muestra el mensaje (MSG-4) indicando que se ha agregado correctamente el registro.\ref{MSG4}
			\UCpaso Continúa en el paso \ref{paso:CUG1.1buscarNiveles} del \UCref{CUG1.1}.
	\end{UCtrayectoria}

	\begin{UCtrayectoriaA}{A}{Cancelar operación}{El usuario abandona el Caso de Uso.}
			\UCpaso[\UCactor] Decide ya no agregar un nuevo Nivel.
			\UCpaso[\UCactor] Oprime el botón \IUbutton{Cancelar}.
			\UCpaso Continúa en el paso \ref{paso:CUG1.1buscarNiveles} del \UCref{CUG1.1}.
	\end{UCtrayectoriaA}
		
	\begin{UCtrayectoriaA}{B}{Datos del Nivel Incompletos}{Algunos campos del Nivel no han sido capturados.}
			\UCpaso Muestra el mensaje (MSG-1).\ref{MSG1}
			\UCpaso Continúa en el paso \ref{paso:CUG1.1.1IngresaDatosNivel} del \UCref{CUG1.1.1}.
	\end{UCtrayectoriaA}

	\begin{UCtrayectoriaA}{C}{Datos del Nivel Incorrectos}{Los datos del Nivel no cumplen con lo especificado en el diccionario de datos.}
			\UCpaso Muestra el mensaje (MSG-2).\ref{MSG2}
			\UCpaso Continúa en el paso \ref{paso:CUG1.1.1IngresaDatosNivel} del \UCref{CUG1.1.1}.
	\end{UCtrayectoriaA}

		\begin{UCtrayectoriaA}{D}{El Nivel ya se encuentra registrado}{El Nivel que se desea agregar ya se encuentra registrado.}
			\UCpaso Muestra el mensaje (MSG-3) que indica que el nombre está repetido.\ref{MSG3}
			\UCpaso Continúa en el paso \ref{paso:CUG1.1.1IngresaDatosNivel} del \UCref{CUG1.1.1}.
		\end{UCtrayectoriaA}

%--------- Modificar Nivel
	\begin{UseCase}{CUG1.1.2}{Modificar Nivel}{El Gerente selecciona un Nivel registrado para modificar sus datos.}
			\UCitem{Versión}{1.0}
			\UCitem{Actor(es)}{Gerente.}
			\UCitem{Propósito}{Modificar los datos de un Nivel y actualizar el registro en el sistema.}
			\UCitem{Resumen}{El sistema despliega los Datos de los Niveles registrados y permite seleccionar un Nivel para modificar sus datos, guardando los cambios.}
			\UCitem{Entradas}{Identificador del Nivel seleccionado.Datos a actualizar del Nivel seleccionado \ref{dd:Nivel}.}
			\UCitem{Salidas}{Datos actuales del Nivel seleccionado \ref{dd:Nivel}.}
			\UCitem{Precondiciones}{Que exista el registro del Nivel seleccionado.}
			\UCitem{Postcondiciones}{El Nivel seleccionado se actualiza.}
			\UCitem{Autor}{Pérez Domínguez Alberto Michel.}
			\UCitem{Referencias}{SIDAM-BESP-P1-Especificación de Catálogos.}
			\UCitem{Tipo}{Secundario. Viene del \UCref{CUG1.1}.}
			\UCitem{Módulo}{Gerencia.}
	\end{UseCase}

	\begin{UCtrayectoria}{Principal}			
			\UCpaso[\UCactor] Oprime el botón \IUbutton{\includegraphics[scale=0.1]{images/icons/editar.png}} de la fila del registro que desea modificar en la pantalla \IUref{IUGestNiveles}{Gestionar Niveles}.	
			\UCpaso Muestra la pantalla \IUref{IUModificarNivel}{Modificar Nivel} con los datos del Nivel seleccionado y sujeto a la regla de negocio \BRref{RN38}.
                        \UCpaso [\UCactor] Ingresa los datos del Nivel.\label{paso:CUG1.1.2ingresaDatosNivel}\Trayref{A}
			\UCpaso [\UCactor] Oprime el botón \IUbutton{Aceptar}.
			\UCpaso Verifica que se cumpla la regla de negocio \BRref{RN2}. \Trayref{B} 
			\UCpaso Revisa los datos de acuerdo al diccionario de datos \ref{dd:Nivel}. \Trayref{C}
			\UCpaso Verifica que se cumpla las regla de negocio \BRref{RN37} . \Trayref{D} 
			\UCpaso Actualiza los datos del Nivel.
			\UCpaso Muestra el mensaje (MSG-4) indicando que se ha modificado correctamente el registro.\ref{MSG4}
			\UCpaso Continúa en el paso \ref{paso:CUG1.1buscarNiveles} del \UCref{CUG1.1}.
	\end{UCtrayectoria}

		\begin{UCtrayectoriaA}{A}{Cancelar operación}{El usuario abandona el Caso de Uso.}
			\UCpaso[\UCactor] Decide ya no modificar los datos del Nivel.
			\UCpaso[\UCactor] Oprime el botón \IUbutton{Cancelar}.
			\UCpaso Continúa en el paso \ref{paso:CUG1.1buscarNiveles} del \UCref{CUG1.1}.
		\end{UCtrayectoriaA}

	\begin{UCtrayectoriaA}{B}{Datos del Nivel Incompletos}{Algunos campos del Nivel no han sido capturados.}
			\UCpaso Muestra el mensaje (MSG-1).\ref{MSG1}
			\UCpaso Continúa en el paso \ref{paso:CUG1.1.1IngresaDatosNivel} del \UCref{CUG1.1.1}.
	\end{UCtrayectoriaA}

	\begin{UCtrayectoriaA}{C}{Datos del Nivel Incorrectos}{Los datos del Nivel no cumplen con lo especificado en el diccionario de datos.}
			\UCpaso Muestra el mensaje (MSG-2).\ref{MSG2}
			\UCpaso Continúa en el paso \ref{paso:CUG1.1.1IngresaDatosNivel} del \UCref{CUG1.1.1}.
	\end{UCtrayectoriaA}

		\begin{UCtrayectoriaA}{D}{El Nivel ya se encuentra registrado}{El Nivel que se desea agregar ya se encuentra registrado.}
			\UCpaso Muestra el mensaje (MSG-3) que indica que el nombre está repetido.\ref{MSG3}
			\UCpaso Continúa en el paso \ref{paso:CUG1.1.1IngresaDatosNivel} del \UCref{CUG1.1.1}.
		\end{UCtrayectoriaA}

%--------- Eliminar Tipos Aviso
	\begin{UseCase}{CUG1.1.3}{Eliminar Nivel}{El usuario elimina un Nivel perteneciente a un Programa 1N registrado.}
			\UCitem{Versión}{1.0}
			\UCitem{Actor(es)}{Gerente.}
			\UCitem{Propósito}{Eliminar un Nivel.}
			\UCitem{Resumen}{El sistema despliega los Datos de los Niveles registrados y permite seleccionar un Nivel para eliminarlo.}
			\UCitem{Entradas}{Identificador del Eje Temático seleccionado.}
			\UCitem{Salidas}{Datos del Nivel seleccionado \ref{dd:Nivel}.}
			\UCitem{Precondiciones}{Que exista el registro del Nivel seleccionado.}
			\UCitem{Postcondiciones}{El Nivel se elimina de los registros.}
			\UCitem{Autor}{Pérez Domínguez Alberto Michel.}
			\UCitem{Referencias}{SIDAM-BESP-P1-Especificación de Catálogos.}
			\UCitem{Tipo}{Secundario. Viene del \UCref{CUG1.1}.}
			\UCitem{Módulo}{Gerencia.}
	\end{UseCase}

	\begin{UCtrayectoria}{Principal}
			\UCpaso[\UCactor] Oprime el botón \IUbutton{\includegraphics[scale=0.1]{images/icons/eliminar.png}} de la fila del registro que desea eliminar en la pantalla \IUref{IUGestNiveles}{Gestionar Niveles}.
			\UCpaso Verifica que se cumpla el segundo aspecto de la regla de negocio \BRref{RN39}.\Trayref{A}
			\UCpaso Muestra la pantalla \IUref{IUEliminarNivel}{Eliminar Nivel} con los datos del Nivel seleccionado. 
			\UCpaso [\UCactor] Oprime el botón \IUbutton{Aceptar}. \Trayref{B}
			\UCpaso Verifica que se cumpla el primer aspecto de la regla de negocio \BRref{RN39}. \Trayref{C}
			\UCpaso El Nivel seleccionado se elimina.
			\UCpaso Muestra el mensaje (MSG-4) indicando que se ha eliminado correctamente el registro.\ref{MSG4}
			\UCpaso Continúa en el paso \ref{paso:CUG1.1buscarNiveles} del \UCref{CUG1.1}.
	\end{UCtrayectoria}

		\begin{UCtrayectoriaA}{A}{No se puede eliminar el nivel seleccionado}{El nivel que ha sido seleccionado es un nivel intermedio del Programa.}
			\UCpaso Muestra al Director el mensaje (MSG-6D1) indicando que no se puede eliminar el nivel seleccionado.\ref{MSG-6D1}
			\UCpaso Continúa en el paso \ref{paso:CUG1.1buscarNiveles} del \UCref{CUG1.1}.
		\end{UCtrayectoriaA}

		\begin{UCtrayectoriaA}{B}{Cancelar operación}{El Gerente abandona el Caso de Uso.}
			\UCpaso[\UCactor] Decide ya no eliminar el Nivel.
			\UCpaso[\UCactor] Oprime el botón \IUbutton{Cancelar}.
			\UCpaso Continúa en el paso \ref{paso:CUG1.1buscarNiveles} del \UCref{CUG1.1}.
		\end{UCtrayectoriaA}
		
		\begin{UCtrayectoriaA}{C}{Datos Asociados}{El Nivel que se desea eliminar tiene datos que están asociados.}
			\UCpaso Muestra el mensaje (MSG-6D2) al Director indicando que no se puede eliminar el nivel ya que tiene datos asociados.\ref{MSG-6D2}
			\UCpaso Continúa en el paso \ref{paso:CUG1.1buscarNiveles} del \UCref{CUG1.1}.
		\end{UCtrayectoriaA}
%-------------------------------------- TERMINA descripción del caso de uso.
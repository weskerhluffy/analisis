% Descripción: Describe la funcionalidad ofrecida por el CU
% Propósito: Describe el objetivo o razón de ser del CU
% Resumen: Describe brevemente lo que hace el CU

\begin{UseCase}{CUA2.4}{Gestionar Tipo de Contacto}{Ofrece un mecanismo para ver los Tipos de Contactos registrados y brinda opciones para agregar, eliminar y modificar estos Tipos de Contactos.}
		\UCitem{Versión}{0.1}
		\UCitem{Actor(es)}{Administrador}
		\UCitem{Propósito}{Contar con un mecanismo que ayude a Gestionar los Tipos de Contactos que se le pueden asignar a los Contactos.}
		\UCitem{Resumen}{Se muestran los diferentes {\bf Tipos de Contacto} registrados para los Contactos con la posibilidad de agregar, modificar y eliminar.}
		\UCitem{Entradas}{Ninguna.}
		\UCitem{Salidas}{Lista con los Tipos de Contacto y una descripción.}
		\UCitem{Precondiciones}{Ninguna}
		\UCitem{Postcondiciones}{Ninguna.}
		\UCitem{Autor}{Sánchez Pacheco Humberto.}
		\UCitem{Referencias}{SIDAM-BESP-P1-Especificación de Catálogos}
		\UCitem{Tipo}{Primario.}
		\UCitem{Módulo}{Administracion}
	\end{UseCase}
	
	
	\begin{UCtrayectoria}{Principal}
		\UCpaso[\UCactor] Selecciona la opción \IUbutton{Gestionar Tipo de Contacto} de la pantalla \IUref{IUMenuAdministrador}{Menú del Administrador}.
		\UCpaso Busca los Tipos de Contacto registrados.
		\UCpaso Muestra en la pantalla \IUref{IUGestTipoContactos}{Gestionar Tipo de Contacto.} los datos de los Tipos de Contacto se muestran registrados ordenados por nombre y las opciones para Agregar, Modificar o Eliminar un Tipo de Contacto.\UCExtensionPoint{CUA2.4.1}{Agregar Tipo de Contacto} \UCExtensionPoint{CUA2.4.2}{Modificar Tipo de Contacto} \UCExtensionPoint{CUA2.4.3}{Eliminar Tipo de Contacto}\label{PECU12}
	\end{UCtrayectoria}


%--------- Agregar Tipo de Contacto
	
	\begin{UseCase}{CUA2.4.1}{Agregar Tipo de Contacto}{El administrador agrega un nuevo Tipo de Contacto.}
			\UCitem{Versión}{0.1}
			\UCitem{Actor(es)}{Administrador}
			\UCitem{Propósito}{Agregar un nuevo Tipo de Contacto.}
			\UCitem{Resumen}{Se agrega un nuevo Tipo de Contacto llenando los registros correspondientes.}
			\UCitem{Entradas}{Datos del Tipo de Contacto \ref{dd:TipoContacto}.}
			\UCitem{Salidas}{Mensaje de operación exitosa.}
			\UCitem{Precondiciones}{Ninguna.}
			\UCitem{Postcondiciones}{Nuevo Tipo de Contacto agregado. El Tipo de Contacto ahora está disponible para ser utilizado.}
			\UCitem{Autor}{Sánchez Pacheco Humberto.}
			\UCitem{Referencias}{SIDAM-BESP-P1-Especificación de Catálogos}
			\UCitem{Tipo}{Secundario. Viene del \UCref{CUA2.4}}
			\UCitem{Módulo}{Administracion}
	\end{UseCase}

	\begin{UCtrayectoria}{Principal}
			\UCpaso[\UCactor] Selecciona la opción \IUbutton{\includegraphics[scale=0.1]{images/icons/agregar.png}} de la pantalla \IUref{IUGestTipoContacto}{Gestionar Tipo de Contacto}.
			\UCpaso Muestra la pantalla \IUref{IUAgregarTipoContacto}{Agregar Tipo de Contacto}.
			\UCpaso [\UCactor] Ingresa los datos del nuevo Tipo de Contacto\ref{dd:TipoContacto}.
			\UCpaso [\UCactor] Oprime el botón \IUbutton{Aceptar}.\Trayref{A}\label{PCU12-5}
                        \UCpaso Valida de acuerdo a la regla de negocio \BRref{RN2}. \Trayref{B} 
                        \UCpaso Valida de acuerdo al diccionario de datos. \Trayref{C}\ref{dd:TipoContacto}.
                        \UCpaso Valida de acuerdo a la regla de negocio \BRref{RN1}. \Trayref{D}
			\UCpaso Registra el nuevo Tipo de Contacto.
			\UCpaso Regresa al paso 3 de \UCref{CUA2.4}mostrando el mensaje (MSG-4).\ref{MSG4}
	\end{UCtrayectoria}

	\begin{UCtrayectoriaA}{A}{Cancelar operación}{El Administrador abandona el Caso de Uso.}
			\UCpaso[\UCactor] Decide ya no agregar el Nuevo Tipo de Contacto.
			\UCpaso[\UCactor] Oprime el botón \IUbutton{Cancelar}.
			\UCpaso Regresa a la pantalla anterior.
	\end{UCtrayectoriaA}

        \begin{UCtrayectoriaA}{B}{El Administrador ingresa datos que infrinjen alguna regla de negocio}{Los datos ingresados infringe la regla de negocio RN2.}
                        \UCpaso Muestra un mensaje de error (MSG-1) indicando al Administrador que no ingreso todos los campos.\ref{MSG1}.
			\UCpaso[\UCactor] Ingresa los datos correspondientes.
			\UCpaso Continúa en el paso \ref{PCU12-5} del \UCref{CU12}.
	\end{UCtrayectoriaA}
		
        \begin{UCtrayectoriaA}{C}{El Administrador infringe el diccionario de datos}{Los datos ingresados no concuerdan con el diccionario de datos}
                        \UCpaso Muestra un mensaje de error (MSG-1) indicando al Administrador los campos que no cumplen con lo especificado en el diccionario de datos.\ref{MSG1}.
			\UCpaso[\UCactor] Ingresa los datos correspondientes
			\UCpaso Continúa en el paso \ref{PCU12-5} del \UCref{CU12}.
	\end{UCtrayectoriaA}

        \begin{UCtrayectoriaA}{D}{El Administrador ingresa datos que infrinjen alguna regla de negocio}{El nombre del Tipo de contacto ya existe}
                        \UCpaso Muestra el mensaje (MSG-3) indicando que el Tipo de Contacto ya esta registrado.\ref{MSG3}
			\UCpaso[\UCactor] Ingresa otro Tipo de Contacto que no este registrado.
			\UCpaso Continúa en el paso \ref{PCU12-5} del \UCref{CU12}.
	\end{UCtrayectoriaA}



%--------- Modificar Tipo de Contacto
	\begin{UseCase}{CUA2.4.2}{Modificar Tipo de Contacto}{El Administrador modifica los campos de un Tipo de Contacto registrado.}
			\UCitem{Versión}{1.0}
			\UCitem{Actor(es)}{Administrador.}
			\UCitem{Propósito}{Modificar los datos de un Tipo de Contacto.}
			\UCitem{Resumen}{Se muestra la información de un Tipo de Contacto registrado para modificar sus datos.}
			\UCitem{Entradas}{Identificadir del Tipo de Contacto y los datos a modificar del Tipo de Contacto \ref{dd:TipoContacto}.}
			\UCitem{Salidas}{Datos actuales del Tipo de contacto \ref{dd:TipoContacto} y el mensaje de operación exitosa.}
			\UCitem{Precondiciones}{Que el Tipo de Contaco que se desea modificar se encuentre registrado.}
			\UCitem{Postcondiciones}{Se actualizan los datos del Tipo de Contacto modificado.}
			\UCitem{Autor}{Sánchez Pacheco Humberto.}
			\UCitem{Referencias}{SIDAM-BESP-P1-Especificación de Catálogos}
			\UCitem{Tipo}{Secundario. Viene del \UCref{CUA2.4}}
			\UCitem{Módulo}{Administacion}
	\end{UseCase}

	\begin{UCtrayectoria}{Principal}
			\UCpaso[\UCactor] Oprime el botón \IUbutton{\includegraphics[scale=0.1]{images/icons/editar.png}} del Tipo de Contacto a modificar de la pantalla \IUref{IUGestTipoContacto}{Gestionar Tipo de Contacto}.
			\UCpaso Muestra la pantalla \IUref{IUModificarTipoContacto}{Modificar Tipo de Contacto}.
                        \UCpaso [\UCactor] Modifica los datos del Tipo de Contacto.
                       	\UCpaso [\UCactor] Oprime el botón \IUbutton{Aceptar}.\Trayref{A}
                        \UCpaso Valida de acuerdo a la regla de negocio \BRref{RN2}. \Trayref{B} 
                        \UCpaso Valida de acuerdo al diccionario de datos. \Trayref{C}\ref{dd:TipoContacto}.
			\UCpaso Actualiza los datos del Tipo de Contacto modificado.
			\UCpaso Regresa al paso 3 de \UCref{CUA2.4} mostrando el mensaje MSG-4.\ref{MSG4}
	\end{UCtrayectoria}

		\begin{UCtrayectoriaA}{A}{Cancelar operación}{El Administrador abandona el Caso de Uso.}
			\UCpaso[\UCactor] Decide ya no modificar los datos del Tipo de Contacto.
			\UCpaso[\UCactor] Oprime el botón \IUbutton{Cancelar}.
			\UCpaso Regresa a la pantalla anterior.
		\end{UCtrayectoriaA}
		
        \begin{UCtrayectoriaA}{B}{El Administrador ingresa datos que infrinje alguna regla de negocio}{Los datos ingresados infringe la regla de negocio RN2.}
                        \UCpaso Muestra un mensaje de error (MSG-1) indicando al Administrador que no ingreso todos los campos.\ref{MSG1}.
			\UCpaso[\UCactor] Ingresa los datos correspondientes.
			\UCpaso Continúa en el paso \ref{PCU12-5} del \UCref{CUA2.4}.
	\end{UCtrayectoriaA}

        \begin{UCtrayectoriaA}{C}{El Administrador ingresa datos que no corresponde con el diccionario de datos}{Los datos ingresados no concuerdan con el diccionario de datos}
                        \UCpaso Muestra un mensaje de error (MSG-1) indicando al Administrador los campos que no cumplen con lo especificado en el diccionario de datos.\ref{MSG1}.
			\UCpaso[\UCactor] Ingresa los datos correspondientes
			\UCpaso Continúa en el paso \ref{PCU12-5} del \UCref{CUA2.4}.
	\end{UCtrayectoriaA}

%--------- Eliminar Tipo de Contacos
	\begin{UseCase}{CU12.3}{Eliminar Tipo de Contacto}{El Administrador elimina un Tipo de Contacto registrado.}
			\UCitem{Versión}{1.0}
			\UCitem{Actor(es)}{Administrador.}
			\UCitem{Propósito}{Eliminar un Tipo de Contacto.}
			\UCitem{Resumen}{El sistema despliega los Tipos de Contacto registrados para Eliminar el Tipo de Contacto deseado.}
			\UCitem{Entradas}{Identificador del Tipo de Contacto seleccionado.}
			\UCitem{Salidas}{Datos del Tipo de Contacto \ref{dd:TipoContacto} y mensaje de operación exitosa.}
			\UCitem{Precondiciones}{Que el Tipo de Contaco que se desea eliminar se encuentre registrado y sea diferente de los Tipos de Contacto Telefono y Correo.}
			\UCitem{Postcondiciones}{El Tipo de Contacto se elimina del registro.}
			\UCitem{Autor}{Sánchez Pacheco Humberto}
			\UCitem{Referencias}{SIDAM-BESP-P1-Especificación de Catálogos}
			\UCitem{Tipo}{Secundario. Viene del \UCref{CUA2.4}}
			\UCitem{Módulo}{Gerencia.}
	\end{UseCase}

	\begin{UCtrayectoria}{Principal}
			\UCpaso[\UCactor] Selecciona el Tipo de Contacto que desea eliminar en la pantalla \IUref{IUGestTipoContacto}{Gestionar Tipo de Contacto}.
			\UCpaso[\UCactor] Oprime el botón \IUbutton{\includegraphics[scale=0.1]{images/icons/eliminar.png}} del Tipo de contacto a Eliminar de la pantalla \IUref{IUGestTipoContacto}{Gestionar Tipo de Contacto}.
			\UCpaso Muestra la pantalla \IUref{IUEliminarTipoContacto}{Eliminar Tipo Contacto.}
			\UCpaso [\UCactor] Oprime el botón \IUbutton{Aceptar}.\Trayref{A}
			\UCpaso Muestra la pantalla \IUref{IUEliminarTipoContacto2}{Confirmar Eliminar Tipo de Contacto.}
			\UCpaso [\UCactor] Oprime el botón \IUbutton{Aceptar}.\Trayref{A} 
                        \UCpaso Valida de acuerdo a la regla de negocio \BRref{RN29}(1). \Trayref{B} 
			\UCpaso Valida de acuerdo a la regla de negocio \BRref{RN29}(2).  \Trayref{C}
                        \UCpaso Elimina el Tipo de Contacto seleccionado de los registros.
			\UCpaso Muestra el mensaje (MSG-4) de operación exitosa.\ref{MSG4}
			\UCpaso Continúa en el paso 3 del \UCref{CUA2.4}.
	\end{UCtrayectoria}

		\begin{UCtrayectoriaA}{A}{Cancelar operación}{El Administrador abandona el Caso de Uso.}
			\UCpaso[\UCactor] Decide ya no eliminar el Tipo de Contacto.
			\UCpaso[\UCactor] Oprime el botón \IUbutton{Cancelar}.
			\UCpaso Regresa a la pantalla anterior.
		\end{UCtrayectoriaA}

		\begin{UCtrayectoriaA}{B}{Hay contactos asignados al Tipo de Contacto}{El Tipo de Contacto esta asignado a uno o mas contactos, por lo que no se puede elminar el Perfil.}
			\UCpaso Muestra el mensaje MSG-6B. \ref{MSG6B}.
			\UCpaso Regresa a la pantalla anterior.
		\end{UCtrayectoriaA}	

		\begin{UCtrayectoriaA}{C}{Los tipos de Contacto son restringidos}{El Tipo de Contacto que se desea eliminar es obligatorio, por lo tanto no se pued eliminar, regla de negocio \BRref{29}}
			\UCpaso Muestra el mensaje MSG-6B.  \ref{MSG6B}.
			\UCpaso Regresa a la pantalla anterior.
		\end{UCtrayectoriaA}	
		
%-------------------------------------- TERMINA descripción del caso de uso.

% Descripción: Describe la funcionalidad ofrecida por el CU
% Propósito: Describe el objetivo o razón de ser del CU
% Resumen: Describe brevemente lo que hace el CU

	\begin{UseCase}{CUD2}{Gestionar Tema Transversal}{Gestiona un catálogo en el que se pueden realizar Altas, Bajas y Cambios de los Temas transversales.}
		\UCitem{Versión}{1.1}
		\UCitem{Actor(es)}{Secretario.}
		\UCitem{Propósito}{Mantener organizada la información con respecto a los Temas transversales, así como visualizar los Temas transversales registrados.}
		\UCitem{Resumen}{Se muestran los Temas transversales registrados con la posibilidad de agregar, modificar y eliminar.}
		\UCitem{Entradas}{Ninguna}
		\UCitem{Salidas}{Lista de los Temas transversales registrados.}
		\UCitem{Precondiciones}{Ninguna.}
		\UCitem{Postcondiciones}{Ninguna.}
		\UCitem{Autor}{Reyes De los Santos Miguel Ángel.}
		\UCitem{Referencias}{SIDAM-BESP-P1-Especificación de Catálogos.}
		\UCitem{Tipo}{Primario.}
		\UCitem{Módulo}{Dirección.}
	\end{UseCase}
	
	\begin{UCtrayectoria}{Principal}
		\UCpaso[\UCactor] Oprime el botón \IUbutton{Gestionar Temas Transversales} en el menú \IUref{IUMenuDirector}{Menú para Secretarios.}.
		\UCpaso Busca los Temas transversales registrados. \label{paso:CUD2buscarTemasTransversales} 
		\UCpaso Muestra en la pantalla \IUref{IUGestTemasTransversales}{Gestionar Temas Transversales} los datos de los temas transversales registrados ordenados por nombre y las opciones para Agregar, Modificar o Eliminar un registro seleccionado.\UCExtensionPoint{CUD2.1}{Agregar Tema Transversal} \UCExtensionPoint{CUD2.2}{Modificar Tema Transversal} \UCExtensionPoint{CUD2.3}{Eliminar Tema Transversal}  \Trayref{A}
	\end{UCtrayectoria}

		\begin{UCtrayectoriaA}{A}{No existen datos de Tema Transversal}{No se encontraron Temas Transversales registrados}
			\UCpaso Muestra el mensaje (MSG-5).\ref{MSG5}			
	\end{UCtrayectoriaA}
	  	
%--------- Agregar Tipo de Aviso
	\begin{UseCase}{CUD2.1}{Agregar Tema Transversal}{El usuario registra un nuevo Tema Transversal con sus respectivos datos.}
			\UCitem{Versión}{1.2}
			\UCitem{Actor(es)}{Secretario.}
			\UCitem{Propósito}{Agregar un nuevo Tema Transversal.}
			\UCitem{Resumen}{Se agrega un Tema Transversal registrando sus datos correspondientes.}
			\UCitem{Entradas}{Datos del Tema Transversal \ref{dd:TemaTransversal}.}
			\UCitem{Salidas}{Nuevo registro del Tema Transversal y el mensaje de operación exitosa MSG4\ref{MSG4}.}
			\UCitem{Precondiciones}{Ninguna}
			\UCitem{Postcondiciones}{Tema Transversal registrado. El Tema Transversal está disponible para ser utilizado.}
			\UCitem{Autor}{Reyes De los Santos Miguel Ángel.}
			\UCitem{Referencias}{SIDAM-BESP-P1-Especificación de Catálogos.}
			\UCitem{Tipo}{Secundario. Viene del \UCref{CUD2}.}
			\UCitem{Módulo}{Secretaría.}
	\end{UseCase}

	\begin{UCtrayectoria}{Principal}
			\UCpaso[\UCactor] Selecciona la opción \IUbutton{\includegraphics[scale=0.15]{images/icons/agregar.png}} de la pantalla \IUref{IUGestTemasTransversales}{Gestionar Temas Transversales}.
			\UCpaso Muestra la pantalla \IUref{IUAgregarTemaTransversal}{Agregar Tema Transversal}.
			\UCpaso [\UCactor] Ingresa los datos del Tema Transversal. \Trayref{A} \label{paso:CUD2.1ValidacionDatos_TemaTransversal}
			\UCpaso [\UCactor] Oprime el botón \IUbutton{Aceptar}.
			\UCpaso Verifica que se cumpla las regla de negocio \BRref{2}. \Trayref{B} 
			\UCpaso Revisa que los datos correspondan con la Definición de Datos \ref{dd:TemaTransversal}. \Trayref{C}
			\UCpaso Verifica que se cumpla la regla de negocio \BRref{1}. \Trayref{D} 
			\UCpaso Registra el nuevo Tema Transversal.
			\UCpaso Muestra el mensaje (MSG-4) indicando que se ha agregado correctamente el registro.\ref{MSG4}
	\end{UCtrayectoria}

	\begin{UCtrayectoriaA}{A}{Cancelar operación}{El usuario abandona el Caso de Uso.}
			\UCpaso[\UCactor] Decide ya no agregar un nuevo Tema Transversal.
			\UCpaso[\UCactor] Oprime el botón \IUbutton{Cancelar}.
			\UCpaso Continúa en el paso \ref{paso:CUD2buscarTemasTransversales} del \UCref{CUD2}.
	\end{UCtrayectoriaA}
		
	\begin{UCtrayectoriaA}{B}{Datos del Tema Transversal Incompletos}{Existen campos nulos en el formulario de registro.}
			\UCpaso Muestra un mensaje (MSG-1) indicando que los datos ingresados estan incompletos.\ref{MSG1}
			\UCpaso Continúa en el paso \ref{paso:CUD2.1ValidacionDatos_TemaTransversal} del \UCref{CUD2.1}.
	\end{UCtrayectoriaA}

	\begin{UCtrayectoriaA}{C}{Datos del Tema Transversal Incorrectos}{Los datos del Tema Transversal no cumplen con lo especificado en el diccionario de datos.}
			\UCpaso Muestra un mensaje (MSG-2) indicando que los datos ingresados son incorrectos.\ref{MSG2}
			\UCpaso Continúa en el paso \ref{paso:CUD2.1ValidacionDatos_TemaTransversal} del \UCref{CUD2.1}.
	\end{UCtrayectoriaA}

		\begin{UCtrayectoriaA}{D}{El Tema Transversal ya se encuentra registrado}{El nombre del Tema Transversal que se desea agregar ya se encuentra registrado.}
			\UCpaso Muestra un mensaje (MSG-3) indicando que el Nombre ya existe.\ref{MSG3}
			\UCpaso Continúa en el paso \ref{paso:CUD2.1ValidacionDatos_TemaTransversal} del \UCref{CUD2.1}.
		\end{UCtrayectoriaA}


%--------- Modificar Tipos Aviso
	\begin{UseCase}{CUD2.2}{Modificar Tema Transversal}{El usuario selecciona un Tema Transversal registrado para modificar sus datos.}
			\UCitem{Versión}{1.2}
			\UCitem{Actor(es)}{Secretario.}
			\UCitem{Propósito}{Modificar los datos de un Tema Transversal y actualizar el registro en el sistema.}
			\UCitem{Resumen}{El sistema despliega los Datos de los Temas Transversales registrados y permite seleccionar un Tema Transversal para modificar sus datos, guardando los cambios.}
			\UCitem{Entradas}{Identificador del Tema Transversal seleccionado. Datos a actualizar del Tema Transversal \ref{dd:TemaTransversal}.}
			\UCitem{Salidas}{Datos del Tema Transversal seleccionado \ref{dd:TemaTransversal}.}
			\UCitem{Precondiciones}{Que exista el registro del Tema Transversal seleccionado.}
			\UCitem{Postcondiciones}{El Tema Transversal seleccionado se actualiza.}
			\UCitem{Autor}{Reyes De los Santos Miguel Ángel.}
			\UCitem{Referencias}{SIDAM-BESP-P1-Especificación de Catálogos.}
			\UCitem{Tipo}{Secundario. Viene del \UCref{CUD2}.}
			\UCitem{Módulo}{Secretaría.}
	\end{UseCase}

	\begin{UCtrayectoria}{Principal}			
			\UCpaso[\UCactor] Oprime el botón \IUbutton{\includegraphics[scale=0.1]{images/icons/editar.png}} de la fila del registro que desea modificar en la pantalla \IUref{IUGestTemasTransversales}{Gestionar Temas Transversales}.	
			\UCpaso Muestra la pantalla \IUref{IUModificarTemaTransversal}{Modificar Tema Transversal} con los datos del Tema Transversal seleccionado.
                        \UCpaso [\UCactor] Ingresa los datos del Tema Transversal.\label{paso:CUD2.2ingresaDatosTemaTransversal}\Trayref{A}
			\UCpaso [\UCactor] Oprime el botón \IUbutton{Aceptar}.
			\UCpaso Verifica que se cumpla las regla de negocio \BRref{2}. \Trayref{B} 
			\UCpaso Revisa los datos de acuerdo al diccionario de datos \ref{dd:TemaTransversal}. \Trayref{C}
			\UCpaso Actualiza los datos del Tema Transversal.
			\UCpaso Muestra el mensaje (MSG-4) indicando que se ha modificado correctamente el registro.\ref{MSG4}
			\UCpaso Continúa en el paso \ref{paso:CUD2buscarTemasTransversales} del \UCref{CUD2}.
	\end{UCtrayectoria}

		\begin{UCtrayectoriaA}{A}{Cancelar operación}{El usuario abandona el Caso de Uso.}
			\UCpaso[\UCactor] Decide ya no modificar los datos del Tema Transversal.
			\UCpaso[\UCactor] Oprime el botón \IUbutton{Cancelar}.
			\UCpaso Continúa en el paso \ref{paso:CUG2buscarTemasTransversales} del \UCref{CUD2}.
		\end{UCtrayectoriaA}

	\begin{UCtrayectoriaA}{B}{Datos del Tema Transversal Incompletos}{Existen campos nulos en el formulario de registro.}
			\UCpaso Muestra un mensaje (MSG-1) indicando que los datos ingresados estan incompletos.\ref{MSG1}
			\UCpaso Continúa en el paso \ref{paso:CUD2.1ValidacionDatos_TemaTransversal} del \UCref{CUD2.1}.
	\end{UCtrayectoriaA}

	\begin{UCtrayectoriaA}{C}{Datos del Tema Transversal Incorrectos}{Los datos del Tema Transversal no cumplen con lo especificado en el diccionario de datos.}
			\UCpaso Muestra un mensaje (MSG-2) indicando que los datos ingresados son incorrectos.\ref{MSG2}
			\UCpaso Continúa en el paso \ref{paso:CUD2.1ValidacionDatos_TemaTransversal} del \UCref{CUD2.1}.
	\end{UCtrayectoriaA}

%--------- Eliminar Tipos Aviso
	\begin{UseCase}{CUD2.3}{Eliminar Tema Transversal}{El usuario elimina un Tema Transversal registrado.}
			\UCitem{Versión}{1.1}
			\UCitem{Actor(es)}{Secretario.}
			\UCitem{Propósito}{Eliminar un Tema Transversal.}
			\UCitem{Resumen}{El sistema despliega los Datos de los Temas Transversales registrados y permite seleccionar un Tema Transversal  para eliminarlo.}
			\UCitem{Entradas}{Identificador del Tema Transversal seleccionado.}
			\UCitem{Salidas}{Datos del Tema Transversal seleccionado \ref{dd:TemaTransversal}.}
			\UCitem{Precondiciones}{Que exista el registro del Tema Transversal seleccionado.}
			\UCitem{Postcondiciones}{El Tema Transversal se elimina de los registros.}
			\UCitem{Autor}{Reyes De los Santos Miguel Ángel.}
			\UCitem{Referencias}{SIDAM-BESP-P1-Especificación de Catálogos.}
			\UCitem{Tipo}{Secundario. Viene del \UCref{CUD2}.}
			\UCitem{Módulo}{Secretaría.}
	\end{UseCase}

	\begin{UCtrayectoria}{Principal}
			\UCpaso[\UCactor] Oprime el botón \IUbutton{\includegraphics[scale=0.1]{images/icons/eliminar.png}} de la fila del registro que desea eliminar en la pantalla \IUref{IUGestTemasTransversales}{Gestionar Temas Transversales}.	
			\UCpaso Muestra la pantalla \IUref{IUEliminarTemaTransversal}{Eliminar Tema Transversal} con los datos del Tema Transversal seleccionado. 
			\UCpaso [\UCactor] Oprime el botón \IUbutton{Aceptar}. \Trayref{A}
			\UCpaso Verifica que no hay datos asociados. \Trayref{B}
			\UCpaso El Tema Transversal seleccionado se elimina.
			\UCpaso Muestra el mensaje (MSG-4) indicando que se ha eliminado correctamente el registro.\ref{MSG4}
			\UCpaso Continúa en el paso \ref{paso:CUD2buscarTemasTransversales} del \UCref{CUD2}.
	\end{UCtrayectoria}

		\begin{UCtrayectoriaA}{A}{Cancelar operación}{El usuario abandona el Caso de Uso.}
			\UCpaso[\UCactor] Decide ya no eliminar el Tema Transversal.
			\UCpaso[\UCactor] Oprime el botón \IUbutton{Cancelar}.
			\UCpaso Continúa en el paso \ref{paso:CUD2buscarTemasTransversales} del \UCref{CUD2}.
		\end{UCtrayectoriaA}
		
		\begin{UCtrayectoriaA}{B}{Datos Asociados}{El Tema Transversal que se desea eliminar tiene datos que están asociados.}
			\UCpaso Muestra un mensaje (MSG-6G2) indicando que no se puede eliminar porque tiene datos asociados.\ref{MSG-6G2}
			\UCpaso [\UCactor] Oprime el botón \IUbutton{Cancelar}.
			\UCpaso Continúa en el paso \ref{paso:CUD2buscarTemasTransversales} del \UCref{CUD2}.
		\end{UCtrayectoriaA}
%-------------------------------------- TERMINA descripción del caso de uso.
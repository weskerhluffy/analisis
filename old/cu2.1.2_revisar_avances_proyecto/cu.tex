% Descripción: Describe la funcionalidad ofrecida por el CU
% Propósito: Describe el objetivo o razón de ser del CU
% Resumen: Describe brevemente lo que hace el CU

	\begin{UseCase}{CU2.1.2}{Revisar Avances de un Proyecto}{Permite ver las acciones registradas de un proyecto, el avance de las acciones de un proyecto, los presupuestos solicitados, aprobados y ejercidos en el proyecto.}
		\UCitem{Versión}{3.0}
		%\UCitem{Estado}{Finalizado}
		\UCitem{Actor(es)}{Coordinador, Secretario, Gerente y/o Director General}
		\UCitem{Propósito}{Contar con un mecanismo que ayude a revisar los avances de un proyecto.}
		\UCitem{Resumen}{Se muestran los datos de las acciones de un proyecto, su avance y presupuestos.}
		\UCitem{Entradas}{Ninguna.}
		\UCitem{Salidas}{Datos del proyecto seleccionado.}
		\UCitem{Precondiciones}{\begin{itemize}
						\item Que exista al menos un proyecto registrado.
						\item Que se haya obtenido el perfil de Coordinador.
					\end{itemize}
  					}
		\UCitem{Postcondiciones}{Ninguna.}
		\UCitem{Autor}{Felipe}
		\UCitem{Referencias}{CU-P2-061011}
		\UCitem{Tipo}{Secundario. Viene de \UCref{CU2.1}}
		\UCitem{Módulo}{Coordinación.}
	\end{UseCase}
	
	% 1.- escriba solo una trayectoria principal
	% 2.- El actor es quien siempre inicia un CU
	% 3.- evite usar ``si ... entonces ...'' o ``mientras ...'' o ``para cada ...''
	% 4.- No olvide mencionar todas las verificacion%TODO cambiar la referencia a la pantalla Menu del coordinadores y cálculos realizados por el sistema
	% 5.- Evite mencionar palabras como: Tabla, BD, conexión, etc.
	
	\begin{UCtrayectoria}{Principal}

		\UCpaso[\UCactor] Selecciona la opción \IUbutton{Avances de Proyecto} de la pantalla \IUref{IURevisarAvancesProyecto}{Menú Coordinador}.\Trayref{A}

		\UCpaso Busca las Acciones registradas al Proyecto seleccionado.

		\UCpaso Genera las alertas con base en los avances y la fecha actual.\BRref{RN79}
		\UCpaso Muestra en la pantalla \IUref{IURevisarAvancesProyecto}{Revisar Avances de proyecto}. los datos de las acciones registradas en el proyecto, su correspondiente diagrama de gantt y los presupuestos  que se le han asignado.
		\UCpaso Muestra los botones \IUbutton{Visualizar avances }, \IUbutton{Indicadores}, \IUbutton{Editar} y \IUbutton{Cancelar} \Trayref{A} \UCExtensionPoint{CUC5}{Registrar Ejercicio de Presupuesto} \UCExtensionPoint{CUC6}{Reportar Avance del Proyecto} \UCExtensionPoint{CUC2.1.2.1}{Revisar Reportes de Avances de un Poyecto} \UCExtensionPoint{CUC8}{Cerrar Proyecto}

		\UCpaso Cuenta las restricciones atendidas, pendientes y turnadas de este proyecto.
		\UCpaso El sistema busca los presupuestos y los ejercicios.
		\UCpaso Calcula el resto disponible por presupuesto y el total del proyecto. \BRref{RN80} (Refrerenciar regla de negocio que diga cmo hacer esto.)
		\UCpaso Muestra en la pantalla \IUref{IURevisarAvancesProyecto}{Revisar Avances de proyecto} las acciones registradas, los avances, las alertas, los presupuestos, los ejercicios y los restos calculados.% en el proyecto, su correspondiente diagrama de gantt y los presupuestos  que se le han asignado.
		\Trayref{A} \UCExtensionPoint{CUC5}{Registrar Ejercicio de Presupuesto} \UCExtensionPoint{CUC6}{Reportar Avance del Proyecto} \UCExtensionPoint{CUC2.1.2.1}{Revisar Reportes de Avances de un Poyecto} \UCExtensionPoint{CUC8}{Cerrar Proyecto}

		%TODO:Poner puntos de extencion
		%\UCpaso Muestra los botones [INDICADORES] , [EDITAR], [CANCELAR]. \Trayref{B}  \UCExtensionPoint{CUC2.2.1}{Gestionar Acciones} \UCExtensionPoint{CUC2.2.1.4}{Gestionar Indicadores Físicos}

%\UCExtensionPoint{CUC2.2.1}{Gestionar Acciones} \UCExtensionPoint{CUC2.2.1.4}{Gestionar Indicadores Físicos}	

		%\UCpaso Muestra en la pantalla \IUref{IURevisarAvancesProyecto}{Revisar Avances del Proyecto} los datos del proyecto seleccionado con la opción \UIref{}{Indicadores} y \UIref{}{Editar} activadas.\Trayref{B} \UCExtensionPoint{CUC4}{Registrar Aprobación de Presupuesto.} \UCExtensionPoint{CUC5}{Registrar Ejercicio de Presupuesto} \UCExtensionPoint{CUC6}{Reportar Avance del Proyecto} \UCExtensionPoint{CUC2.1.2.1}{Revisar Reportes de Avances de un Poyecto} \UCExtensionPoint{CUC8}{Cerrar Proyecto}
		%%%%%%%%%%%%%%%%%%%%%%Las referencias de CU anteriores se colocaron en base a la imagen ``analisis/images/CUcoordinador.png
	\end{UCtrayectoria}
	
%		\begin{UCtrayectoriaA}{A}{No hay proyectos registrados.}{No se encuentran proyectos registrados.}
%			\UCpaso Se muestra el mensaje (MSG5) notificando al usuario que no se encontraron registros.\ref{MSG5}
%			\UCpaso[\UCactor] Oprime el botón \IUbutton{Cancelar}.
%			\UCpaso Regresa a la pantalla \IUref{IURevisarAvancesProyecto}{Menú Coordinador}
%		\end{UCtrayectoriaA}


		\begin{UCtrayectoriaA}{A}{Cancelar operación}{El Coordinador decide cancelar la operación}
			\UCpaso[\UCactor] Oprime el botón \IUbutton{Cancelar}.
			\UCpaso Regresa a la pantalla \IUref{IURevisarAvancesProyecto}{Menú Coordinador} 
		\end{UCtrayectoriaA}
%-------------------------------------- TERMINA descripción del caso de uso.

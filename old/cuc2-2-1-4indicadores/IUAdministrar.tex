\section{Pantalla: Gestión de Indicadores}

\subsubsection{Objetivo}
Mostrar la información correspondiente a los Indicadores asociados con la Acción seleccionada, con el fin de mantener organizada la información que permite observar la forma de Evaluar una Acción, ver Figura~\ref{IUEvaluarAccion}. Viene de la pantalla ``Editar Proyecto''.

\IUfig[0.9]{cu2-2-1-4indicador/Gestionar.png}{IUEvaluarAccion}{Evaluar Acción. Esta vista permite tener un control sobre los Indicadores correspondientes a la Acción.}

\subsubsection{Salidas}
En esta pantalla se muestra la lista de Indicadores registrados. Los registros esta ordenados por nombre en una tabla. Además se muestra una grafica que ilustra la distribucion de los Indicadores en función de su \textit{Peso}.

\subsubsection{Controles}

A la izquierda de cada Indicador aparece un {\em radio button} que permite seleccionarlo y realizar las operaciones de Modificación y Eliminación.

\subsubsection{Comandos}
\begin{itemize}
 \item \IUbutton{\includegraphics[scale=0.1]{images/icons/agregar.png}}:Esta opción permite registrar un nuevo Indicador, al oprimirlo se mostrará la pantalla\IUref{IUAgregarIndicador}{Agregar indicador}Si el Indicador se registra correctamente esta aparecerá en la tabla.

 \item \IUbutton{\includegraphics[scale=0.1]{images/icons/eliminar.png}}:Esta opción permite eliminar los datos de un Indicador, tras seleccionar un Indicador esta opción lo llevará a la pantalla \IUref{IUEliminarIndicador}{Eliminar Indicador}Si El Indicador se elimina correctamente, esta deberá desaparecer de la tabla.

 \item \IUbutton{\includegraphics[scale=0.1]{images/icons/Modificar.png}}:Esta opción permite actualizar los datos de un Indicador, tras seleccionar un Indicador esta opción lo llevará a la pantalla \IUref{IUModificarIndicador}{Modificar Indicador}Si los datos son actualizados correctamente, los cambios se verán en la tabla.

\end{itemize}
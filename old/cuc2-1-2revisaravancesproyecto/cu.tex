% Descripción: Describe la funcionalidad ofrecida por el CU
% Propósito: Describe el objetivo o razón de ser del CU
% Resumen: Describe brevemente lo que hace el CU
%  
	\begin{UseCase}{CUC2.1.2}{Revisar Avances de un proyecto}{Permite ver las acciones registradas de un proyecto, el avance de las acciones de un proyecto, los presupuestos solicitados, aprobados y ejercidos en el proyecto.}
		\UCitem{Versión}{3.0}
		\UCitem{Actor(es)}{Coordinador.}
		\UCitem{Propósito}{Contar con un mecanismo que ayude a revisar los avances de un proyecto}
		\UCitem{Resumen}{Se muestran los datos de las acciones de un proyecto, su avance y presupuestos.}
		\UCitem{Entradas}{Ninguna.}
		\UCitem{Salidas}{Datos de un proyecto seleccionado.}
		\UCitem{Precondiciones}{\begin{itemize}
		                        	\item Que exista al menos un proyecto registrado.
						\item Que se haya obtenido el perfil de Coordinador.
						\item Que se halla pasado por 
		                        \end{itemize}
					}
		\UCitem{Postcondiciones}{Ninguna.}
		\UCitem{Autor}{Pérez Vargas Roberto}
		\UCitem{Referencias}{CU-P2-061011}
		\UCitem{Tipo}{Secundario. Viene de }
		\UCitem{Módulo}{Coordinación.}
	\end{UseCase}
	
	% 1.- escriba solo una trayectoria principal
	% 2.- El actor es quien siempre inicia un CU
	% 3.- evite usar ``si ... entonces ...'' o ``mientras ...'' o ``para cada ...''
	% 4.- No olvide mencionar todas las verificaciones y cálculos realizados por el sistema
	% 5.- Evite mencionar palabras como: Tabla, BD, conexión, etc.
		
	\begin{UCtrayectoria}{Principal}
		\UCpaso[\UCactor] Selecciona la opción [Revisar avances de proyecto] del menú [Menu coordinador] \Trayref{A}
%Selecciona la opción \IUbutton{Acción} del registro deseado en el menú \IUref{IUacciones}{Menú de gestión de proyectos} 
		\UCpaso Busca las Acciones registradas en el Proyecto seleccionado.
		\UCpaso Muestra en la pantalla [Pantalla] los datos de las acciones registradas en el proyecto, su correspondiente diagrama de gantt y los presupuestos que se le han asignado.
		\UCpaso Muestra los botones [INDICADORES], [EDITAR] y [Cancelar]  \Trayref{B}
%\IUref{IUGestAcciones}{Gestionar Acciones} los datos de las acciones registradas en el proyecto y el diagrama de gantt.
		\UCpaso el sistema muestra las opciones de agregar, eliminar y modificar acciones, asi como reportar avances y definición de indicadores.\UCExtensionPoint{CUC2.2.1.1}{Agregar Acción}, \UCExtensionPoint{CUC2.2.1.2}{Modificar Acción}, \UCExtensionPoint{CUC2.2.1.3}{Eliminar Acción}, \UCExtensionPoint{CUC2.2.1.4}{Gestión de indicadores Fisicos}.
	\end{UCtrayectoria}
	
	\begin{UCtrayectoriaA}{A}{No hay proyectos registrados}{No se encuentran proyectos registrados}
			\UCpaso[\UCactor] Muestra el mensaje (MSG-5) indicando que no se encontrarón acciones registrados.\ref{MSG5}
			\UCpaso Continúa en el paso \ref{paso:CUC13buscarAcciones} del \UCref{CUG13}.
	\end{UCtrayectoriaA}

	\begin{UCtrayectoriaA}{B}{Cancelar operación}{El Coordinador decide cancelar la operación}
			\UCpaso[\UCactor] Muestra el mensaje (MSG-5) indicando que no se encontrarón acciones registrados.\ref{MSG5}
			\UCpaso Continúa en el paso \ref{paso:CUC13buscarAcciones} del \UCref{CUG13}.
	\end{UCtrayectoriaA}
%-------------------------------------- TERMINA descripción del caso de uso.
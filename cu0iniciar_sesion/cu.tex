  	\begin{UseCase}{CU0}{Iniciar Sesión}{Permite el acceso al sistema mediante un nombre de usuario registrado y una contraseña, además, proporciona la funcionalidad de Recuperar Contraseña.}
		\UCitem{Versión}{1.0}
		\UCitem{Actor(es)}{Administrador, Coordinador, Director, Gerente y Secretario.}
		\UCitem{Propósito}{Accesar al sistema de acuerdo a las necesidades de cada uno de los diferentes tipos de usuario.}
		\UCitem{Resumen}{Otorga acceso a los usuarios registrados de acuerdo al perfil de usuario de cada uno de ellos.}
		\UCitem{Entradas}{Nombre de login y Contraseña\ref{dd:Usuario}.}
		\UCitem{Salidas}{Menú de la Sesión iniciada.}
		\UCitem{Precondiciones}{Debe existir al menos un usuario registrado.}
		\UCitem{Postcondiciones}{Ninguna.}
		%\UCitem{Autor}{Jessica Nayeli Ramos Gonzalez}
		\UCitem{Referencias}{SIDAM-BESP-P0-Especificación de Catálogos.}
		\UCitem{Tipo}{Primario.}
		\UCitem{Módulo}{Control de acceso.}

	\end{UseCase}	

	\begin{UCtrayectoria}{Principal}
		\UCpaso [\UCactor] Ingresa al Sistema.
		\UCpaso Muestra la pantalla \IUref{IULoggueo}{Iniciar Sesión}. 
		\UCpaso [\UCactor] Ingresa el login y contraseña.\label{paso:CU0ingresarDatos} \UCExtensionPoint{CU0.1}{Recuperar Contraseña.}
		\UCpaso [\UCactor] Oprime el botón \IUbutton{Aceptar}.
		\UCpaso Valida los datos ingresados.\Trayref{A}
		\UCpaso Obtiene el perfil del usuario.
		\UCpaso Muestra el menú correspondiente al perfil obtenido.
	\end{UCtrayectoria}

	\begin{UCtrayectoriaA}{A}{Datos Incorrectos}{Los datos ingresados no son correctos.}
			\UCpaso Muestra el mensaje (MSG2-2) indicando que los datos ingresados no son correctos.\ref{MSG2-2}
			\UCpaso Continúa con el paso \ref{paso:CU0ingresarDatos}.
		\end{UCtrayectoriaA}


	\begin{UseCase}{CU0.1}{Recuperar Contraseña}{Se lleva a cabo cuando el actor hace una petición al sistema para recuperar su contraseña de acceso a éste.El proceso consiste en recibir un correo electrónico de parte del sistema a la dirección de correo principal otorgando al actor su nombre de usuario registrado y su contraseña.}
			\UCitem{Versión}{1.0}
			\UCitem{Actor(es)}{Administrador, Coordinador, Director, Gerente y Secretario.}
			\UCitem{Propósito}{Brindar al usuario su Contraseña.}
			\UCitem{Resumen}{Proporcionar al usuario su contraseña mediante una petición al sistema de ésta.}
			\UCitem{Entradas}{Login de Usuario.}
			\UCitem{Salidas}{Mensaje de confirmación.}
			\UCitem{Precondiciones}{Que el usuario tenga acceso al sistema.}
			\UCitem{Postcondiciones}{Correo electrónico enviado.}
			%\UCitem{Autor}{Jessica Nayeli Ramos Gonzalez.}
			\UCitem{Referencias}{SIDAM-BESP-P0-Especificación de Catálogos.}
			\UCitem{Tipo}{Secundario. Viene del caso de uso \UCref{CU0}.}
			\UCitem{Módulo}{Control de acceso.}
		\end{UseCase}

	\begin{UCtrayectoria}{Principal}
			\UCpaso[\UCactor] Oprime el vínculo \underline{¿Ha olvidado su contraseña?} en la página \IUref{IULoggueo}{Iniciar Sesión}.
			\UCpaso Muestra la pantalla \IUref{IURecuperarPassword}{Recuperar Contraseña}	
			\UCpaso [\UCactor] Introduce el usuario.\Trayref{A}\label{paso:CU0.1recpass}
			\UCpaso [\UCactor] Oprime el botón \IUbutton{Aceptar}.
			\UCpaso Busca los datos asociados al usuario ingresado.\Trayref{B}
			\UCpaso Envía un correo electrónico al e-mail asociado al usuario ingresado.\Trayref{C}
			\UCpaso Muestra un mensaje indicando que se ha enviado un mensaje a su correo electrónico.
	\end{UCtrayectoria}

		\begin{UCtrayectoriaA}{A}{Cancelar operación}{El actor decide ya no recuperar su contraseña}
			\UCpaso[\UCactor] Oprime el botón \IUbutton{· Cancelar}.
			\UCpaso Continúa en el paso \ref{paso:CU0ingresarDatos} del \UCref{CU0}.
		\end{UCtrayectoriaA}

		\begin{UCtrayectoriaA}{B}{Datos Incorrectos}{El usuario ingresado no existe.}
			\UCpaso Muestra el mensaje (MSG2-3) indicando que el usuario ingresado no existe.\ref{MSG2-3}
			\UCpaso Continúa en el paso \ref{paso:CU0.1recpass}.
		\end{UCtrayectoriaA}

		\begin{UCtrayectoriaA}{C}{E-mail no valido}{La cuenta de correo electrónico asociado al usuario ingresado no es correcto.}
			\UCpaso Muestra un mensaje indicando que ha ocurrido un error con la cuenta de correo electrónico.
			\UCpaso Continúa en el paso \ref{paso:CU0.1recpass}.
		\end{UCtrayectoriaA}
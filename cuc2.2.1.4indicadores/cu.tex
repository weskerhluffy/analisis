
	\begin{UseCase}{CUC2.2.1.4}{Gestionar Indicadores Físicos}{Administra un catálogo en el que se puede realizar Altas, Bajas y Cambios de Indicadores Físicos pertenecientes a una \textit{Acción}.}
		\UCitem{Versión}{2.0}
		\UCitem{Actor(es)}{Coordinador.}
		\UCitem{Propósito}{Visualizar los Indicadores registrados y su distribución por \textit{Acción} de acuerdo al peso asignado.}
		\UCitem{Resumen}{Se muestran los Indicadores registrados con la posibilidad de agregar, modificar y eliminar Indicadores.}
		\UCitem{Entradas}{Identificador de la Acción.}
		\UCitem{Salidas}{Despliegue de los Indicadores registrados.}
		\UCitem{Precondiciones}{
			\begin{itemize}
			 	\item Haber pasado por el \UCref{CU2.1}.
			 	\item Que exista una acción registrada para el proyecto.
			\end{itemize}

		}
		\UCitem{Postcondiciones}{Ninguna.}
		%\UCitem{Autor}{Pérez Domínguez Alberto Michel}
		\UCitem{Referencias}{SIDAM-BESP-P2}
		\UCitem{Tipo}{Secundario, viene de \UCref{CUC2.1}.}
		\UCitem{Módulo}{Coordinación}
	\end{UseCase}
	
	% 1.- escriba solo una trayectoria principal
	% 2.- El actor es quien siempre inicia un CU
	% 3.- evite usar ``si ... entonces ...'' o ``mientras ...'' o ``para cada ...''
	% 4.- No olvide mencionar todas las verificaciones y cálculos realizados por el sistema
	% 5.- Evite mencionar palabras como: Tabla, BD, conexión, etc.
	
	\begin{UCtrayectoria}{Principal}
		\UCpaso[\UCactor] Selecciona una Acción de la pantalla \IUref{IUPlaneacion}{Planeación de proyecto}.
		\UCpaso[\UCactor] Oprime el botón \IUbutton{\includegraphics[scale=0.1]{images/icons/indicadores.png}}.
		\UCpaso Busca los Indicadores registrados de la Acción seleccionada.\Trayref{A}\label{paso:CUC14BuscarIndicadores}
		\UCpaso Muestra la pantalla \IUref{IUEvaluarAccion}{Gestión de Indicadores Físicos} donde se despliegan los de datos de los Indicadores registrados y las opciones de Agregar, Modificar y Eliminar un Indicador.\UCExtensionPoint{CUC2.2.1.4.1}{Agregar Indicador} \UCExtensionPoint{CUC2.2.1.4.2}{Modificar Indicador} \UCExtensionPoint{CUC2.2.1.4.3}{Eliminar Indicador}
	\end{UCtrayectoria}

	\begin{UCtrayectoriaA}{A}{Resultado de busqueda vacía}{No se encuentrarón resultados.}
		\UCpaso Muestra mensaje (MSG-5) indicando que no se encontraron Indicadores para esta Acción.\ref{MSG5}
	\end{UCtrayectoriaA}

%--------- Agregar Indicadores
	\begin{UseCase}{CUC2.2.1.4.1}{Agregar Indicador Físico}{El Coordinador registra un nuevo Indicador para una determinada Acción.}
			\UCitem{Versión}{2.0}
			\UCitem{Actor(es)}{Coordinador.}
			\UCitem{Propósito}{Agregar un nuevo Indicador que funcionará como parametro para poder evaluar una Acción de un Proyecto.}
			\UCitem{Resumen}{Se agrega un nuevo Indicador a la Acción.}
			\UCitem{Entradas}{Datos del Indicador \ref{dd:Indicador}.}
			\UCitem{Salidas}{El mensaje (MSG-4) de operación exitosa.\ref{MSG4}}
			\UCitem{Precondiciones}{Que la suma del peso de los indicadores registrados para una acción no sea mayor que 100.}
			\UCitem{Postcondiciones}{Nuevo registro del Indicador para evaluar una Acción.}
			%\UCitem{Autor}{Pérez Domínguez Alberto Michel}
			\UCitem{Referencias}{SIDAM-BESP-P2}
			\UCitem{Tipo}{Secundario. Viene del \UCref{CUC2.2.1.4}}
			\UCitem{Módulo}{Coordinación}
	\end{UseCase}

	\begin{UCtrayectoria}{Principal}
			\UCpaso[\UCactor] Oprime el botón \IUbutton{Nuevo indicador} de la pantalla \IUref{IUEvaluarAccion}{Gestión de Indicadores Físicos}.
			\UCpaso Busca los Tipos de Unidades registradas.
			\UCpaso Muestra la pantalla \IUref{IUAgregarIndicador}{Registrar Indicador Físico}.
			\UCpaso [\UCactor] Ingresa los datos del Indicador.\label{paso:CUC1.4.1ingresaDatosIndicador}\Trayref{A}
			\UCpaso [\UCactor] Selecciona el Tipo de Unidad necesitada para el indicador.\Trayref{B}
			\UCpaso Busca las Unidades registradas para el Tipo de Unidad seleccionada.\label{paso:CUC141buscaUnidades}\Trayref{C}
			\UCpaso [\UCactor] Selecciona la Unidad.\Trayref{C}
			\UCpaso [\UCactor] Oprime el botón \IUbutton{Aceptar}.\label{paso:CUCagregarIndicador}
			\UCpaso Verifica que se cumpla la regla de negocio \BRref{RN2}. \Trayref{D}
			\UCpaso Verifica que los datos ingresados correspondan  con la definición del diccionario de datos \ref{dd:Indicador}. \Trayref{E}
			\UCpaso Verifica que se cumpla la regla de negocio \BRref{RN53}.\Trayref{F}
			\UCpaso Verifica que se cumpla la regla de negocio \BRref{RN61}.\Trayref{G}
			\UCpaso Verifica que se cumpla la regla de negocio \BRref{RN54}.\Trayref{H}
			\UCpaso Agrega el Indicador.
			\UCpaso Muestra el mensaje (MSG-4) de operación exitosa.\ref{MSG4}
			\UCpaso Continúa en el paso \ref{paso:CUC14BuscarIndicadores} del \UCref{CUC2.2.1.4}.
	\end{UCtrayectoria}

	\begin{UCtrayectoriaA}{A}{Cancelar operación}{El Coordinador abandona el Caso de Uso.}
			\UCpaso[\UCactor] Decide ya no agregar un nuevo Indicador.
			\UCpaso[\UCactor] Oprime el botón \IUbutton{Cancelar}.
			\UCpaso Continúa en el paso \ref{paso:CUC14BuscarIndicadores} del \UCref{CUC2.2.1.4}.
	\end{UCtrayectoriaA}

	\begin{UCtrayectoriaA}{B}{Tipo de Unidad Ponderada.}{El Coordinador elige como Tipo de Unidad ``Ponderada''.}
			\UCpaso Establece el campo Meta con un valor de 100.
			\UCpaso Establece como Unidad ``Porcentaje''.
			\UCpaso Continúa en el paso \ref{paso:CUCagregarIndicador} del \UCref{CUC2.2.1.4.1}.
	\end{UCtrayectoriaA}


	\begin{UCtrayectoriaA}{C}{No existe la Unidad necesitada}{No existe la Unidad necesitada para el Indicador.}
			\UCpaso [\UCactor] Solicita al Administrador Agregar la Unidad necesitada para el Indicador.
			\UCpaso Continúa en el paso \ref{paso:CUC141buscaUnidades} del \UCref{CUC2.2.1.4.1}.
	\end{UCtrayectoriaA}

	\begin{UCtrayectoriaA}{D}{Datos del Indicador Incompletos.}{Algunos campos del formulario de registro no han sido capturados.}
			\UCpaso Muestra el mensaje (MSG-1) indicando al administrador verifique que todos los campos hayan sido capturados.\ref{MSG1}
			\UCpaso Continúa en el paso \ref{paso:CUC1.4.1ingresaDatosIndicador} del \UCref{CUC2.2.1.4.1}.
	\end{UCtrayectoriaA}

	\begin{UCtrayectoriaA}{E}{Datos del Indicador son Incorrectos}{Los datos del Indicador no cumplen con lo especificado en el diccionario de datos.}
			\UCpaso Muestra el mensaje (MSG-2) indicando que los datos ingresados son incorrectos.\ref{MSG2}
			\UCpaso Continúa en el paso \ref{paso:CUC1.4.1ingresaDatosIndicador} del \UCref{CUC2.2.1.4.1}.
	\end{UCtrayectoriaA}

	\begin{UCtrayectoriaA}{F}{La Descripción del Indicador tiene un valor repetido.}{El valor del campo \textit{Descripción del Indicador} ya se encuntra registrado en la Acción.}
		\UCpaso Muestra el mensaje (MSG-3) indicando que la Descripción del Indicador ya existe.\ref{MSG3}
			\UCpaso Continúa en el paso \ref{paso:CUC1.4.1ingresaDatosIndicador} del \UCref{CUC2.2.1.4.1}.
	\end{UCtrayectoriaA}

	\begin{UCtrayectoriaA}{G}{El Peso del Indicador es inválido.}{El valor del campo \textit{Peso} no cumple con la regla de negocios \BRref{RN61}}
			\UCpaso Muestra el mensaje (MSG-RN-61) indicando que debe aumentar el peso.\ref{MSG-RN61}
			\UCpaso Continúa en el paso \ref{paso:CUC1.4.2ingresaDatosIndicador} del \UCref{CUC2.2.1.4.1}.
	\end{UCtrayectoriaA}

	\begin{UCtrayectoriaA}{H}{El Peso del Indicador supera el valor de 100.}{El valor del campo \textit{Peso} en sumatoria con el Peso de los Indicadores registrados anteriormente ha superado el limite de 100 para la \textbf{Acción}.}
			\UCpaso Muestra el mensaje (MSG-RN-54) indicando se verifique el Peso del Indicador.\ref{MSG-RN54}
			\UCpaso Continúa en el paso \ref{paso:CUC1.4.1ingresaDatosIndicador} del \UCref{CUC2.2.1.4.1}.
	\end{UCtrayectoriaA}

%--------- Modificar Indicadores
	\begin{UseCase}{CUC2.2.1.4.2}{Modificar Indicador Físico}{El Coordinador modifica el registro de un Indicador perteneciente a alguna Acción.}
			\UCitem{Versión}{2.0}
			\UCitem{Actor(es)}{Coordinador.}
			\UCitem{Propósito}{Modificar el Indicador seleccionado.}
			\UCitem{Resumen}{Se actualizan los datos de un Indicador.}
			\UCitem{Entradas}{Datos nuevos del Indicador \ref{dd:Indicador}.}
			\UCitem{Salidas}{Datos actuales del Indicador seleccionado\ref{dd:Indicador}. El mensaje (MSG-4) de operación exitosa.\ref{MSG4}}
			\UCitem{Precondiciones}{Que exista almenos un indicador para la acción seleccionada.}
			\UCitem{Postcondiciones}{Registro actualizado del Indicador para evaluar una Acción.}
			%\UCitem{Autor}{Pérez Domínguez Alberto Michel}
			\UCitem{Referencias}{SIDAM-BESP-P2}
			\UCitem{Tipo}{Secundario. Viene del \UCref{CUC2.2.1.4}}
			\UCitem{Módulo}{Coordinación}
	\end{UseCase}
% 
	\begin{UCtrayectoria}{Principal}
			\UCpaso[\UCactor] Selecciona un Indicador de la pantalla  \IUref{IUEvaluarAccion}{Gestión de Indicadores Físicos}.
			\UCpaso[\UCactor] Oprime el botón \IUbutton{\includegraphics[scale=0.1]{images/icons/Modificar.png}}.
			\UCpaso Muestra la pantalla \IUref{IUModificarIndicador}{Modificar Indicador Físico} con los datos actuales del Indicador seleccionado.
			\UCpaso [\UCactor] Ingresa los nuevos datos del Indicador.\label{paso:CUC1.4.2ingresaDatosIndicador}\Trayref{A}
			\UCpaso [\UCactor] Oprime el botón \IUbutton{Aceptar}.
			\UCpaso Verifica que se cumpla la regla de negocio \BRref{RN2}. \Trayref{B}
			\UCpaso Verifica que los datos ingresados correspondan  con la definición del diccionario de datos \ref{dd:Indicador}. \Trayref{C}
			\UCpaso Verifica que se cumpla la regla de negocio \BRref{RN53}.\Trayref{D}
			\UCpaso Verifica que se cumpla la regla de negocio \BRref{RN61}.\Trayref{E}
			\UCpaso Verifica que se cumpla la regla de negocio \BRref{RN55}.\Trayref{F}
			\UCpaso Verifica que se cumpla la regla de negocio \BRref{RN56}.\Trayref{G}
			\UCpaso Actualiza el registro del Indicador.
			\UCpaso Muestra el mensaje (MSG-4) de operación exitosa.\ref{MSG4}
			\UCpaso Continúa en el paso \ref{paso:CUC14BuscarIndicadores} del \UCref{CUC2.2.1.4.4}.
	\end{UCtrayectoria}
% 
	\begin{UCtrayectoriaA}{A}{Cancelar operación}{El usuario abandona el Caso de Uso.}
			\UCpaso[\UCactor] Decide ya no agregar un nuevo Indicador.
			\UCpaso[\UCactor] Oprime el botón \IUbutton{Cancelar}.
			\UCpaso Continúa en el paso \ref{paso:CUC14BuscarIndicadores} del \UCref{CUC2.2.1.4}.
	\end{UCtrayectoriaA}
% 
	\begin{UCtrayectoriaA}{B}{Datos del Indicador Incompletos.}{Algunos campos del formulario de registro no han sido capturados.}
			\UCpaso Muestra el mensaje (MSG-1) indicando al administrador verifique que todos los campos hayan sido capturados.\ref{MSG1}
			\UCpaso Continúa en el paso \ref{paso:CUC1.4.2ingresaDatosIndicador} del \UCref{CUC1.3.4.2}.
	\end{UCtrayectoriaA}
% 
	\begin{UCtrayectoriaA}{C}{Datos del Indicador son Incorrectos}{Los datos del Indicador no cumplen con lo especificado en el diccionario de datos.}
			\UCpaso Muestra el mensaje (MSG-2) indicando que los datos ingresados son incorrectos.\ref{MSG2}
			\UCpaso Continúa en el paso \ref{paso:CUC1.4.2ingresaDatosIndicador} del \UCref{CUC2.2.1.4.2}.
	\end{UCtrayectoriaA}
% 
	\begin{UCtrayectoriaA}{D}{La Descripción del Indicador tiene un valor repetido.}{El valor del campo \textit{Descripción del Indicador} ya se encuntra registrado en la Acción.}
			\UCpaso Muestra el mensaje (MSG-3) indicando que la Descripción del Indicador ya existe.\ref{MSG3}
			\UCpaso Continúa en el paso \ref{paso:CUC1.4.2ingresaDatosIndicador} del \UCref{CUC1.3.4.2}.
	\end{UCtrayectoriaA}
	\begin{UCtrayectoriaA}{E}{El Peso del Indicador es inválido.}{El valor del campo \textit{Peso} no cumple con la regla de negocios \BRref{RN61}}
			\UCpaso Muestra el mensaje (MSG-RN-61) indicando que debe aumentar el peso.\ref{MSG-RN61}
			\UCpaso Continúa en el paso \ref{paso:CUC1.4.2ingresaDatosIndicador} del \UCref{CUC2.2.1.4.2}.
	\end{UCtrayectoriaA}
	\begin{UCtrayectoriaA}{F}{El Peso del Indicador supera el valor de 100.}{El valor del campo \textit{Peso} en sumatoria con el Peso de los Indicadores registrados anteriormente ha superado el limite de 100 para la \textbf{Acción}.}
			\UCpaso Muestra el mensaje (MSG-RN-54) indicando se verifique el Peso del Indicador.\ref{MSG-RN54}
			\UCpaso Continúa en el paso \ref{paso:CUC1.4.2ingresaDatosIndicador} del \UCref{CUC2.2.1.4.2}.
	\end{UCtrayectoriaA}

	\begin{UCtrayectoriaA}{G}{El valor de la Meta del Indicador no puede ser modificado.}{El valor del campo \textit{Meta} es menor al ya reportado en los avances del Proyecto.}
			\UCpaso Muestra el mensaje (MSG-RN-55) indicando se verifique el Peso del Indicador.\ref{MSG-RN55}
			\UCpaso Continúa en el paso \ref{paso:CUC1.4.2ingresaDatosIndicador} del \UCref{CUC2.2.1.4.2}.
	\end{UCtrayectoriaA}
%--------- Eliminar Indicadores
	\begin{UseCase}{CUC2.2.1.4.3}{Eliminar Indicador Físico}{El Coordinador selecciona un Indicador registrado perteneciente a alguna acción para eliminarlo del sistema.}
			\UCitem{Versión}{2.0}
			\UCitem{Actor(es)}{Coordinador.}
			\UCitem{Propósito}{Eliminar un Indicador.}
			\UCitem{Resumen}{El sistema muestra los Indicadores registrados por Acción, permitiendo al administrador seleccionar un Indicador para eliminar su registro del sistema.}
			\UCitem{Entradas}{Identificador del Indicador seleccionado.}
			\UCitem{Salidas}{Datos del Indicador seleccionado \ref{dd:Indicador}. Mensaje de operación exitosa (MSG-4).\ref{MSG4}}
			\UCitem{Precondiciones}{Que exista almenos un indicador.}
			\UCitem{Postcondiciones}{El Indicador se elimina de los registros.}
			%\UCitem{Autor}{Alberto Michel Pérez Domínguez.}
			\UCitem{Referencias}{SIDAM-BESP-P2}
			\UCitem{Tipo}{Secundario. Viene del \UCref{CUC2.2.1.4.4}}
			\UCitem{Módulo}{Coordinación}
	\end{UseCase}

	\begin{UCtrayectoria}{Principal}
			\UCpaso [\UCactor] Selecciona el Indicador a eliminar de la pantalla  \IUref{IUEvaluarAccion}{Gestión de Indicadores Físicos}.
			\UCpaso Verifica que se cumpla la regla de negocio \BRref{RN56}.\Trayref{A}
			\UCpaso[\UCactor] Oprime el botón \IUbutton{\includegraphics[scale=0.1]{images/icons/eliminar.png}}.
			\UCpaso Muestra la pantalla \IUref{IUEliminarIndicador}{Eliminar Indicador} con los datos del Indicador seleccionado.
			\UCpaso [\UCactor] Oprime el botón \IUbutton{Aceptar}. \Trayref{B}
			\UCpaso Elimina el registro del Indicador.
			\UCpaso Muestra el mensaje (MSG-4) de operación exitosa.\ref{MSG4}
			\UCpaso Continúa en el paso \ref{paso:CUC14BuscarIndicadores} del \UCref{CUC2.2.1.4}.
	\end{UCtrayectoria}
		\begin{UCtrayectoriaA}{A}{Proyecto con Avances}{Se han reportado avances del Proyecto.}
			\UCpaso Muestra el mensaje (MSG-6I3) indicando al Coordinador que ya se han reportado avances del Proyecto y que el indicador no se puede eliminar.\ref{MSG-6I3}
			\UCpaso Continúa en el paso \ref{paso:CUC14BuscarIndicadores} del \UCref{CUC2.2.1.4}.
		\end{UCtrayectoriaA}

		\begin{UCtrayectoriaA}{B}{Cancelar operación}{El usuario abandona el Caso de Uso.}
			\UCpaso[\UCactor] Decide ya no eliminar el Indicador.
			\UCpaso[\UCactor] Oprime el botón \IUbutton{Cancelar}.
			\UCpaso Continúa en el paso \ref{paso:CUC14BuscarIndicadores} del \UCref{CUC2.2.1.4}.
		\end{UCtrayectoriaA}


%-------------------------------------- TERMINA descripción del caso de uso.
%-------------------------Registrar Proyecto
	\begin{UseCase}{CUC1}{Editar datos de Proyecto}{Se muestran los datos del Proyecto que pueden ser sujetos a edición, permite asociar Ejes Tematicos y Temas Transversales, además, alinearlo a un programa y/o con el programa sectorial.}
		\UCitem{Versión}{2.0}
		\UCitem{Actor(es)}{Coordinador}
		\UCitem{Propósito}{Someter los datos del registro de alguno de los proyectos existentes a edición y establecer su alineación.}
		\UCitem{Resumen}{Editar los datos de un Proyecto registrado asociado a un Coordinador.}
		\UCitem{Entradas}{
			\begin{itemize}
				\item Datos editables de Proyecto \ref{dd:DatosEditablesProyecto}.
				\item Seleccion de estructuras del programa sectorial.
				\item Seleccion de estructuras de un programa.
				\item Seleccion de ejes tematicos.
				\item Seleccion de temas transversales.
			\end{itemize}
		}
		\UCitem{Salidas}{Mensaje \ref{MSGCUC1}.}
		\UCitem{Precondiciones}{Se presentan las siguientes precondiciones
		  \begin{itemize}
		    \item Que existan programas registrados.
		    \item Que existan estructuras definidas para los programas.
		    \item Que existan Ejes Temáticos registrados.
		    \item Que existan Temas Transversales registrados.
		    \item Que exista al menos un Proyecto registrado.
		  \end{itemize}}
		\UCitem{Postcondiciones}{Ninguna.}
		%\UCitem{Autor}{Adrian Martinez. Ernesto Alvarado Gaspar}
		\UCitem{Referencias}{Ninguna}
		\UCitem{Tipo}{Secundario. Viene del caso de uso \UCref{CU2.1.1}}
		\UCitem{Módulo}{Coordinación}	
	\end{UseCase}

	\begin{UCtrayectoria}{Principal}
			\UCpaso[\UCactor] Oprime el botón \IUbutton{Asociar proyecto} de la pantalla \IUref{IURevisarInformacionProyecto}{Revisar Informacion de Proyecto}.
			\UCpaso Obtiene los datos del usuario.\label{paso:CUC1_ObtieneDatosUsuario}
			\UCpaso Obtiene los datos del proyecto que se ha seleccionado.
			\UCpaso Obtiene los Ejes Temáticos disponibles.
			\UCpaso Obtiene los Temas Transversales disponibles.
			\UCpaso Obtiene los programas sectoriales de la Agenda Ambiental disponibles.
			\UCpaso Obtiene otros programas diponibles con los que se pueda alinear el proyecto.
			\UCpaso Muestra la pantalla \IUref{IUEditarDatosProyecto}{Editar Datos de Proyecto}.
			\UCpaso [\UCactor] Edita los datos del proyecto y los de su alineacion.\label{paso:CUC1_IngresaDatosProyecto} \Trayref{A}
			\UCpaso [\UCactor] Oprime el botón \IUbutton{Aceptar}.\label{paso:CUC1_OprimeAceptar}
			\UCpaso Revisa que los datos correspondan con la Definición de Datos. \Trayref{B} \Trayref{C}
			\UCpaso Verifica que se cumpla la regla de negocio \BRref{RN32}.\Trayref{D}
			\UCpaso Verifica que se cumpla la regla de negocio \BRref{RN33}.\Trayref{E}
			\UCpaso Verifica que se cumpla la regla de negocio \BRref{RN40}.\Trayref{F}
			\UCpaso Valida que el periodo cumpla con  \BRref{RN60} \Trayref{G}.
			\UCpaso Valida que el periodo cumpla con  \BRref{RN7} \Trayref{G}.
			\UCpaso Registra el Proyecto.
			\UCpaso Muestra el mensaje de operación exitosa. \ref{MSG4}
			\UCpaso Regresa a la pantalla \IUref{IURevisarInformacionProyecto}{Revisar Informacion de Proyecto}, mostrando las actualizaciónes hechas.
	\end{UCtrayectoria}

	\begin{UCtrayectoriaA}{A}{Cancelar operación}{El usuario abandona el Caso de Uso.}
			\UCpaso[\UCactor] Decide ya no editar los datos de un Proyecto.
			\UCpaso[\UCactor] Oprime el botón \IUbutton{Cancelar}.
			\UCpaso Regresa a la pantalla \IUref{IURevisarInformacionProyecto}{Revisar Informacion de Proyecto}.
	\end{UCtrayectoriaA}


	\begin{UCtrayectoriaA}{B}{Datos del Proyecto Incompletos}{Los datos de Proyecto proporcionados estan incompletos.}
			\UCpaso Muestra el mensaje (MSG-1) indicando que los datos ingresados estan incompletos. \ref{MSG1}
			\UCpaso Continúa en el paso \ref{paso:CUC1_IngresaDatosProyecto}.
	\end{UCtrayectoriaA}

	\begin{UCtrayectoriaA}{C}{Datos del proyecto Incorrectos}{Los datos del Proyecto no corresponden con lo especificado en el diccionario de datos.}
			\UCpaso Muestra el mensaje (MSG-2) indicando los valores que son incorrectos. \ref{MSG2}
			\UCpaso Continúa en el paso \ref{paso:CUC1_IngresaDatosProyecto}.
	\end{UCtrayectoriaA}

	\begin{UCtrayectoriaA}{D}{Periodo mal definido}{El periodo del Proyecto que se desea registrar no está bien definido.}
			\UCpaso Muestra el mensaje (MSG-RN-32).\ref{MSG_RN32}
			\UCpaso Continúa en el paso \ref{paso:CUC1_IngresaDatosProyecto}.
	\end{UCtrayectoriaA}


	\begin{UCtrayectoriaA}{E}{El nombre del Proyecto repetido}{El Nombre del Proyecto ya se encuentra registrado.}
			\UCpaso Muestra el mensaje (MSG-3) \ref{MSG3}
			\UCpaso Continúa en el paso \ref{paso:CUC1_IngresaDatosProyecto}.
	\end{UCtrayectoriaA}
	
	\begin{UCtrayectoriaA}{F}{Las siglas del Proyecto repetidas}{Las siglas del Proyecto ya se encuentran registradas.}
			\UCpaso Muestra el mensaje (MSG-3) \ref{MSG3}
			\UCpaso Continúa en el paso \ref{paso:CUC1_IngresaDatosProyecto}.
	\end{UCtrayectoriaA}

	\begin{UCtrayectoriaA}{G}{Periodo invalido}{El periodo que se pretende dar a la nueva estructura es invalido.}
		\UCpaso Muestra el mensaje (MSG-RN-7).\ref{MSG_RN7}.
		\UCpaso Continua en el paso \ref{paso:CUC1_IngresaDatosProyecto}.
	\end{UCtrayectoriaA}




%-------------------------------------- TERMINA descripción del caso de uso.



















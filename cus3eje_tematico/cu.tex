% Descripción: Describe la funcionalidad ofrecida por el CU
% Propósito: Describe el objetivo o razón de ser del CU
% Resumen: Describe brevemente lo que hace el CU

	\begin{UseCase}{CUA7}{Gestionar Ejes Temáticos}{En el menú para Administradores, existe un enlace para ver los ejes temáticos registrados y las opciones para agregar, eliminar y modificar los mismos.}
		\UCitem{Versión}{2.0}
		\UCitem{Actor(es)}{Administrador}
		\UCitem{Propósito}{Contar con un mecanismo que ayude a manejar los ejes temáticos para el control de los proyectos.}
		\UCitem{Resumen}{Se muestran los ejes temáticos registrados con la posibilidad de agregar, modificar y eliminar.}
		\UCitem{Entradas}{Ninguna.}
		\UCitem{Salidas}{Lista de las ejes temáticos registrados.}
		\UCitem{Precondiciones}{Ninguna.}
		\UCitem{Postcondiciones}{Ninguna.}
		%\UCitem{Autor}{Raul Alberto Soto Cruz.}
		\UCitem{Referencias}{SIDAM-BESP-P1-Especificación de Catálogos}
		\UCitem{Tipo}{Primario.}
		\UCitem{Módulo}{Administración.}
	\end{UseCase}
	
	% 1.- escriba solo una trayectoria principal
	% 2.- El actor es quien siempre inicia un CU
	% 3.- evite usar ``si ... entonces ...'' o ``mientras ...'' o ``para cada ...''
	% 4.- No olvide mencionar todas las verificaciones y cálculos realizados por el sistema
	% 5.- Evite mencionar palabras como: Tabla, BD, conexión, etc.

	\begin{UCtrayectoria}{Principal}
		\UCpaso[\UCactor] Selecciona el enlace \textit{Gestión de Ejes Temáticos} del menú \IUref{IUMenuAdministrador}{Menú para Administrador.}.
		\UCpaso Busca los ejes temáticos registrados \label{paso:CUA7buscarEjesTematicos}. \Trayref{A}
		\UCpaso Muestra en la pantalla \IUref{IUGestEjesTematicos}{Gestionar Ejes Temáticos} los datos de los ejes temáticos  registrados ordenados por nombre y las opciones para Agregar, Modificar o Eliminar un Eje Temático.\UCExtensionPoint{CUA7.1}{Agregar Eje Temático} \UCExtensionPoint{CUA7.2}{Modificar Eje Temático} \UCExtensionPoint{CUA7.3}{Eliminar Eje Temático}
	\end{UCtrayectoria}

	\begin{UCtrayectoriaA}{A}{Resultado de busqueda vacía}{No se encuentra ningún eje temático.}
			\UCpaso Muestra mensaje (MSG-5) indicando que no se encontró eje temático MSG5 \ref{MSG5}.
	\end{UCtrayectoriaA}

%--------- Agregar eje tematico
	\begin{UseCase}{CUA7.1}{Agregar Eje Temático}{El usuario registra un nuevo Eje Temático con sus respectivos datos.}
			\UCitem{Versión}{2.0}
			\UCitem{Actor(es)}{Secretaría.}
			\UCitem{Propósito}{Permitir agregar nuevos ejes temáticos.}
			\UCitem{Resumen}{Se ingresan los datos correspondientes y se agrega el nuevo eje temático.}
			\UCitem{Entradas}{Datos del Eje Temático \ref{dd:EjeTematico}.}
			\UCitem{Salidas}{Mensaje de operación exitosa.}
			\UCitem{Precondiciones}{Ninguna.}
			\UCitem{Postcondiciones}{Eje Temático registrado.}
			%\UCitem{Autor}{Hermosillo García Karen Adriana.Raul Alberto Soto Cruz}
			\UCitem{Referencias}{SIDAM-BESP-P1-Especificación de Catálogos}
			\UCitem{Tipo}{Secundario. Viene del \UCref{CUA7}}
			\UCitem{Módulo}{Administración.}
	\end{UseCase}
	\begin{UCtrayectoria}{Principal}
			\UCpaso[\UCactor] Oprime el botón \IUbutton{Nuevo Eje Temático} en la página \IUref{IUGestEjesTematicos}{Gestionar Ejes Temáticos}.
			\UCpaso Muestra la pantalla \IUref{IUAgregarEjeTematico}{Agregar Eje Temático} 
			\UCpaso [\UCactor] Ingresa los datos del Eje Temático. \Trayref{A} \label{paso:CUA7.1ingresarDatos}.
			\UCpaso [\UCactor] Oprime el botón \IUbutton{Aceptar}.
			\UCpaso Revisa los datos de acuerdo a la regla de negocios \BRref{RN2}. \Trayref{B}
			\UCpaso Revisa los datos de acuerdo al diccionario \ref{dd:EjeTematico} \Trayref{C}
			\UCpaso Verifica que se cumpla la regla de negocios \BRref{RN1}. \Trayref{D} 
			\UCpaso Agrega el Eje Temático.
			\UCpaso Muestra mensaje (MSG-4) indicando que se agrego de forma correcta.\ref{MSG4}
			\UCpaso Continúa en el paso \ref{paso:CUA7buscarEjesTematicos} del \UCref{CUA7}.
	\end{UCtrayectoria}
	\newpage
	\begin{UCtrayectoriaA}{A}{Cancelar operación}{El usuario abandona el Caso de Uso.}
			\UCpaso[\UCactor] Decide no Agregar el Eje Temático.
			\UCpaso[\UCactor] Oprime el botón \IUbutton{Cancelar}.
			\UCpaso Continúa en el paso \ref{paso:CUA7buscarEjesTematicos} del \UCref{CUA7}.
	\end{UCtrayectoriaA}
		
	\begin{UCtrayectoriaA}{B}{Datos nulos}{Los datos ingresados por el usuario  no cumplen con la Regla de necocios \BRref{2}.}
			\UCpaso Muestra mensaje (MSG-1).\ref{MSG1}
			\UCpaso Continúa en el paso \ref{paso:CUA7.1ingresarDatos} del \UCref{CUA7.1}.
	\end{UCtrayectoriaA}

	\begin{UCtrayectoriaA}{C}{Datos incorrectos}{Los datos ingresados por el usuario  no corresponden a los datos del Eje Temático \ref{dd:EjeTematico}.}
			\UCpaso Muestra mensaje (MSG-2).\ref{MSG2}
			\UCpaso Continúa en el paso \ref{paso:CUA7.1ingresarDatos} del \UCref{CUA7.1}.
	\end{UCtrayectoriaA}

	\begin{UCtrayectoriaA}{D}{Nombre de eje temático repetido}{El nombre del eje temático ya existe.}
		\UCpaso Muestra mesnaje (MSG-3).\ref{MSG3}
		\UCpaso Continúa en el paso \ref{paso:CUA7.1ingresarDatos} del \UCref{CUA7.1}.
	\end{UCtrayectoriaA}

%--------- Modificar eje tematico
	\begin{UseCase}{CUA7.2}{Modificar Eje Temático}{El sistema despliega los Datos de los Ejes Temáticos registrados y permite seleccionar uno de ellos para modificar sus datos, guardando los cambios.}
			\UCitem{Versión}{2.0}
			\UCitem{Actor}{Dirección.}
			\UCitem{Propósito}{Modificar los datos de un Eje Temático y actualizar el registro en el sistema.}
			\UCitem{Resumen}{El usuario modifica los datos de un Eje Temático registrado.}
			\UCitem{Entradas}{Identificador del Eje Temático seleccionado. Datos a modificar del Eje Temático \ref{dd:EjeTematico}.}
			\UCitem{Salidas}{Datos del eje temático seleccionado \ref{dd:EjeTematico} y mensaje de operación exitosa.}
			\UCitem{Precondiciones}{Que exista el registro del Eje Temático seleccionado.}
			\UCitem{Postcondiciones}{El eje temático seleccionado se actualiza.}
			%\UCitem{Autor}{Hermosillo García Karen Adriana.Raul Alberto Soto Cruz}
			\UCitem{Referencias}{SIDAM-BESP-P1-Especificación de Catálogos}
			\UCitem{Tipo}{Secundario. Viene del \UCref{CUA7}}
			\UCitem{Módulo}{Dirección}
			\end{UseCase}

	\begin{UCtrayectoria}{Principal}
			\UCpaso[\UCactor] Oprime el botón \IUbutton{\includegraphics[scale=0.1]{images/icons/editar.png}} del eje temático que desea modificar 
			\IUref{IUGestEjesTematicos}{Gestionar Ejes Temáticos}.
			\UCpaso Muestra la pantalla \IUref{IUModificarEjeTematico}{Editar Eje Temático} con los datos del eje temático seleccionado 
			\UCpaso [\UCactor] Ingresa los datos del eje temático.\label{paso:CUA7.2ingresarDatos}\Trayref{A}
			\UCpaso [\UCactor] Oprime el botón \IUbutton{Aceptar}.
			\UCpaso Revisa los datos de acuerdo a la regla de negocios \BRref{RN2}. \Trayref{B}
			\UCpaso	Revisa los datos de acuerdo al diccionario de datos \ref{dd:EjeTematico} \Trayref{C}
			\UCpaso Actualiza los datos del Eje Temático.
			\UCpaso Muestra el mensaje (MSG-4).\ref{MSG4}
			\UCpaso Continúa en el paso \ref{paso:CUA7buscarEjesTematicos} del \UCref{CUA7}.
	\end{UCtrayectoria}
	\newpage
		\begin{UCtrayectoriaA}{A}{Cancelar operación}{El usuario abandona el Caso de Uso.}
			\UCpaso[\UCactor] Decide ya no modificar los datos del eje temático.
			\UCpaso[\UCactor] Oprime el botón \IUbutton{Cancelar}.
			\UCpaso Continúa en el paso \ref{paso:CUA7buscarEjesTematicos} del \UCref{CUA7}.
		\end{UCtrayectoriaA}
		
		 \begin{UCtrayectoriaA}{B}{Datos nulos}{Los datos del eje temático no cumplen con la regla de negocios \BRref{2}..}
			\UCpaso Muestra el mensaje (MSG-1).\ref{MSG1}
			\UCpaso Continúa en el paso \ref{paso:CUA7.2ingresarDatos} del \UCref{CUA7.2}.
		\end{UCtrayectoriaA}

		 \begin{UCtrayectoriaA}{C}{Datos incorrectos}{Los datos del eje temático no cumplen con lo especificado en el diccionario de datos \ref{dd:EjeTematico}}
			\UCpaso Muestra el mensaje (MSG-2).\ref{MSG2}
			\UCpaso Continúa en el paso \ref{paso:CUA7.2ingresarDatos} del \UCref{CUA7.2}.
		\end{UCtrayectoriaA}

%--------- Eliminar tematico
	\begin{UseCase}{CUA7.3}{Eliminar Eje Temático}{El sistema despliega los Datos de los ejes temáticos registrados y permite seleccionar un eje temático para eliminarlo.}
			\UCitem{Versión}{1.0}
			\UCitem{Actor(es)}{Secretario.}
			\UCitem{Propósito}{Permitir al usuario eliminar un eje temático cuando este no sea necesario.}
			\UCitem{Resumen}{Alguno de los ejes temáticos existentes no es necesario y el administrador desea eliminarlo.}
			\UCitem{Entradas}{Identificador del Eje Temático seleccionado.}
			\UCitem{Salidas}{Datos del eje temático seleccionado \ref{dd:EjeTematico} y mensaje de operación exitosa.}
			\UCitem{Precondiciones}{Que exista el registro del Eje Temático seleccionado.}
			\UCitem{Postcondiciones}{El eje temático se elimina de los registros.}
			%\UCitem{Autor}{Hermosillo García Karen Adriana. Raul Alberto Soto Cruz}
			\UCitem{Referencias}{SIDAM-BESP-P1-Especificación de Catálogos}
			\UCitem{Tipo}{Secundario. Viene del \UCref{CUA7}}
			\UCitem{Módulo}{Dirección.}
	\end{UseCase}

	\begin{UCtrayectoria}{Principal}
			\UCpaso[\UCactor] Oprime el botón \IUbutton{\includegraphics[scale=0.1]{images/icons/eliminar.png}} del eje temático que desea eliminar.
			\IUref{IUGestEjesTematicos}{Gestión de Ejes Temáticos}.
			\UCpaso Muestra la pantalla \IUref{IUEliminarEjeTematico}{Eliminar Eje Temático} \Trayref{A} 
			\UCpaso [\UCactor] Oprime el botón \IUbutton{Aceptar}.
			\UCpaso Verifica que no tenga datos asociados. \Trayref{B} 
			\UCpaso El Eje Temático seleccionado se elimina.
			\UCpaso Muestra el mensaje (MSG-4).\ref{MSG4}
			\UCpaso Continúa en el paso \ref{paso:CUA7buscarEjesTematicos} del \UCref{CUA7}.
	\end{UCtrayectoria}

		\begin{UCtrayectoriaA}{A}{Cancelar operación}{El usuario abandona el Caso de Uso.}
			\UCpaso[\UCactor] El usuario ya no desea eliminar el eje temático seleccionado.
			\UCpaso[\UCactor] Oprime el botón \IUbutton{Cancelar}.
			\UCpaso Continúa en el paso \ref{paso:CUA7buscarEjesTematicos} del \UCref{CUA7}.
		\end{UCtrayectoriaA}

		 \begin{UCtrayectoriaA}{B}{Datos asociados}{Existen proyectos que cuentan con este eje temático y no se puede eliminar porque se encuentra asociado.}
			\UCpaso Muestra el mensaje (MSG-6G3).\ref{MSG-6G3}.
			\UCpaso Continúa en el paso \ref{paso:CUA7buscarEjesTematicos} del \UCref{CUA7}.
		\end{UCtrayectoriaA}
%-------------------------------------- TERMINA descripción del caso de uso.
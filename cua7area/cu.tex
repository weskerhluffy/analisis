% Descripción: Describe la funcionalidad ofrecida por el CU
% Propósito: Describe el objetivo o razón de ser del CU
% Resumen: Describe brevemente lo que hace el CU

	\begin{UseCase}{CU7}{Gestionar Áreas}{Ofrece un mecanismo para ver las áreas registradas y opciones de agregar, eliminar y modificar las mismas.}
		\UCitem{Versión}{1.0}
		\UCitem{Estado}{Finalizado}
		\UCitem{Actor(es)}{Administrador}
		\UCitem{Propósito}{Contar con un mecanismo que ayude a manejar las áreas para el control de los proyectos.}
		\UCitem{Resumen}{Se muestran las áreas registradas con la posibilidad de agregar, modificar y eliminar.}
		\UCitem{Entradas}{Ninguna.}
		\UCitem{Salidas}{Lista de las áreas registradas.}
		\UCitem{Precondiciones}{Ninguna.}
		\UCitem{Postcondiciones}{Ninguna.}
		\UCitem{Autor}{Hermosillo García Karen Adriana }
		\UCitem{Revisa}{Jaime Lopez.}
		\UCitem{Referencias}{SIDAM-BESP-P1-Especificación de Catálogos}
		\UCitem{Tipo}{Primario.}
		\UCitem{Módulo}{Administración.}
	\end{UseCase}
	
	% 1.- escriba solo una trayectoria principal
	% 2.- El actor es quien siempre inicia un CU
	% 3.- evite usar ``si ... entonces ...'' o ``mientras ...'' o ``para cada ...''
	% 4.- No olvide mencionar todas las verificaciones y cálculos realizados por el sistema
	% 5.- Evite mencionar palabras como: Tabla, BD, conexión, etc.
	
	
	
	
	\begin{UCtrayectoria}{Principal}
		\UCpaso[\UCactor] Selecciona la opción \IUbutton{Gestionar Áreas} de la pantalla \IUref{IUMenuAdministrador}{Menú del Administrador}.
		\UCpaso Busca las áreas registradas. \Trayref{A}\label{paso:CU7buscarAreas}
		\UCpaso Muestra en la pantalla \IUref{IUGestAreas}{Gestionar Áreas} los datos de las áreas  registradas ordenadas por nombre y las opciones para Agregar, Modificar o Eliminar un Área.\UCExtensionPoint{CU7.1}{Agregar Área} \UCExtensionPoint{CU7.2}{Modificar Área} \UCExtensionPoint{CU7.3}{Eliminar Área}
		\UCpaso Vuelve a la pantalla anterior. 
	\end{UCtrayectoria}

	\begin{UCtrayectoriaA}{A}{Resultado de busqueda vacía}{No se encuentra ningún área.}
		\UCpaso Muestra mensaje (MSG-5) indicando que no se encontraron áreas.\ref{MSG5}
	\end{UCtrayectoriaA}

%--------- Agregar área
	\begin{UseCase}{CU7.1}{Agregar Área}{El usuario registra una nueva Área con sus respectivos datos.}
			\UCitem{Versión}{1.0}
			\UCitem{Estado}{Finalizado}
			\UCitem{Actor(es)}{Administrador.}
			\UCitem{Propósito}{Permitir agregar nuevas áreas.}
			\UCitem{Resumen}{Se ingresan los datos correspondientes y se agrega la nueva área.}
			\UCitem{Entradas}{Datos del Área \ref{dd:Area}.}
			\UCitem{Salidas}{Mensaje que indica que el área ha sido agregado de forma correcta.}
			\UCitem{Precondiciones}{Ninguna.}
			\UCitem{Postcondiciones}{Área registrada. }
			\UCitem{Autor}{Hermosillo García Karen Adriana }
			\UCitem{Referencias}{SIDAM-BESP-P1-Especificación de Catálogos}
			\UCitem{Tipo}{Secundario. Viene del \UCref{CU7}}
			\UCitem{Módulo}{Administración.}
	\end{UseCase}

	\begin{UCtrayectoria}{Principal}
			\UCpaso[\UCactor] Oprime el botón \IUbutton{\includegraphics[scale=0.2]{images/icons/agregar.png}} en la página \IUref{IUGestAreas}{Gestionar Áreas}.
			\UCpaso Muestra la pantalla \IUref{IUAgregarArea}{Agregar Área} 
			\UCpaso [\UCactor] Ingresa los datos del Área. \Trayref{A} \label{paso:CU7.1ingresarDatos}.
			\UCpaso [\UCactor] Oprime el botón \IUbutton{Aceptar}.
			\UCpaso Revisa que los datos cumplan la regla de negocios \BRref{2}. \Trayref{B}
			\UCpaso Revisa los datos de acuerdo al diccionario \ref{dd:Area} \Trayref{C}
			\UCpaso Verifica que se cumpla la regla de negocios \BRref{1}. \Trayref{D} 
			\UCpaso Agrega el Área.
			\UCpaso Muestra el mensaje (MSG-4) de operación exitosa.\ref{MSG4}
			\UCpaso Continúa en el paso \ref{paso:CU7buscarAreas} del \UCref{CU7}.
	\end{UCtrayectoria}
	\newpage
	\begin{UCtrayectoriaA}{A}{Cancelar operación}{El usuario abandona el Caso de Uso.}
			\UCpaso[\UCactor] Decide no Agregar el Área.
			\UCpaso[\UCactor] Oprime el botón \IUbutton{Cancelar}.
			\UCpaso Continúa en el paso \ref{paso:CU7buscarAreas} del \UCref{CU7}.
	\end{UCtrayectoriaA}
		
	\begin{UCtrayectoriaA}{B}{Datos nulos}{Los datos ingresados por el usuario  no cumplen con la regla de negocios \BRref{2}.}
			\UCpaso Muestra el mensaje (MSG-1).\ref{MSG1}
			\UCpaso Continúa en el paso \ref{paso:CU7.1ingresarDatos} del \UCref{CU7.1}.
	\end{UCtrayectoriaA}
	\begin{UCtrayectoriaA}{C}{Datos incorrectos}{Los datos ingresados por el usuario  no cumplen con los datos del Área \ref{dd:Area}.}
			\UCpaso Muestra el mensaje (MSG-2).\ref{MSG2}
			\UCpaso Continúa en el paso \ref{paso:CU7.1ingresarDatos} del \UCref{CU7.1}.
	\end{UCtrayectoriaA}


	\begin{UCtrayectoriaA}{D}{Área repetida}{El nombre del área ya se encuentra registrada \BRref{1}.}
		\UCpaso Muestram el mensaje (MSG-3). \ref{MSG3}
		\UCpaso Continúa en el paso \ref{paso:CU7.1ingresarDatos} del \UCref{CU7.1}.
	\end{UCtrayectoriaA}

%--------- Modificar Área
	\begin{UseCase}{CU7.2}{Modificar Área}{El usuario modifica los datos de un Área registrada.}
			\UCitem{Versión}{1.0}
			\UCitem{Estado}{Finalizado}
			\UCitem{Actor}{Administrador.}
			\UCitem{Propósito}{Modificar los datos de un Área y actualizar el registro en el sistema.}
			\UCitem{Resumen}{El sistema despliega los Datos de las Áreas registradas y permite seleccionar un Área para modificar sus datos, guardando los cambios.}
			\UCitem{Entradas}{Identificador del Área seleccionado y datos a modificar \ref{dd:Area}.}
			\UCitem{Salidas}{Datos del Área seleccionada \ref{dd:Area} y mensaje de operación exitosa.}
			\UCitem{Precondiciones}{Que exista el registro del Área que se desea modificar.}
			\UCitem{Postcondiciones}{El área seleccionada se actualiza.}
			\UCitem{Autor}{Hermosillo García Karen Adriana }
			\UCitem{Referencias}{SIDAM-BESP-P1-Especificación de Catálogos}
			\UCitem{Tipo}{Secundario. Viene del \UCref{CU7}}
			\UCitem{Módulo}{Administración}
			\end{UseCase}

	\begin{UCtrayectoria}{Principal}
			\UCpaso[\UCactor] Oprime el botón \IUbutton{\includegraphics[scale=0.1]{images/icons/editar.png}} del area que desea modificar 	\UCpaso Muestra la pantalla \IUref{IUModificarArea}{Modificar Área} con los datos del área seleccionada 
			\UCpaso [\UCactor] Ingresa los datos del Área.\label{paso:CU7.2ingresarDatos}\Trayref{A}
			\UCpaso [\UCactor] Oprime el botón \IUbutton{Aceptar}.
			\UCpaso Revisa que se cumpla regla de negocios \BRref{2}. \Trayref{B}
			\UCpaso Revisa los datos de acuerdo al diccionario de datos \ref{dd:Area} \Trayref{C}
			\UCpaso Actualiza los datos del Área.
			\UCpaso Muestra el mensaje (MSG-4) de operación exitosa.\ref{MSG4}
			\UCpaso Continúa en el paso \ref{paso:CU7buscarAreas} del \UCref{CU7}.
	\end{UCtrayectoria}
	\newpage
		\begin{UCtrayectoriaA}{A}{Cancelar operación}{El usuario abandona el Caso de Uso.}
			\UCpaso[\UCactor] Decide ya no modificar los datos del Área.
			\UCpaso[\UCactor] Oprime el botón \IUbutton{Cancelar}.
			\UCpaso Continúa en el paso \ref{paso:CU7buscarAreas} del \UCref{CU7}.
		\end{UCtrayectoriaA}
		
		 \begin{UCtrayectoriaA}{B}{Datos nulos}{Existen campos nulos en el formulario de registro.}
			\UCpaso Muestra MSG1 \ref{MSG1}.
			\UCpaso Continúa en el paso \ref{paso:CU7.2ingresarDatos} del \UCref{CU7.2}.
		\end{UCtrayectoriaA}
		 \begin{UCtrayectoriaA}{C}{Datos incorrectos}{Los datos del área no cumplen con lo especificado en el diccionario de datos \ref{dd:Area}}
			\UCpaso Muestra MSG2 \ref{MSG2}.
			\UCpaso Continúa en el paso \ref{paso:CU7.2ingresarDatos} del \UCref{CU7.2}.
		\end{UCtrayectoriaA}



%--------- Eliminar Área
	\begin{UseCase}{CU7.3}{Eliminar Área}{Una de las áreas existentes no es necesaria y el usuario desea eliminarla.}
			\UCitem{Versión}{1.0}
			\UCitem{Estado}{Finalizado}
			\UCitem{Actor(es)}{Administrador.}
			\UCitem{Propósito}{Permitir al usuario eliminar un área cuando esta no sea necesaria.}
			\UCitem{Resumen}{El sistema despliega los Datos de las áreas registradas y permite seleccionar un área para eliminarla.}
			\UCitem{Entradas}{Identificador del Área seleccionada.}
			\UCitem{Salidas}{Datos del área seleccionada \ref{dd:Area} y mensaje de operación exitosa.}
			\UCitem{Precondiciones}{Que exista el registro del Área que se desea eliminar.}
			\UCitem{Postcondiciones}{El área se elimina de los registros.}
			\UCitem{Autor}{Hermosillo García Karen Adriana }
			\UCitem{Revisa}{Jaime López.}
			\UCitem{Referencias}{SIDAM-BESP-P1-Especificación de Catálogos}
			\UCitem{Tipo}{Secundario. Viene del \UCref{CU7}}
			\UCitem{Módulo}{Administración.}
	\end{UseCase}

	\begin{UCtrayectoria}{Principal}
			\UCpaso[\UCactor] Oprime el botón \IUbutton{\includegraphics[scale=0.1]{images/icons/eliminar.png}} del área que desea eliminar.
			\UCpaso Muestra la pantalla \IUref{IUEliminarArea}{Eliminar Área} con los datos del área seleccionada \Trayref{A} 
			\UCpaso [\UCactor] Oprime el botón \IUbutton{Aceptar}.
			\UCpaso Verifica que cumpla con la regla de negocio \BRref{RN50}. \Trayref{B} 
			\UCpaso El Área seleccionada se elimina.
			\UCpaso Muestra el mensaje (MSG-4) de operación exitosa.\ref{MSG4}
			\UCpaso Continúa en el paso \ref{paso:CU7buscarAreas} del \UCref{CU7}.
	\end{UCtrayectoria}

		\begin{UCtrayectoriaA}{A}{Cancelar operación}{El usuario abandona el Caso de Uso.}
			\UCpaso[\UCactor] El usuario ya no desea eliminar el área seleccionada.
			\UCpaso[\UCactor] Oprime el botón \IUbutton{Cancelar}.
			\UCpaso Continúa en el paso \ref{paso:CU7buscarAreas} del \UCref{CU7}.
		\end{UCtrayectoriaA}

		 \begin{UCtrayectoriaA}{B}{Datos asociados}{Existen usuarios que se encuentran registrados dentro de esta área, y no se puede eliminar porque está asociada.}
			\UCpaso Muestra el mensaje (MSG-6A7).\ref{MSG-6A7}.
			\UCpaso Continúa en el paso \ref{paso:CU7buscarAreas} del \UCref{CU7}.
		\end{UCtrayectoriaA}
%-------------------------------------- TERMINA descripción del caso de uso.
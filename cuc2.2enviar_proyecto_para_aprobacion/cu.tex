%------------------------------------- Enviar Proyecto para aprobación

\begin{UseCase}{CUC2.2}{Enviar Proyecto para aprobación}{Una vez que un proyecto ha sido registrado con éxito, el coordinador tiene la opción de enviarlo al Gerente para aprobación, esto es, si el Gerente le da visto bueno, el proyecto pasará a estado de ejecución, de lo contrario, si éste es rechazado seguirá en estado de edición hasta que cumpla las expectativas del Gerente.}
		\UCitem{Versión}{2.0}
		\UCitem{Actor(es)}{Coordinador.}
		\UCitem{Propósito}{Enviar un proyecto para ser evaluado.}
		\UCitem{Resumen}{Permite enviar un proyecto al gerente de parte del coordinador que registró el proyecto para que lo evalúe y pueda ser aprobado o rechazado.}
		\UCitem{Entradas}{Ninguna.}
		\UCitem{Salidas}{Mensaje de confirmación de modificación del envío del proyecto.}
		\UCitem{Precondiciones}{
			\begin{itemize}
			 	\item Haber pasado por el \UCref{CU2}.
			\end{itemize}
		}
		\UCitem{Postcondiciones}{El proyecto pasa al estado de ``Revisión''.}
		%\UCitem{Autor}{Julio César Ruiz Leal} 
		\UCitem{Referencias}{CU-P2-061011}
		\UCitem{Tipo}{Secundario. Viene de \UCref{CUC2}}
		\UCitem{Módulo}{Coordinación}
\end{UseCase}
	
	
\begin{UCtrayectoria}{Principal}
		\UCpaso[\UCactor] Oprime el botón \IUbutton{Enviar proyecto para aprobación} de la pantalla \IUref{IUPlaneacion}{Planeación de proyecto}.
		\UCpaso Se muestra la pantalla \IUref{IUEnviarProyectoAprobacion}{Enviar proyecto para Aprobación}
		\UCpaso[\UCactor] Oprime el botón \IUbutton{Aceptar}. \Trayref{A}
		\UCpaso Verifica que se cumpla la regla de negocio \BRref{RN59}. \Trayref{B}
		\UCpaso Cambia el proyecto del estado de ``Edición'' al estado de ``Revisión''.
\end{UCtrayectoria}

\begin{UCtrayectoriaA}{A}{Cancelar envío}{El usuario decide cancelar el envío del proyecto.}
			\UCpaso[\UCactor] Presiona el botón \IUbutton{Cancelar}.
			\UCpaso Continúa en el paso \ref{paso:CUC1_ObtieneDatosUsuario} del \UCref{CUC1} .
\end{UCtrayectoriaA}

\begin{UCtrayectoriaA}{B}{Datos del proyecto incompletos}{No se puede enviar el proyecto porque hace falta información del proyecto.}
			\UCpaso Muestra el mensaje \ref{MSG_RN59}.
			\UCpaso Continúa en el paso 1.
\end{UCtrayectoriaA}

%-------------------------------------- TERMINA descripción del caso de uso.
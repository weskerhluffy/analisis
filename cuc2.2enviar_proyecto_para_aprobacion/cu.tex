%------------------------------------- Enviar Proyecto para aprobación

\begin{UseCase}{CUC2.2}{Enviar Proyecto para aprobación}{Envía un proyecto para su aprobación.}
		\UCitem{Versión}{1.1}
		\UCitem{Actor(es)}{Coordinador.}
		UCitem{Propósito}{Enviar un proyecto para ser evaluado.}
		\UCitem{Resumen}{Permite enviar un proyecto al gerente del coordinador que registró el proyecto para que lo evalúe y pueda ser aprobado o no aprobado.}
		\UCitem{Entradas}{Ninguna.}
		\UCitem{Salidas}{Mensaje de confirmación de modificación del envío del proyecto.}
		\UCitem{Precondiciones}{
			\begin{itemize}
			 	\item Haber pasado por el \UCref{CU2}.
			\end{itemize}
		}
		\UCitem{Postcondiciones}{El proyecto pasa al estado de ``Revisión''.}
		\UCitem{Autor}{Jessica Ramos} 
		\UCitem{Referencias}{CU-P2-061011}
		\UCitem{Tipo}{Secundario. Viene de \UCref{CUC2}}
		\UCitem{Módulo}{Coordinación}
\end{UseCase}
	
	
\begin{UCtrayectoria}{Principal}
		\UCpaso[\UCactor] Oprime el botón \IUbutton{Enviar Proyecto} de la pantalla \IUref{IU}{Planeación}%TODO:Corregir ID y nombre de la pantalla
		%TODO:Corregir Agregar todas la validaciones 
		\UCpaso Se muestra la pantalla \IUref{IUEnviarProyectoAprobacion}{Enviar Proyecto para Aprobación}
		\UCpaso[\UCactor] Oprime el botón \IUbutton{Aceptar}. \Trayref{A}
		\UCpaso Verifica que se cumpla la regla de negocio \BRref{RN59}. \Trayref{B}
		\UCpaso Cambia el proyecto del estado de ``Edición'' al estado de ``Revisión''.
\end{UCtrayectoria}

\begin{UCtrayectoriaA}{A}{Cancelar envío}{El usuario decide cancelar el envío del proyecto.}
			\UCpaso[\UCactor] Presiona el botón \IUbutton{Cancelar}.
			\UCpaso Continúa en el paso .
\end{UCtrayectoriaA}

\begin{UCtrayectoriaA}{B}{Datos del proyecto incompletos}{No se puede enviar el proyecto porque hace falta información del proyecto.}
			\UCpaso Muestra el mensaje \ref{MSG_RN59}.
			\UCpaso Continúa en el paso 1.
\end{UCtrayectoriaA}

%-------------------------------------- TERMINA descripción del caso de uso.
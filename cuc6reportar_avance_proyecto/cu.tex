
	\begin{UseCase}{CUC6}{Reportar avance del proyecto}{Reportar avance de uno o varios indicadores de una acción.}
		\UCitem{Versión}{1.5}
		\UCitem{Actor(es)}{Coordinador.}
		\UCitem{Propósito}{Reportar avance de una acción.}
		\UCitem{Resumen}{Se muestran los indicadores de la acción. \ref{dd:DatosIndicador}}
		\UCitem{Entradas}{La selección de los indicadores a los que se modificará el avance y su avance. }
		\UCitem{Salidas}{Mensaje \ref{MSG_CUC6}.}
		\UCitem{Precondiciones}{
			\begin{itemize}
				\item Debe existir al menos un indicador cuyo avance sea menor a la meta.
			\end{itemize}
		 }
		\UCitem{Postcondiciones}{Se actualizan los avances de los indicadores.}
		%\UCitem{Autor}{Ernesto Alvarado.}
		\UCitem{Referencias}{Ninguna.}
		\UCitem{Tipo}{Secundario. Viene del \UCref{CU2.1.2}}
		\UCitem{Módulo}{Coordinación}
	\end{UseCase}

	\begin{UCtrayectoria}{Principal}
		\UCpaso[\UCactor] Selecciona algun proyecto del perfil del Coordinador.
		\UCpaso[\UCactor] Selecciona la opción \IUbutton{Avances} en la interfaz del perfil del Coordinador.
		\UCpaso Muestra la pantalla \IUref{IURevisarAvancesProyecto}{Revisar Avances de Proyecto} \label{paso:CU6reportarAvanceIndicador}. 
		\UCpaso[\UCactor] Selecciona alguna acción de la lista desplegada.
		\UCpaso[\UCactor] Presiona la opción \IUbutton{Reportar avance} de la pantalla \IUref{IURevisarAvancesProyecto}{Revisar Avances de Proyecto}.
		\UCpaso Muestra la pantalla \IUref{IUReportarAvances}{Reportar avance} \label{paso:CU6reportarAvanceIndicador}. \Trayref{A}, \Trayref{B}
		\UCpaso[\UCactor] Habilita la casilla ``Reportar avance`` del indicador que desee modificar. 
		\UCpaso Habilita y muestra una campo de texto.
		\UCpaso[\UCactor] El usuario introduce el avance del indicador que desee modificar. 
		\UCpaso[\UCactor] Presiona la opción \IUbutton{Aceptar}. 
		\UCpaso Valida la \BRref{RN71}.
	\end{UCtrayectoria}

	\begin{UCtrayectoriaA}{A}{Cancelar}{No desea modificar la estructura seleccionada.}
		\UCpaso[\UCactor] Indica que no desea continuar con el CU.
		\UCpaso Regresa al paso 1 del \UCref{CU2.1.2}.
	\end{UCtrayectoriaA}

	\begin{UCtrayectoriaA}{B}{No se reporto avance en algún indicador.}{El usuario no reporto ningún avance.}
		\UCpaso[\UCactor] Presiona la opción \IUbutton{Aceptar}.
		\UCpaso Muestra el mensaje (MSG-RN-71).\ref{MSG_RN71}.
		\UCpaso Regresa al paso \ref{paso:CU6reportarAvanceIndicador} de \UCref{CUC6}. 
	\end{UCtrayectoriaA}


	\begin{UseCase}{CUA5}{Gestionar Programas}{Se muestra un catálogo de Programas para su gestión, en el que se puede realizar Altas, Bajas y Cambios de los mismos.}
		\UCitem{Versión}{2.0}
		\UCitem{Actor(es)}{Administrador.}
		\UCitem{Propósito}{Mantener organizada la información con respecto a los Programas así como visualizar los Programas registrados.}
		\UCitem{Resumen}{Agregar, modificar y eliminar uno o varios Programas.}
		\UCitem{Entradas}{Ninguna.}
		\UCitem{Salidas}{Despliegue de los Programas registrados o el mensaje MSG5\ref{MSG5}.}
		\UCitem{Precondiciones}{Ninguna.}
		\UCitem{Postcondiciones}{Ninguna.}
		%\UCitem{Autor}{Saucedo Almazán Rodolfo. Ernesto Alvarado Gaspar.}
		\UCitem{Referencias}{SIDAM-BESP-P0-Especificación de Catálogos}
		\UCitem{Tipo}{Primario.}
		\UCitem{Módulo}{Administrador}
	\end{UseCase}
	
	% 1.- escriba solo una trayectoria principal
	% 2.- El actor es quien siempre inicia un CU
	% 3.- evite usar ``si ... entonces ...'' o ``mientras ...'' o ``para cada ...''
	% 4.- No olvide mencionar todas las verificaciones y c\'alculos realizados por el sistema
	% 5.- Evite mencionar palabras como: Tabla, BD, conexi\'on, etc.
	
	
	\begin{UCtrayectoria}{Principal}
		\UCpaso[\UCactor] Selecciona \textit{Gestión de Programas} en el men\'u \IUref{IUMenuAdministrador}{Menú para Administrador.}
		\UCpaso Busca los Programas registrados.\label{paso:CUS1buscarProgramas1N}\Trayref{A}
		\UCpaso Muestra la pantalla \IUref{IUGestProgramas1N}{Gestión de Programa} donde se muestran los datos de los Programas1N registrados y las opciones para Agregar, Modificar y Eliminar Programas.\UCExtensionPoint{CUA5.1}{Agregar Programa} \UCExtensionPoint{CUA5.2}{Modificar Programa} \UCExtensionPoint{CUA5.3}{Eliminar Programa}
	\end{UCtrayectoria}

         \begin{UCtrayectoriaA}{A}{No hay Programas}{No se encuentra ningún Programas registrado.}
		\UCpaso Muestra el mensaje (MSG-5) notificando al Usuario que no se han encontrado registros de Programa de primer nivel.\ref{MSG5}
	\end{UCtrayectoriaA}
	  
%--------- Agregar Programa1N
	\begin{UseCase}{CUA5.1}{Agregar Programa}{El Administrador General registra un nuevo Programa con sus respectivos datos, posteriormente, el programa registrado se muestra en el sistema.}
		\UCitem{Versión}{2.0}
		\UCitem{Actor(es)}{Administrador.}
		\UCitem{Propósito}{Agregar un nuevo Programa.}
		\UCitem{Resumen}{Se agrega un Programa registrando sus datos correspondientes.}
		\UCitem{Entradas}{Datos del Programa \ref{dd:Pp1n}.}
		\UCitem{Salidas}{Nuevo registro del Programa o el mensaje MSG-4\ref{MSG4}.}
		\UCitem{Precondiciones}{Ninguna.}
		\UCitem{Postcondiciones}{Nuevo Programa registrado. El Programa está disponible para ser utilizado.}
		%\UCitem{Autor}{Omar Juarez. Ernesto Alvarado Gaspar}
		\UCitem{Referencias}{SIDAM-BESP-P0-Especificación de Catálogos}
		\UCitem{Tipo}{Secundario. Viene del \UCref{CUA5}}
		\UCitem{Módulo}{Administrador}
	\end{UseCase}

	\begin{UCtrayectoria}{Principal}
		\UCpaso[\UCactor] Oprime el bot\'on \IUbutton{Nuevo Programa} en la pantalla \IUref{IUGestProgramas1N}{Gestión de Programa}.
		\UCpaso Busca los Programas registrados \label{paso:CUS1.1buscaProgramas1N}.
		\UCpaso Muestra la pantalla \IUref{IUAgregarPrograma1N}{Agregar Programa}.
		\UCpaso Verifica si un Programa ha sido marcado como programa sectorial según la regla de negocio \BRref{RN42}.
		\UCpaso Desactiva la casilla de verificaci\'on de programa sectorial. \Trayref{A}
		\UCpaso [\UCactor] Ingresa los datos del Programa.\label{paso:CUS1.1ingresaDatosPrograma1N} \Trayref{B}
		\UCpaso [\UCactor] Oprime el bot\'on \IUbutton{Aceptar}.
		\UCpaso Verifica que se cumpla la regla de negocio \BRref{RN2}.\Trayref{C}
		\UCpaso Verifica que los datos correspondan con la definici\'on en el diccionario de datos \ref{dd:Pp1n}. \Trayref{D}
		\UCpaso Verifica que se cumplan las reglas de negocio \BRref{RN1}. \Trayref{E}
		\UCpaso Verifica que se cumpla la regla de negocio \BRref{RN32}. \Trayref{F} 
		\UCpaso Verifica que se cumpla la regla de negocio \BRref{RN7}. \Trayref{G} 
		\UCpaso Registra el Programa1N.
                \UCpaso Muestra mensaje (MSG-4) de operacion exitosa.\ref{MSG4}
		\UCpaso Contin\'ua en el paso \ref{paso:CUS1buscarProgramas1N} del \UCref{CUA5}.
	\end{UCtrayectoria}

	\begin{UCtrayectoriaA}{A}{Programa sectorial no marcado}{Ning\'un Programa ha sido marcado como programa sectorial.}
		\UCpaso Activa la casilla de verificaci\'on de programa sectorial.
		\UCpaso Contin\'ua en el paso \ref{paso:CUS1.1ingresaDatosPrograma1N} del \UCref{CUA5.2}.
	\end{UCtrayectoriaA}
	
	\begin{UCtrayectoriaA}{B}{Cancelar operaci\'on}{El usuario abandona el Caso de Uso.}
		\UCpaso[\UCactor] Decide ya no agregar un nuevo Programa.
		\UCpaso[\UCactor] Oprime el bot\'on \IUbutton{Cancelar}.
			\UCpaso Contin\'ua en el paso \ref{paso:CUS1buscarProgramas1N} del \UCref{CUA5}.
	\end{UCtrayectoriaA}

	\begin{UCtrayectoriaA}{C}{Datos incompletos}{Algunos datos del Programa no han sido capturados.}
			\UCpaso Muestra el mensaje (MSG-1) indicando al usuario que los datos est\'an incompletos.\ref{MSG1}
			\UCpaso Contin\'ua en el paso \ref{paso:CUS1.1ingresaDatosPrograma1N} del \UCref{CUA5.1}.
	\end{UCtrayectoriaA}
		
	\begin{UCtrayectoriaA}{D}{Error de datos}{Los datos del Programa no cumplen con lo especificado en el diccionario de datos.}
			\UCpaso Muestra el mensaje (MSG-2) indicando que los datos ingresados son incorrectos.\ref{MSG2}
			\UCpaso Contin\'ua en el paso \ref{paso:CUS1.1ingresaDatosPrograma1N} del \UCref{CUA5.1}.
	\end{UCtrayectoriaA}

	\begin{UCtrayectoriaA}{E}{Nombre repetido}{El nombre que se desea asignar al Programa ya se encuentra registrado.}
			\UCpaso Muestra el mensaje (MSG-3) indicando al usuario que el nombre ya ha sido registrado.\ref{MSG3}
			\UCpaso Contin\'ua en el paso \ref{paso:CUS1.1ingresaDatosPrograma1N} del \UCref{CUA5.1}.
	\end{UCtrayectoriaA}

\begin{UCtrayectoriaA}{E}{Periodo mal definido}{El periodo que se pretende dar al nuevo programa no esta bien definido.}
		\UCpaso Muestra el mensaje (MSG-RN-32).\ref{MSG_RN32}.
		\UCpaso Continua en el paso \ref{paso:CUS1.1ingresaDatosPrograma1N} del \UCref{CUA5.1}.
	\end{UCtrayectoriaA}

	\begin{UCtrayectoriaA}{F}{Periodo invalido}{El periodo que se pretende dar al nuevo programa es invalido.}
		\UCpaso Muestra el mensaje (MSG-RN-7).\ref{MSG_RN7}.
		\UCpaso Continua en el paso \ref{paso:CUS1.1ingresaDatosPrograma1N} del \UCref{CUA5.1}.
	\end{UCtrayectoriaA}

%--------- Modificar Programa1N
	\begin{UseCase}{CUA5.2}{ModificarPrograma1N}{El Administrador General selecciona un Programa registrado en el sistema para modificar sus datos y al realizar dicha operación, se actualiza el sistema mostrando los nuevos cambios en el programa seleccionado.}
			\UCitem{Versi\'on}{2.0}
			\UCitem{Actor(es)}{Administrador.}
			\UCitem{Prop\'osito}{Modificar los datos de un Programa y actualizar su registro en el sistema.}
			\UCitem{Resumen}{El sistema muestra los datos de un Programa seleccionado para la modificaci\'on de sus datos.}
			\UCitem{Entradas}{Identificador del Programa seleccionado. Nuevos datos para el Programa1N seleccionado \ref{dd:Pp1n}.}
			\UCitem{Salidas}{Datos del Programa seleccionado \ref{dd:Pp1n} o el mensaje \ref{MSG4}.}
			\UCitem{Precondiciones}{Que exista el registro del Programa seleccionado}
			\UCitem{Postcondiciones}{Se actualizan los datos del Programa seleccionado.}
			%\UCitem{Autor}{Omar Juarez. Ernesto Alvarado Gaspar}
			\UCitem{Referencias}{SIDAM-BESP-P0-Especificaci\'on de Cat\'alogos}
			\UCitem{Tipo}{Secundario. Viene del \UCref{CUA5}}
			\UCitem{M\'odulo}{Administrador}
	\end{UseCase}

	\begin{UCtrayectoria}{Principal}
                       \UCpaso[\UCactor] Oprime el bot\'on \IUbutton{\includegraphics[scale=0.1]{images/icons/editar.png}} del registro seleccionado de la pantalla \IUref{IUGestProgramas1N}{Gestión de Programa}.
                         \UCpaso [\UCactor] Busca el Programa a modificar.
			\UCpaso Muestra la pantalla \IUref{IUModificarPrograma1N}{Modificar Programa} con los datos del Programa seleccionado.
			\UCpaso Verifica que el Programa seleccionado no ha sido marcado como programa sectorial.\Trayref{A}
			\UCpaso [\UCactor] Ingresa los datos del Programa.\label{paso:CUS1.2ingresaDatosPrograma1N}
			\UCpaso [\UCactor] Oprime el bot\'on \IUbutton{Aceptar}.
			\UCpaso Verifica que se cumpla la regla de negocio \BRref{RN2}.\Trayref{B}
			\UCpaso Verifica que los datos est\'en de acuerdo al diccionario de datos \ref{dd:Pp1n}. \Trayref{C}
			\UCpaso Verifica que se cumplan las reglas de negocio\BRref{RN42}. \Trayref{D}
			\UCpaso Verifica que se cumpla la regla de negocio \BRref{RN32}. \Trayref{E} 
			\UCpaso Verifica que se cumpla la regla de negocio \BRref{RN7}. \Trayref{F} 
			\UCpaso Actualiza el registro del Programa.\label{paso:CUS1.2ActualizaDatosPrograma1N}
                        \UCpaso Muestra mensaje (MSG-4) de operación exitosa.\ref{MSG4}
			\UCpaso Contin\'ua en el paso \ref{paso:CUS1buscarProgramas1N} del \UCref{CUA5}.
	\end{UCtrayectoria}

	\begin{UCtrayectoriaA}{A}{Programa marcado como programa sectorial}{El Programa1N seleccionado est\'a marcado como programa sectorial.}
		\UCpaso Se desactiva la casilla de verificaci\'on de programa sectorial.
		\UCpaso Contin\'ua en el paso \ref{paso:CUS1.2ingresaDatosPrograma1N} del \UCref{CUA5.2}.
	\end{UCtrayectoriaA}

	\begin{UCtrayectoriaA}{A}{Cancelar operaci\'on}{El usuario abandona el Caso de Uso.}
			\UCpaso[\UCactor] Decide ya no modificar los datos del Programa.
			\UCpaso[\UCactor] Oprime el bot\'on \IUbutton{Cancelar}.
			\UCpaso Contin\'ua en el paso \ref{paso:CUS1buscarProgramas1N} del \UCref{CUA5}.
	\end{UCtrayectoriaA}
		
	\begin{UCtrayectoriaA}{B}{Datos incompletos}{Algunos datos del Programa no han sido capturados.}
			\UCpaso Muestra el mensaje (MSG-1) indicando al usuario que los datos est\'an incompletos.\ref{MSG1}
			\UCpaso Contin\'ua en el paso \ref{paso:CUS1.1ingresaDatosPrograma1N} del \UCref{CUA5.1}.
	\end{UCtrayectoriaA}
		
	\begin{UCtrayectoriaA}{C}{Error de datos}{Los datos del Programa no cumplen con lo especificado en el diccionario de datos.}
			\UCpaso Muestra el mensaje (MSG-2) indicando que los datos ingresados son incorrectos.\ref{MSG2}
			\UCpaso Contin\'ua en el paso \ref{paso:CUS1.1ingresaDatosPrograma1N} del \UCref{CUA5.1}.
	\end{UCtrayectoriaA}

	\begin{UCtrayectoriaA}{D}{Cambio de programa sectorial}{El usuario marca a un nuevo Programa como programa sectorial}
                        \UCpaso El Programa marcado anteriormente como programa sectorial se desmarca.
			\UCpaso Marca el Programa seleccionado como programa sectorial.
			\UCpaso Contin\'ua en el paso \ref{paso:CUS1.2ActualizaDatosPrograma1N} del \UCref{CUA5.2}.
	\end{UCtrayectoriaA}

	\begin{UCtrayectoriaA}{E}{Periodo mal definido}{El periodo que se pretende dar al nuevo programa no esta bien definido.}
		\UCpaso Muestra el mensaje (MSG-RN-32).\ref{MSG_RN32}.
		\UCpaso Continua en el paso \ref{paso:CUS1.1ingresaDatosPrograma1N} del \UCref{CUA5.1}.
	\end{UCtrayectoriaA}

	\begin{UCtrayectoriaA}{F}{Periodo invalido}{El periodo que se pretende dar al nuevo programa es invalido.}
		\UCpaso Muestra el mensaje (MSG-RN-7).\ref{MSG_RN7}.
		\UCpaso Continua en el paso \ref{paso:CUS1.1ingresaDatosPrograma1N} del \UCref{CUA5.1}.
	\end{UCtrayectoriaA}

%--------- Eliminar Programa1N
	\begin{UseCase}{CUA5.3}{Eliminar Programa}{El Administrador General selecciona y elimina un Programa registrado,posteriormente se actualiza el sistema y realiza los cambios respectivos a dicha eliminación.}
			\UCitem{Versi\'on}{2.0}
			\UCitem{Actor(es)}{Administrador.}
			\UCitem{Prop\'osito}{Eliminar un Programa.}
			\UCitem{Resumen}{El sistema muestra los Programas registrados y permite seleccionar un Programa para eliminarlo.}
			\UCitem{Entradas}{Identificador del Programa seleccionado.}
			\UCitem{Salidas}{Datos del Programa seleccionado \ref{dd:Pp1n} o el mensaje\ref{MSG4}.}
			\UCitem{Precondiciones}{Que exista el registro del Programa seleccionado.}
			\UCitem{Postcondiciones}{El Programa se elimina de los registros.}
			%\UCitem{Autor}{Saucedo Almaz\'an Rodolfo.}
			\UCitem{Referencias}{SIDAM-BESP-P0-Especificaci\'on de Cat\'alogos}
			\UCitem{Tipo}{Secundario. Viene del \UCref{CUA5}}
			\UCitem{M\'odulo}{Administrador}
	\end{UseCase}

	\begin{UCtrayectoria}{Principal}
                        \UCpaso[\UCactor] Oprime el bot\'on \IUbutton{\includegraphics[scale=0.1]{images/icons/eliminar.png}} del registro seleccionado de la pantalla \IUref{IUGestProgramas1N}{Gestión de Programa}.
                        \UCpaso [\UCactor] Busca el Programa a eliminar.
			\UCpaso Muestra la pantalla \IUref{IUEliminarPrograma1N}{Eliminar Programa} con los datos del Programa seleccionado.
			\UCpaso [\UCactor] Oprime el bot\'on \IUbutton{Aceptar}. \Trayref{A}
                        \UCpaso Verifica que se cumplan la regla de negocio \BRref{RN13}\Trayref{B} 
			\UCpaso Verifica que se cumplan la regla de negocio \BRref{RN45}\Trayref{C} 
			\UCpaso El Programa seleccionado se elimina.
                        \UCpaso Muestra mensaje (MSG-4) de operacion exitosa.\ref{MSG4}
			\UCpaso Contin\'ua en el paso \ref{paso:CUS1buscarProgramas1N} del \UCref{CUA5}.
	\end{UCtrayectoria}

		\begin{UCtrayectoriaA}{A}{Cancelar operaci\'on}{El usuario abandona el Caso de Uso.}
			\UCpaso[\UCactor] Decide ya no eliminar el Programa. \label{Datos_Asoc_Equipo}
			\UCpaso[\UCactor] Oprime el bot\'on \IUbutton{Cancelar}.
			\UCpaso Contin\'ua en el paso \ref{paso:CUS1buscarProgramas1N} del \UCref{CUA5}.
		\end{UCtrayectoriaA}

                \begin{UCtrayectoriaA}{B}{Programa con proyectos asociados}{El Programa tiene proyectos asociados.}
			\UCpaso Muestra mensaje de error (MSG-6G1).\ref{MSG-6G1} \label{Datos_Asoc_Equipo}
			\UCpaso[\UCactor] Oprime el bot\'on \IUbutton{Cancelar}.
			\UCpaso Contin\'ua en el paso \ref{paso:CUS1buscarProgramas1N} del \UCref{CUA5}.
		\end{UCtrayectoriaA}
		
		\begin{UCtrayectoriaA}{C}{Programa marcado como sectorial}{El Programa est\'a marcado como programa sectorial.}
			\UCpaso Muestra el mensaje de error (MSG-6G1.1).\ref{MSG-6G1.1}
			\UCpaso[\UCactor] Oprime el bot\'on \IUbutton{Cancelar}.
			\UCpaso Contin\'ua en el paso \ref{paso:CUS1buscarProgramas1N} del \UCref{CUA5}.
		\end{UCtrayectoriaA}
%-------------------------------------- TERMINA descripci\'on del caso de uso.